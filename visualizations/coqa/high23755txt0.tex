\documentclass[11pt,a4paper, onecolumn]{article}
\usepackage{times}
\usepackage{latexsym}
\usepackage{url}
\usepackage{textcomp}
\usepackage{bbm}
\usepackage{amsmath}
\usepackage{booktabs}
\usepackage{tabularx}
\usepackage{graphicx}
\usepackage{dialogue}
\usepackage{mathtools}
\usepackage{hyperref}
%\hypersetup{draft}

\usepackage{multirow}
\usepackage{mdframed}
\usepackage{tcolorbox}

\usepackage{xcolor,pifont}
%\newcommand{\cmark}{\ding{51}}
%\newcommand{\xmark}{\ding{55}}

\setcounter{topnumber}{2}
\setcounter{bottomnumber}{2}
\setcounter{totalnumber}{4}
\renewcommand{\topfraction}{0.75}
\renewcommand{\bottomfraction}{0.75}
\renewcommand{\textfraction}{0.05}
\renewcommand{\floatpagefraction}{0.6}

\newcommand\cmark {\textcolor{green}{\ding{52}}}
\newcommand\xmark {\textcolor{red}{\ding{55}}}
\mdfdefinestyle{dialogue}{
    backgroundcolor=yellow!20,
    innermargin=5pt
}
\usepackage{amssymb}
\usepackage{soul}
\makeatletter

\begin{document}

\hspace{15pt}{\textbf{Section}:high23755.txt0\\}
\\ Context: A new study suggests that early exposure to germs strengthens the immune system. That means letting children get a little dirty might be good for their health later in life. The study involved laboratory mice. It found that adult mice raised in a germ-free environment were more likely to develop allergies, asthma and other autoimmune disorders. There are more than eighty disorders where cells that normally defend the body instead attack tissues and organs. Richard Blumberg,who led the study,is a professor at Harvard Medical School in Boston,Massachusetts. He says,in 1989,medical researchers who sought to explain these diseases, first discovered that the increasing use of antibacterial soaps and other products, especially early in life, could weaken immune systems. Now, Dr. Blumberg and his team have what is the first biological evidence to link early exposure to germs to stronger adult immune systems. They say this exposure could prevent the development of some autoimmune diseases. In the adult germ-free mice, they found that inflammation in the lungs and colon was caused by so-called killer T cells. These normally fight infection. But they became overactive and targeted healthy tissue--an autoimmune condition seen in asthma and a disease called ulcerative colitis . Dr. Blumberg says the mice raised in a normal environment did not have the same reaction. He says their immune systems had been ''educated'' by early exposure to germs. Rates of autoimmune disorders are rising worldwide, but mostly in wealthier, industrialized countries. According to Dr. Blumberg, it might be high time that people were warned to be more careful with the early use of antibiotics and the prescription from their doctors. Rob Dunn is a professor of ecology and evolutionary biology at North Carolina State University in Raleigh. He says the new study does not mean people should stop washing. ''Wash your hands , but don't do it with antibacterial soap. Let your kids play in a reasonable amount of dirt and get outside and get exposed to a diversity of things'', says Rob Dunn. CANNOTANSWER

\begin{figure}[t] \small \begin{tcolorbox}[boxsep=0pt,left=5pt,right=0pt,top=2pt,colback = yellow!5] \begin{dialogue}
 \small 
 \speak{Student}{\bf Are kids getting dirty a good thing? }
\speak{Teacher}\colorbox{pink!25}{$\hookrightarrow$}
\colorbox{red!25}{Yes,}
{ ``yes'' (A ) }
\\
\speak{Student}{\bf why? }
\speak{Teacher}\colorbox{pink!25}{$\hookrightarrow$}
{ ``early exposure to germs strengthens the immune system'' (early exposure to germs strengthens the immune system ) }
\\
\speak{Student}{\bf Where kids tested in the study? }
\speak{Teacher}\colorbox{pink!25}{$\hookrightarrow$}
\colorbox{red!25}{No,}
{ ``no'' (early exposure to germs strengthens the immune system ) }
\\
\speak{Student}{\bf What was? }
\speak{Teacher}\colorbox{pink!25}{$\hookrightarrow$}
{ ``laboratory mice'' (laboratory mice ) }
\\
\speak{Student}{\bf What was found? }
\speak{Teacher}\colorbox{pink!25}{$\hookrightarrow$}
{ ``adult mice raised in a germ-free environment were more likely to develop allergies, asthma and other autoimmune disorders'' (adult mice raised in a germ-free environment were more likely to develop allergies, asthma and other autoimmune disorders ) }
\\
\speak{Student}{\bf Who led the study? }
\speak{Teacher}\colorbox{pink!25}{$\hookrightarrow$}
{ ``Richard Blumberg'' (Richard Blumberg ) }
\\
\speak{Student}{\bf From where? }
\speak{Teacher}\colorbox{pink!25}{$\hookrightarrow$}
{ ``Boston'' (Boston ) }
\\
\speak{Student}{\bf At what school? }
\speak{Teacher}\colorbox{pink!25}{$\hookrightarrow$}
{ ``Harvard Medical School'' (Harvard Medical School ) }
 \end{dialogue}\end{tcolorbox}\end{figure}\begin{figure}[t] \small \begin{tcolorbox}[boxsep=0pt,left=5pt,right=0pt,top=2pt,colback = yellow!5] \begin{dialogue}
 \small 
 \speak{Student}{\bf Early exposure to what makes a stronger adult immune system? }
\speak{Teacher}\colorbox{pink!25}{$\hookrightarrow$}
{ ``germs'' (germs ) }
\\
\speak{Student}{\bf What type of enviornment were the mice raised in? }
\speak{Teacher}\colorbox{pink!25}{$\hookrightarrow$}
{ ``germ-free'' (germ-free ) }
\\
\speak{Student}{\bf Did they raise others in a germ envirnment? }
\speak{Teacher}\colorbox{pink!25}{$\hookrightarrow$}
\colorbox{red!25}{Yes,}
{ ``yes'' (germ-free ) }
\\
\speak{Student}{\bf Was there a difference in the two? }
\speak{Teacher}\colorbox{pink!25}{$\hookrightarrow$}
\colorbox{red!25}{Yes,}
{ ``yes'' (germ-free ) }
\\
\speak{Student}{\bf Was it the same reaction? }
\speak{Teacher}\colorbox{pink!25}{$\hookrightarrow$}
\colorbox{red!25}{No,}
{ ``no'' (germ-free ) }
\\
\speak{Student}{\bf What caused inflammation in the lungs? }
\speak{Teacher}\colorbox{pink!25}{$\hookrightarrow$}
{ ``killer T cells'' (killer T cells ) }
\\
\speak{Student}{\bf So, should we stop washing our kids? }
\speak{Teacher}\colorbox{pink!25}{$\hookrightarrow$}
\colorbox{red!25}{No,}
{ ``no'' (killer T cells ) }
\\
\speak{Student}{\bf Who says? }
\speak{Teacher}\colorbox{pink!25}{$\hookrightarrow$}
{ ``Rob Dunn'' (Rob Dunn ) }
 \end{dialogue}\end{tcolorbox}\end{figure}\begin{figure}[t] \small \begin{tcolorbox}[boxsep=0pt,left=5pt,right=0pt,top=2pt,colback = yellow!5] \begin{dialogue}
 \small 
 \speak{Student}{\bf What should we do? }
\speak{Teacher}\colorbox{pink!25}{$\hookrightarrow$}
{ ``Wash your hands , but don't do it with antibacterial soap'' (Wash your hands , but don't do it with antibacterial soap ) }
\\
\speak{Student}{\bf and what else? }
\speak{Teacher}\colorbox{pink!25}{$\hookrightarrow$}
{ ``Let kids play in a reasonable amount of dirt'' (Let your kids play in a reasonable amount of dirt ) }
\\
\speak{Student}{\bf So they are exposed to what? }
\speak{Teacher}\colorbox{pink!25}{$\hookrightarrow$}
{ ``a diversity of things'' (diversity of things ) }
\\
\speak{Student}{\bf What could weaken the immune system? }
\speak{Teacher}\colorbox{pink!25}{$\hookrightarrow$}
{ ``increasing use of antibacterial soaps'' (increasing use of antibacterial soaps ) }
\\
 \end{dialogue}\end{tcolorbox}\end{figure}

\end{document}

