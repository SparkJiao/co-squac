\documentclass[11pt,a4paper, onecolumn]{article}
\usepackage{times}
\usepackage{latexsym}
\usepackage{url}
\usepackage{textcomp}
\usepackage{bbm}
\usepackage{amsmath}
\usepackage{booktabs}
\usepackage{tabularx}
\usepackage{graphicx}
\usepackage{dialogue}
\usepackage{mathtools}
\usepackage{hyperref}
%\hypersetup{draft}

\usepackage{multirow}
\usepackage{mdframed}
\usepackage{tcolorbox}

\usepackage{xcolor,pifont}
%\newcommand{\cmark}{\ding{51}}
%\newcommand{\xmark}{\ding{55}}

\setcounter{topnumber}{2}
\setcounter{bottomnumber}{2}
\setcounter{totalnumber}{4}
\renewcommand{\topfraction}{0.75}
\renewcommand{\bottomfraction}{0.75}
\renewcommand{\textfraction}{0.05}
\renewcommand{\floatpagefraction}{0.6}

\newcommand\cmark {\textcolor{green}{\ding{52}}}
\newcommand\xmark {\textcolor{red}{\ding{55}}}
\mdfdefinestyle{dialogue}{
    backgroundcolor=yellow!20,
    innermargin=5pt
}
\usepackage{amssymb}
\usepackage{soul}
\makeatletter

\begin{document}

\hspace{15pt}{\textbf{Section}:high3666.txt0\\}
\\ Context: History is full of examples of leaders joining together to meet common goals. But rarely have two leaders worked together with such friendship and cooperation as American President Franklin Roosevelt and British Prime Minister Winston Churchill. They both were born in wealthy families and were active in politics for many years. Both men loved the sea and the navy,history and nature. Roosevelt and Churchill first met when they were lowerlevel officials in World War One. But neither man remembered much about that meeting. However,as they worked together during the Second World War they came to like and trust each other. Roosevelt and Churchill exchanged more than one thousand seven hundred letters and messages during five and a half years. They met many times,at large national gatherings and in private talks. But the closeness of their friendship might be seen best in a story told by one of Roosevelt's close advisors,Harry Hopkins. Hopkins remembered how Churchill was visiting Roosevelt at the White House one day. Roosevelt went into Churchill's room in the morning to say hello. But the president was shocked to see Churchill coming from the washing room with no clothes at all. Roosevelt immediately apologized to the British leader for seeing him naked. But Churchill reportedly said: ''The Prime Minister of Great Britain has nothing to hide from the president of the United States.'' And then both men laughed. The United States and Great Britain were only two of several nations that joined together in the war to resist Hitler and his Allies. In January,1942,twentysix of these nations signed an agreement promising to fight for peace,religious freedom,human rights,and justice. The three major Allies,however,were the most important for the war effort: the United States,Britain,and the Soviet Union. Yet,Churchill and Roosevelt disagreed about when to attack Hitler in western Europe. And Churchill resisted Roosevelt's suggestions that Britain give up some of its colonies. But in general,the friendship between Roosevelt and Churchill,and between the United States and Britain led the two nations to cooperate closely. CANNOTANSWER

\begin{figure}[t] \small \begin{tcolorbox}[boxsep=0pt,left=5pt,right=0pt,top=2pt,colback = yellow!5] \begin{dialogue}
 \small 
 \speak{Student}{\bf Which two leaders worked together? }
\speak{Teacher}\colorbox{pink!25}{$\hookrightarrow$}
{ ``Franklin Roosevelt and Winston Churchill'' (Franklin Roosevelt and British Prime Minister Winston Churchill ) }
\\
\speak{Student}{\bf when did they first meet? }
\speak{Teacher}\colorbox{pink!25}{$\hookrightarrow$}
{ ``World War One'' (World War One ) }
\\
\speak{Student}{\bf did they remember it? }
\speak{Teacher}\colorbox{pink!25}{$\hookrightarrow$}
\colorbox{red!25}{No,}
{ ``no'' (World War One ) }
\\
\speak{Student}{\bf when did they remember each other? }
\speak{Teacher}\colorbox{pink!25}{$\hookrightarrow$}
{ ``Second World War'' (Second World War ) }
\\
\speak{Student}{\bf Did they like each other? }
\speak{Teacher}\colorbox{pink!25}{$\hookrightarrow$}
\colorbox{red!25}{Yes,}
{ ``yes'' (Second World War ) }
\\
\speak{Student}{\bf how many letters exchanged? }
\speak{Teacher}\colorbox{pink!25}{$\hookrightarrow$}
{ ``more than one thousand seven hundred letters'' (more than one thousand seven hundred letters ) }
\\
\speak{Student}{\bf two nations that joined the war? }
\speak{Teacher}\colorbox{pink!25}{$\hookrightarrow$}
{ ``United States and Great Britain'' (United States and Great Britain ) }
\\
\speak{Student}{\bf how many signed agreement? }
\speak{Teacher}\colorbox{pink!25}{$\hookrightarrow$}
{ ``26'' (twentysix ) }
 \end{dialogue}\end{tcolorbox}\end{figure}\begin{figure}[t] \small \begin{tcolorbox}[boxsep=0pt,left=5pt,right=0pt,top=2pt,colback = yellow!5] \begin{dialogue}
 \small 
 \speak{Student}{\bf how many allies? }
\speak{Teacher}\colorbox{pink!25}{$\hookrightarrow$}
{ ``Three'' (three ) }
\\
\speak{Student}{\bf who? }
\speak{Teacher}\colorbox{pink!25}{$\hookrightarrow$}
{ ``United States, Britain, and Soviet Union'' (United States,Britain,and the Soviet Union ) }
\\
\speak{Student}{\bf Did Churchill visit Roosevelt at White house? }
\speak{Teacher}\colorbox{pink!25}{$\hookrightarrow$}
\colorbox{red!25}{Yes,}
{ ``yes'' (United States,Britain,and the Soviet Union ) }
\\
\speak{Student}{\bf what did he do? }
\speak{Teacher}\colorbox{pink!25}{$\hookrightarrow$}
{ ``came from the washing room with no clothes at all'' (from the washing room with no clothes at all ) }
\\
\speak{Student}{\bf who was Roosevelt adviser? }
\speak{Teacher}\colorbox{pink!25}{$\hookrightarrow$}
{ ``Harry Hopkins'' (Harry Hopkins ) }
\\
\speak{Student}{\bf was he close to him? }
\speak{Teacher}\colorbox{pink!25}{$\hookrightarrow$}
\colorbox{red!25}{Yes,}
{ ``yes'' (Harry Hopkins ) }
\\
\speak{Student}{\bf did they disagree? }
\speak{Teacher}\colorbox{pink!25}{$\hookrightarrow$}
\colorbox{red!25}{Yes,}
{ ``yes'' (Harry Hopkins ) }
\\
\speak{Student}{\bf what about? }
\speak{Teacher}\colorbox{pink!25}{$\hookrightarrow$}
{ ``when to attack Hitler in western Europe'' (when to attack Hitler in western Europe ) }
 \end{dialogue}\end{tcolorbox}\end{figure}

\end{document}

