\documentclass[11pt,a4paper, onecolumn]{article}
\usepackage{times}
\usepackage{latexsym}
\usepackage{url}
\usepackage{textcomp}
\usepackage{bbm}
\usepackage{amsmath}
\usepackage{booktabs}
\usepackage{tabularx}
\usepackage{graphicx}
\usepackage{dialogue}
\usepackage{mathtools}
\usepackage{hyperref}
%\hypersetup{draft}

\usepackage{multirow}
\usepackage{mdframed}
\usepackage{tcolorbox}

\usepackage{xcolor,pifont}
%\newcommand{\cmark}{\ding{51}}
%\newcommand{\xmark}{\ding{55}}

\setcounter{topnumber}{2}
\setcounter{bottomnumber}{2}
\setcounter{totalnumber}{4}
\renewcommand{\topfraction}{0.75}
\renewcommand{\bottomfraction}{0.75}
\renewcommand{\textfraction}{0.05}
\renewcommand{\floatpagefraction}{0.6}

\newcommand\cmark {\textcolor{green}{\ding{52}}}
\newcommand\xmark {\textcolor{red}{\ding{55}}}
\mdfdefinestyle{dialogue}{
    backgroundcolor=yellow!20,
    innermargin=5pt
}
\usepackage{amssymb}
\usepackage{soul}
\makeatletter

\begin{document}

\hspace{15pt}{\textbf{Section}:high7339.txt0\\}
\\ Context: PITTSBURGH - For most people, snakes seem unpleasant or even threatening. But Howie Choset sees in their delicate movements a way to save lives. The 37-year-old Carnegie Mellon University professor has spent years developing snake-like robots he hopes will eventually slide through fallen buildings in search of victims trapped after natural disasters or other emergencies. Dan Kara is president of Robotics Trends, a Northboro, Mass.-based company that publishes an online industry magazine and runs robotics trade shows. He said there are other snake-like robots being developed, mainly at universities, but didn't know of one that could climb pipes. The Carnegie Mellon machines are designed to carry cameras and electronic sensors and can be controlled with a joystick . They move smoothly with the help of small electric motors, or servos, commonly used by hobbyists in model airplanes. Built from lightweight materials, the robots are about the size of a human arm or smaller. They can sense which way is up, but are only as good as their human operators, Choset added. Sam Stover, a search term manager with the Federal Emergency Management Agency based in Indiana, said snake-type robots would offer greater mobility than equipment currently available, such as cameras attached to extendable roles. ''It just allows us to do something we've not been able to do before,'' Stover said, ''We needed them yesterday.'' He said snifter dogs are still the best search tool for rescue workers, but that they can only be used effectively when workers have access to damaged building. Stover, among the rescue workers who handled the aftermath of Hurricane Katrina, said snake robots would have helped rescuers search flooded houses in that disaster. Choset said the robots may not be ready for use for another five to ten years, depending on funding. CANNOTANSWER

\begin{figure}[t] \small \begin{tcolorbox}[boxsep=0pt,left=5pt,right=0pt,top=2pt,colback = yellow!5] \begin{dialogue}
 \small 
 \speak{Student}{\bf Who uses snakes to save lives? }
\speak{Teacher}\colorbox{pink!25}{$\hookrightarrow$}
{ ``Howie Choset'' (Howie Choset ) }
\\
\speak{Student}{\bf How old is he? }
\speak{Teacher}\colorbox{pink!25}{$\hookrightarrow$}
{ ``37'' (he ) }
\\
\speak{Student}{\bf Does he teach? }
\speak{Teacher}\colorbox{pink!25}{$\hookrightarrow$}
\colorbox{red!25}{Yes,}
{ ``Yes'' (he ) }
\\
\speak{Student}{\bf Where? }
\speak{Teacher}\colorbox{pink!25}{$\hookrightarrow$}
{ ``Carnegie Mellon'' (Carnegie Mellon ) }
\\
\speak{Student}{\bf What does he make? }
\speak{Teacher}\colorbox{pink!25}{$\hookrightarrow$}
{ ``robots'' (robots ) }
\\
\speak{Student}{\bf Why? }
\speak{Teacher}\colorbox{pink!25}{$\hookrightarrow$}
{ ``To help victims'' (victims ) }
\\
\speak{Student}{\bf What is Robotics Trends?/ }
\speak{Teacher}\colorbox{pink!25}{$\hookrightarrow$}
{ ``a company'' (company ) }
\\
\speak{Student}{\bf What does it do? }
\speak{Teacher}\colorbox{pink!25}{$\hookrightarrow$}
{ ``publishes an online industry magazine'' (publishes an online industry magazine ) }
 \end{dialogue}\end{tcolorbox}\end{figure}\begin{figure}[t] \small \begin{tcolorbox}[boxsep=0pt,left=5pt,right=0pt,top=2pt,colback = yellow!5] \begin{dialogue}
 \small 
 \speak{Student}{\bf Where is it based? }
\speak{Teacher}\colorbox{pink!25}{$\hookrightarrow$}
{ ``Northboro'' (Northboro, ) }
\\
\speak{Student}{\bf Which state? }
\speak{Teacher}\colorbox{pink!25}{$\hookrightarrow$}
{ ``Massachusetts'' (a ) }
\\
\speak{Student}{\bf Does it have a president? }
\speak{Teacher}\colorbox{pink!25}{$\hookrightarrow$}
\colorbox{red!25}{Yes,}
{ ``Yes'' (a ) }
\\
\speak{Student}{\bf Who is it? }
\speak{Teacher}\colorbox{pink!25}{$\hookrightarrow$}
{ ``Dan Kara'' (Dan Kara ) }
\\
\speak{Student}{\bf What are the machines controlled by? }
\speak{Teacher}\colorbox{pink!25}{$\hookrightarrow$}
{ ``a joystick'' (joystick ) }
\\
\speak{Student}{\bf Do they move smoothly? }
\speak{Teacher}\colorbox{pink!25}{$\hookrightarrow$}
\colorbox{red!25}{Yes,}
{ ``Yes'' (joystick ) }
\\
\speak{Student}{\bf What are servos? }
\speak{Teacher}\colorbox{pink!25}{$\hookrightarrow$}
{ ``small electric motors'' (small electric motors, ) }
\\
\speak{Student}{\bf Who else uses them? }
\speak{Teacher}\colorbox{pink!25}{$\hookrightarrow$}
{ ``hobbyists'' (hobbyists ) }
 \end{dialogue}\end{tcolorbox}\end{figure}\begin{figure}[t] \small \begin{tcolorbox}[boxsep=0pt,left=5pt,right=0pt,top=2pt,colback = yellow!5] \begin{dialogue}
 \small 
 \speak{Student}{\bf What are the robots built from? }
\speak{Teacher}\colorbox{pink!25}{$\hookrightarrow$}
{ ``lightweight materials'' (lightweight materials, ) }
\\
\speak{Student}{\bf How big are they? }
\speak{Teacher}\colorbox{pink!25}{$\hookrightarrow$}
{ ``No'' (lightweight materials, ) }
\\
\speak{Student}{\bf Do they know which way is up? }
\speak{Teacher}\colorbox{pink!25}{$\hookrightarrow$}
\colorbox{red!25}{Yes,}
{ ``Yes'' (lightweight materials, ) }
\\
 \end{dialogue}\end{tcolorbox}\end{figure}

\end{document}

