\documentclass[11pt,a4paper, onecolumn]{article}
\usepackage{times}
\usepackage{latexsym}
\usepackage{url}
\usepackage{textcomp}
\usepackage{bbm}
\usepackage{amsmath}
\usepackage{booktabs}
\usepackage{tabularx}
\usepackage{graphicx}
\usepackage{dialogue}
\usepackage{mathtools}
\usepackage{hyperref}
%\hypersetup{draft}

\usepackage{multirow}
\usepackage{mdframed}
\usepackage{tcolorbox}

\usepackage{xcolor,pifont}
%\newcommand{\cmark}{\ding{51}}
%\newcommand{\xmark}{\ding{55}}

\setcounter{topnumber}{2}
\setcounter{bottomnumber}{2}
\setcounter{totalnumber}{4}
\renewcommand{\topfraction}{0.75}
\renewcommand{\bottomfraction}{0.75}
\renewcommand{\textfraction}{0.05}
\renewcommand{\floatpagefraction}{0.6}

\newcommand\cmark {\textcolor{green}{\ding{52}}}
\newcommand\xmark {\textcolor{red}{\ding{55}}}
\mdfdefinestyle{dialogue}{
    backgroundcolor=yellow!20,
    innermargin=5pt
}
\usepackage{amssymb}
\usepackage{soul}
\makeatletter

\begin{document}

\hspace{15pt}{\textbf{Section}:Zen.txt0\\}
\\ Context: Zen () is a school of Mahayana Buddhism that originated in China during the Tang dynasty as Chan Buddhism. Zen school was strongly influenced by Taoism and developed as a distinct school of Chinese Buddhism. From China, Chan Buddhism spread south to Vietnam, northeast to Korea and east to Japan, where it became known as Japanese Zen. The term Zen is derived from the Japanese pronunciation of the Middle Chinese word 禪 (Chan) which traces its roots to the Indian practice of Dhyana (''meditation''). Zen emphasizes rigorous self-control, meditation-practice, insight into Buddha-nature, and the personal expression of this insight in daily life, especially for the benefit of others. As such, it de-emphasizes mere knowledge of sutras and doctrine and favors direct understanding through zazen and interaction with an accomplished teacher. The teachings of Zen include various sources of Mahayana thought, especially Yogachara, the Tathāgatagarbha sūtras and the Huayan school, with their emphasis on Buddha-nature, totality, and the Bodhisattva-ideal. The Prajñāpāramitā literature and, to a lesser extent, Madhyamaka have also been influential in the shaping of the ''paradoxical language'' of the Zen-tradition. The word ''Zen'' is derived from the Japanese pronunciation of the Middle Chinese word 禪 () (pinyin: ''Chán''), which in turn is derived from the Sanskrit word ''dhyāna'' (ध्यान ), which can be approximately translated as ''absorption'' or ''meditative state''. CANNOTANSWER

\begin{figure}[t] \small \begin{tcolorbox}[boxsep=0pt,left=5pt,right=0pt,top=2pt,colback = yellow!5] \begin{dialogue}
 \small 
 \speak{Student}{\bf Zen is part of which religion? }
\speak{Teacher}\colorbox{pink!25}{$\hookrightarrow$}
{ ``Buddhism'' (Buddhism ) }
\\
\speak{Student}{\bf What is it a school of? }
\speak{Teacher}\colorbox{pink!25}{$\hookrightarrow$}
{ ``Mahayana Buddhism'' (Mahayana Buddhism ) }
\\
\speak{Student}{\bf Where did that type come from? }
\speak{Teacher}\colorbox{pink!25}{$\hookrightarrow$}
{ ``China'' (China ) }
\\
\speak{Student}{\bf When? }
\speak{Teacher}\colorbox{pink!25}{$\hookrightarrow$}
{ ``During the Tang dynasty'' (during the Tang dynasty ) }
\\
\speak{Student}{\bf What influenced Zen? }
\speak{Teacher}\colorbox{pink!25}{$\hookrightarrow$}
{ ``Taoism'' (Taoism ) }
\\
\speak{Student}{\bf Name a country to which it spread from China. }
\speak{Teacher}\colorbox{pink!25}{$\hookrightarrow$}
{ ``Japan'' (Japan ) }
\\
\speak{Student}{\bf Name something the teachings of Zen emphasizes. }
\speak{Teacher}\colorbox{pink!25}{$\hookrightarrow$}
{ ``Meditation-practice'' (meditation-practice ) }
\\
\speak{Student}{\bf What else? }
\speak{Teacher}\colorbox{pink!25}{$\hookrightarrow$}
{ ``Rigorous self-control'' (rigorous self-control ) }
 \end{dialogue}\end{tcolorbox}\end{figure}\begin{figure}[t] \small \begin{tcolorbox}[boxsep=0pt,left=5pt,right=0pt,top=2pt,colback = yellow!5] \begin{dialogue}
 \small 
 \speak{Student}{\bf And what else? }
\speak{Teacher}\colorbox{pink!25}{$\hookrightarrow$}
{ ``Insight into Buddha's nature'' (insight into ) }
\\
\speak{Student}{\bf When should that insight be expressed? }
\speak{Teacher}\colorbox{pink!25}{$\hookrightarrow$}
{ ``All the time, especially for the benefit of others.'' (especially for the benefit of others. ) }
\\
\speak{Student}{\bf Does Zen value mere knowledge of doctrine? }
\speak{Teacher}\colorbox{pink!25}{$\hookrightarrow$}
\colorbox{red!25}{No,}
{ ``No'' (especially for the benefit of others. ) }
\\
\speak{Student}{\bf What does it value instead? }
\speak{Teacher}\colorbox{pink!25}{$\hookrightarrow$}
{ ``Direct understanding'' (direct understanding ) }
\\
\speak{Student}{\bf Through what? }
\speak{Teacher}\colorbox{pink!25}{$\hookrightarrow$}
{ ``Through ''zazen'' and interaction with an accomplished teacher.'' (through zazen and interaction with an accomplished teacher. ) }
\\
\speak{Student}{\bf What Chinese word does Zen trace back to? }
\speak{Teacher}\colorbox{pink!25}{$\hookrightarrow$}
{ ``禪'' (禪 ) }
\\
\speak{Student}{\bf What Sanskrit word does this come from in turn? }
\speak{Teacher}\colorbox{pink!25}{$\hookrightarrow$}
{ ``ध्यान'' ((ध्यान ) }
\\
\speak{Student}{\bf What is a meaning of this word? }
\speak{Teacher}\colorbox{pink!25}{$\hookrightarrow$}
{ ``''absorption'' or ''meditative state'''' (''absorption'' or ''meditative state''. ) }
 \end{dialogue}\end{tcolorbox}\end{figure}\begin{figure}[t] \small \begin{tcolorbox}[boxsep=0pt,left=5pt,right=0pt,top=2pt,colback = yellow!5] \begin{dialogue}
 \small 
 \speak{Student}{\bf Who practiced Dhyana? }
\speak{Teacher}\colorbox{pink!25}{$\hookrightarrow$}
{ ``Indians'' (( ) }
\\
\speak{Student}{\bf Is Zen present in Vietnam? }
\speak{Teacher}\colorbox{pink!25}{$\hookrightarrow$}
\colorbox{red!25}{Yes,}
{ ``Yes'' (( ) }
\\
\speak{Student}{\bf What about Cambodia? }
\speak{Teacher}\colorbox{pink!25}{$\hookrightarrow$}
{ ``unknown'' (CANNOTANSWER ) }
\\
\speak{Student}{\bf What is it called in Japan? }
\speak{Teacher}\colorbox{pink!25}{$\hookrightarrow$}
{ ``Japanese Zen'' (Japanese Zen ) }
\\
 \end{dialogue}\end{tcolorbox}\end{figure}

\end{document}

