\documentclass[11pt,a4paper, onecolumn]{article}
\usepackage{times}
\usepackage{latexsym}
\usepackage{url}
\usepackage{textcomp}
\usepackage{bbm}
\usepackage{amsmath}
\usepackage{booktabs}
\usepackage{tabularx}
\usepackage{graphicx}
\usepackage{dialogue}
\usepackage{mathtools}
\usepackage{hyperref}
%\hypersetup{draft}

\usepackage{multirow}
\usepackage{mdframed}
\usepackage{tcolorbox}

\usepackage{xcolor,pifont}
%\newcommand{\cmark}{\ding{51}}
%\newcommand{\xmark}{\ding{55}}

\setcounter{topnumber}{2}
\setcounter{bottomnumber}{2}
\setcounter{totalnumber}{4}
\renewcommand{\topfraction}{0.75}
\renewcommand{\bottomfraction}{0.75}
\renewcommand{\textfraction}{0.05}
\renewcommand{\floatpagefraction}{0.6}

\newcommand\cmark {\textcolor{green}{\ding{52}}}
\newcommand\xmark {\textcolor{red}{\ding{55}}}
\mdfdefinestyle{dialogue}{
    backgroundcolor=yellow!20,
    innermargin=5pt
}
\usepackage{amssymb}
\usepackage{soul}
\makeatletter

\begin{document}

\hspace{15pt}{\textbf{Section}:Packet switching7\\}
\\ Context: Both X.25 and Frame Relay provide connection-oriented operations. But X.25 does it at the network layer of the OSI Model. Frame Relay does it at level two, the data link layer. Another major difference between X.25 and Frame Relay is that X.25 requires a handshake between the communicating parties before any user packets are transmitted. Frame Relay does not define any such handshakes. X.25 does not define any operations inside the packet network. It only operates at the user-network-interface (UNI). Thus, the network provider is free to use any procedure it wishes inside the network. X.25 does specify some limited re-transmission procedures at the UNI, and its link layer protocol (LAPB) provides conventional HDLC-type link management procedures. Frame Relay is a modified version of ISDN's layer two protocol, LAPD and LAPB. As such, its integrity operations pertain only between nodes on a link, not end-to-end. Any retransmissions must be carried out by higher layer protocols. The X.25 UNI protocol is part of the X.25 protocol suite, which consists of the lower three layers of the OSI Model. It was widely used at the UNI for packet switching networks during the 1980s and early 1990s, to provide a standardized interface into and out of packet networks. Some implementations used X.25 within the network as well, but its connection-oriented features made this setup cumbersome and inefficient. Frame relay operates principally at layer two of the OSI Model. However, its address field (the Data Link Connection ID, or DLCI) can be used at the OSI network layer, with a minimum set of procedures. Thus, it rids itself of many X.25 layer 3 encumbrances, but still has the DLCI as an ID beyond a node-to-node layer two link protocol. The simplicity of Frame Relay makes it faster and more efficient than X.25. Because Frame relay is a data link layer protocol, like X.25 it does not define internal network routing operations. For X.25 its packet IDs---the virtual circuit and virtual channel numbers have to be correlated to network addresses. The same is true for Frame Relays DLCI. How this is done is up to the network provider. Frame Relay, by virtue of having no network layer procedures is connection-oriented at layer two, by using the HDLC/LAPD/LAPB Set Asynchronous Balanced Mode (SABM). X.25 connections are typically established for each communication session, but it does have a feature allowing a limited amount of traffic to be passed across the UNI without the connection-oriented handshake. For a while, Frame Relay was used to interconnect LANs across wide area networks. However, X.25 and well as Frame Relay have been supplanted by the Internet Protocol (IP) at the network layer, and the Asynchronous Transfer Mode (ATM) and or versions of Multi-Protocol Label Switching (MPLS) at layer two. A typical configuration is to run IP over ATM or a version of MPLS. <Uyless Black, X.25 and Related Protocols, IEEE Computer Society, 1991> <Uyless Black, Frame Relay Networks, McGraw-Hill, 1998> <Uyless Black, MPLS and Label Switching Networks, Prentice Hall, 2001> < Uyless Black, ATM, Volume I, Prentice Hall, 1995> CANNOTANSWER

\begin{figure}[t] \small \begin{tcolorbox}[boxsep=0pt,left=5pt,right=0pt,top=2pt,colback = yellow!5] \begin{dialogue}
 \small 
 \speak{Student}{\bf WHat do x.25 and Frame Relay both require }
\speak{Teacher}\colorbox{pink!25}{$\hookrightarrow$}
{ ``'' (connection-oriented operations. But X.25 does it at the network layer of the OSI Model. Frame Relay does it at level two, the data link layer ) }
\\
\speak{Student}{\bf What were X.25 and Frame relay used for  }
\speak{Teacher}\colorbox{pink!25}{$\hookrightarrow$}
{ ``'' (Frame Relay was used to interconnect LANs across wide area networks. However, X.25 and well as Frame Relay have been supplanted ) }
\\
\speak{Student}{\bf What supplanted Frame Relay and X.25  }
\speak{Teacher}\colorbox{pink!25}{$\hookrightarrow$}
{ ``'' (supplanted by the Internet Protocol (IP) at the network layer, and the Asynchronous Transfer Mode (ATM) and or versions of Multi-Protocol Label Switching ) }
\\
\speak{Student}{\bf What is a typical configuration }
\speak{Teacher}\colorbox{pink!25}{$\hookrightarrow$}
{ ``'' (A typical configuration is to run IP over ATM or a version of MPLS ) }
\\
\speak{Student}{\bf What does Frame Relay provide? }
\speak{Teacher}\colorbox{pink!25}{$\hookrightarrow$}
{ ``'' (CANNOTANSWER ) }
\\
\speak{Student}{\bf What is level two of the connection-orientated operation?  }
\speak{Teacher}\colorbox{pink!25}{$\hookrightarrow$}
{ ``'' (CANNOTANSWER ) }
\\
\speak{Student}{\bf What is the "hand shake" between communication parties? }
\speak{Teacher}\colorbox{pink!25}{$\hookrightarrow$}
{ ``'' (CANNOTANSWER ) }
\\
\speak{Student}{\bf What is the protocol suite? }
\speak{Teacher}\colorbox{pink!25}{$\hookrightarrow$}
{ ``'' (CANNOTANSWER ) }
\\
\speak{Student}{\bf Where was the packet switching used in the 1980s?  }
\speak{Teacher}\colorbox{pink!25}{$\hookrightarrow$}
{ ``'' (CANNOTANSWER ) }
\\
\speak{Student}{\bf Frame Relay requires a handshake from what? }
\speak{Teacher}\colorbox{pink!25}{$\hookrightarrow$}
{ ``'' (CANNOTANSWER ) }
\\
\speak{Student}{\bf What does Frame Relay's LAPB provide? }
\speak{Teacher}\colorbox{pink!25}{$\hookrightarrow$}
{ ``'' (CANNOTANSWER ) }
\\
\speak{Student}{\bf What does X.25's integrity operations concern? }
\speak{Teacher}\colorbox{pink!25}{$\hookrightarrow$}
{ ``'' (CANNOTANSWER ) }
\\
\speak{Student}{\bf When was Frame Relay's protocols used at UNI? }
\speak{Teacher}\colorbox{pink!25}{$\hookrightarrow$}
{ ``'' (CANNOTANSWER ) }
\\
\speak{Student}{\bf How is X.25 connection-oriented at layer two? }
\speak{Teacher}\colorbox{pink!25}{$\hookrightarrow$}
{ ``'' (CANNOTANSWER ) }
\\
 \end{dialogue}\end{tcolorbox}\end{figure}

\end{document}

