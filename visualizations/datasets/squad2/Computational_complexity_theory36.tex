\documentclass[11pt,a4paper, onecolumn]{article}
\usepackage{times}
\usepackage{latexsym}
\usepackage{url}
\usepackage{textcomp}
\usepackage{bbm}
\usepackage{amsmath}
\usepackage{booktabs}
\usepackage{tabularx}
\usepackage{graphicx}
\usepackage{dialogue}
\usepackage{mathtools}
\usepackage{hyperref}
%\hypersetup{draft}

\usepackage{multirow}
\usepackage{mdframed}
\usepackage{tcolorbox}

\usepackage{xcolor,pifont}
%\newcommand{\cmark}{\ding{51}}
%\newcommand{\xmark}{\ding{55}}

\setcounter{topnumber}{2}
\setcounter{bottomnumber}{2}
\setcounter{totalnumber}{4}
\renewcommand{\topfraction}{0.75}
\renewcommand{\bottomfraction}{0.75}
\renewcommand{\textfraction}{0.05}
\renewcommand{\floatpagefraction}{0.6}

\newcommand\cmark {\textcolor{green}{\ding{52}}}
\newcommand\xmark {\textcolor{red}{\ding{55}}}
\mdfdefinestyle{dialogue}{
    backgroundcolor=yellow!20,
    innermargin=5pt
}
\usepackage{amssymb}
\usepackage{soul}
\makeatletter

\begin{document}

\hspace{15pt}{\textbf{Section}:Computational complexity theory36\\}
\\ Context: The graph isomorphism problem is the computational problem of determining whether two finite graphs are isomorphic. An important unsolved problem in complexity theory is whether the graph isomorphism problem is in P, NP-complete, or NP-intermediate. The answer is not known, but it is believed that the problem is at least not NP-complete. If graph isomorphism is NP-complete, the polynomial time hierarchy collapses to its second level. Since it is widely believed that the polynomial hierarchy does not collapse to any finite level, it is believed that graph isomorphism is not NP-complete. The best algorithm for this problem, due to Laszlo Babai and Eugene Luks has run time 2O(√(n log(n))) for graphs with n vertices. CANNOTANSWER

\begin{figure}[t] \small \begin{tcolorbox}[boxsep=0pt,left=5pt,right=0pt,top=2pt,colback = yellow!5] \begin{dialogue}
 \small 
 \speak{Student}{\bf What is the problem attributed to defining if two finite graphs are isomorphic? }
\speak{Teacher}\colorbox{pink!25}{$\hookrightarrow$}
{ ``'' (The graph isomorphism problem ) }
\\
\speak{Student}{\bf What class is most commonly not ascribed to the graph isomorphism problem in spite of definitive determination? }
\speak{Teacher}\colorbox{pink!25}{$\hookrightarrow$}
{ ``'' (NP-complete ) }
\\
\speak{Student}{\bf What finite hierarchy implies that the graph isomorphism problem is NP-complete?  }
\speak{Teacher}\colorbox{pink!25}{$\hookrightarrow$}
{ ``'' (polynomial time hierarchy ) }
\\
\speak{Student}{\bf To what level would the polynomial time hierarchy collapse if graph isomorphism is NP-complete? }
\speak{Teacher}\colorbox{pink!25}{$\hookrightarrow$}
{ ``'' (second level ) }
\\
\speak{Student}{\bf Who are commonly associated with the algorithm typically considered the most effective with respect to finite polynomial hierarchy and graph isomorphism? }
\speak{Teacher}\colorbox{pink!25}{$\hookrightarrow$}
{ ``'' (Laszlo Babai and Eugene Luks ) }
\\
\speak{Student}{\bf What is the graph isolation problem?  }
\speak{Teacher}\colorbox{pink!25}{$\hookrightarrow$}
{ ``'' (CANNOTANSWER ) }
\\
\speak{Student}{\bf What is the problem attributed to defining if three finite graphs are isomorphic? }
\speak{Teacher}\colorbox{pink!25}{$\hookrightarrow$}
{ ``'' (CANNOTANSWER ) }
\\
\speak{Student}{\bf What is an important solved problem in complexity theory? }
\speak{Teacher}\colorbox{pink!25}{$\hookrightarrow$}
{ ``'' (CANNOTANSWER ) }
\\
\speak{Student}{\bf What infinite hierarchy implies that the graph isomorphism problem s NQ-complete? }
\speak{Teacher}\colorbox{pink!25}{$\hookrightarrow$}
{ ``'' (CANNOTANSWER ) }
\\
\speak{Student}{\bf What would the polynomial hierarchy collapse if graph isomorphism is NQ-complete? }
\speak{Teacher}\colorbox{pink!25}{$\hookrightarrow$}
{ ``'' (CANNOTANSWER ) }
\\
 \end{dialogue}\end{tcolorbox}\end{figure}

\end{document}

