\documentclass[11pt,a4paper, onecolumn]{article}
\usepackage{times}
\usepackage{latexsym}
\usepackage{url}
\usepackage{textcomp}
\usepackage{bbm}
\usepackage{amsmath}
\usepackage{booktabs}
\usepackage{tabularx}
\usepackage{graphicx}
\usepackage{dialogue}
\usepackage{mathtools}
\usepackage{hyperref}
%\hypersetup{draft}

\usepackage{multirow}
\usepackage{mdframed}
\usepackage{tcolorbox}

\usepackage{xcolor,pifont}
%\newcommand{\cmark}{\ding{51}}
%\newcommand{\xmark}{\ding{55}}

\setcounter{topnumber}{2}
\setcounter{bottomnumber}{2}
\setcounter{totalnumber}{4}
\renewcommand{\topfraction}{0.75}
\renewcommand{\bottomfraction}{0.75}
\renewcommand{\textfraction}{0.05}
\renewcommand{\floatpagefraction}{0.6}

\newcommand\cmark {\textcolor{green}{\ding{52}}}
\newcommand\xmark {\textcolor{red}{\ding{55}}}
\mdfdefinestyle{dialogue}{
    backgroundcolor=yellow!20,
    innermargin=5pt
}
\usepackage{amssymb}
\usepackage{soul}
\makeatletter

\begin{document}

\hspace{15pt}{\textbf{Section}:Ctenophora5\\}
\\ Context: Ctenophores form an animal phylum that is more complex than sponges, about as complex as cnidarians (jellyfish, sea anemones, etc.), and less complex than bilaterians (which include almost all other animals). Unlike sponges, both ctenophores and cnidarians have: cells bound by inter-cell connections and carpet-like basement membranes; muscles; nervous systems; and some have sensory organs. Ctenophores are distinguished from all other animals by having colloblasts, which are sticky and adhere to prey, although a few ctenophore species lack them. CANNOTANSWER

\begin{figure}[t] \small \begin{tcolorbox}[boxsep=0pt,left=5pt,right=0pt,top=2pt,colback = yellow!5] \begin{dialogue}
 \small 
 \speak{Student}{\bf Jellyfish ans sea anemones belong to what phylum? }
\speak{Teacher}\colorbox{pink!25}{$\hookrightarrow$}
{ ``'' (cnidarians ) }
\\
\speak{Student}{\bf What makes ctenophores different from all other animals? }
\speak{Teacher}\colorbox{pink!25}{$\hookrightarrow$}
{ ``'' (by having colloblasts ) }
\\
\speak{Student}{\bf Ctenophora are less complex than which other phylum? }
\speak{Teacher}\colorbox{pink!25}{$\hookrightarrow$}
{ ``'' (bilaterians ) }
\\
\speak{Student}{\bf Which phylum is more complex than sponges? }
\speak{Teacher}\colorbox{pink!25}{$\hookrightarrow$}
{ ``'' (Ctenophores ) }
\\
\speak{Student}{\bf What does ctenophore use to capture prey? }
\speak{Teacher}\colorbox{pink!25}{$\hookrightarrow$}
{ ``'' (colloblasts ) }
\\
\speak{Student}{\bf Jellyfish and sea anemones belong to which group/ }
\speak{Teacher}\colorbox{pink!25}{$\hookrightarrow$}
{ ``'' (cnidarians ) }
\\
\speak{Student}{\bf What do ctenophores have that no other animals have? }
\speak{Teacher}\colorbox{pink!25}{$\hookrightarrow$}
{ ``'' (colloblasts ) }
\\
\speak{Student}{\bf What do ctenophore use to capture their prey? }
\speak{Teacher}\colorbox{pink!25}{$\hookrightarrow$}
{ ``'' (colloblasts ) }
\\
\speak{Student}{\bf Which two groups have cells bound by inter-cell connections and membranes, muscles, a nervous system and sensory organs? }
\speak{Teacher}\colorbox{pink!25}{$\hookrightarrow$}
{ ``'' (ctenophores and cnidarians ) }
\\
\speak{Student}{\bf Ctenophores are less complex than what other group? }
\speak{Teacher}\colorbox{pink!25}{$\hookrightarrow$}
{ ``'' (bilaterians ) }
\\
\speak{Student}{\bf What connections bind sponge cells? }
\speak{Teacher}\colorbox{pink!25}{$\hookrightarrow$}
{ ``'' (CANNOTANSWER ) }
\\
\speak{Student}{\bf What kind of organs do some sponges have? }
\speak{Teacher}\colorbox{pink!25}{$\hookrightarrow$}
{ ``'' (CANNOTANSWER ) }
\\
\speak{Student}{\bf What distinguishes sponges from all other animals? }
\speak{Teacher}\colorbox{pink!25}{$\hookrightarrow$}
{ ``'' (CANNOTANSWER ) }
\\
\speak{Student}{\bf What do sponges use their colloblasts to catch? }
\speak{Teacher}\colorbox{pink!25}{$\hookrightarrow$}
{ ``'' (CANNOTANSWER ) }
\\
\speak{Student}{\bf What sticky cells used to capture prey are missing from a few sponge species? }
\speak{Teacher}\colorbox{pink!25}{$\hookrightarrow$}
{ ``'' (CANNOTANSWER ) }
\\
 \end{dialogue}\end{tcolorbox}\end{figure}

\end{document}

