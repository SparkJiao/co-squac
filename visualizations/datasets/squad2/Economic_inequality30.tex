\documentclass[11pt,a4paper, onecolumn]{article}
\usepackage{times}
\usepackage{latexsym}
\usepackage{url}
\usepackage{textcomp}
\usepackage{bbm}
\usepackage{amsmath}
\usepackage{booktabs}
\usepackage{tabularx}
\usepackage{graphicx}
\usepackage{dialogue}
\usepackage{mathtools}
\usepackage{hyperref}
%\hypersetup{draft}

\usepackage{multirow}
\usepackage{mdframed}
\usepackage{tcolorbox}

\usepackage{xcolor,pifont}
%\newcommand{\cmark}{\ding{51}}
%\newcommand{\xmark}{\ding{55}}

\setcounter{topnumber}{2}
\setcounter{bottomnumber}{2}
\setcounter{totalnumber}{4}
\renewcommand{\topfraction}{0.75}
\renewcommand{\bottomfraction}{0.75}
\renewcommand{\textfraction}{0.05}
\renewcommand{\floatpagefraction}{0.6}

\newcommand\cmark {\textcolor{green}{\ding{52}}}
\newcommand\xmark {\textcolor{red}{\ding{55}}}
\mdfdefinestyle{dialogue}{
    backgroundcolor=yellow!20,
    innermargin=5pt
}
\usepackage{amssymb}
\usepackage{soul}
\makeatletter

\begin{document}

\hspace{15pt}{\textbf{Section}:Economic inequality30\\}
\\ Context: Economist Joseph Stiglitz presented evidence in 2009 that both global inequality and inequality within countries prevent growth by limiting aggregate demand. Economist Branko Milanovic, wrote in 2001 that, ''The view that income inequality harms growth – or that improved equality can help sustain growth – has become more widely held in recent years. ... The main reason for this shift is the increasing importance of human capital in development. When physical capital mattered most, savings and investments were key. Then it was important to have a large contingent of rich people who could save a greater proportion of their income than the poor and invest it in physical capital. But now that human capital is scarcer than machines, widespread education has become the secret to growth.'' CANNOTANSWER

\begin{figure}[t] \small \begin{tcolorbox}[boxsep=0pt,left=5pt,right=0pt,top=2pt,colback = yellow!5] \begin{dialogue}
 \small 
 \speak{Student}{\bf What did Stiglitz present in 2009 regarding global inequality? }
\speak{Teacher}\colorbox{pink!25}{$\hookrightarrow$}
{ ``'' (evidence ) }
\\
\speak{Student}{\bf How does inequality prevent growth? }
\speak{Teacher}\colorbox{pink!25}{$\hookrightarrow$}
{ ``'' (by limiting aggregate demand ) }
\\
\speak{Student}{\bf What are both Branko Milanovic and Joseph Stiglitz? }
\speak{Teacher}\colorbox{pink!25}{$\hookrightarrow$}
{ ``'' (Economist ) }
\\
\speak{Student}{\bf What has been the main reason for the shift to the view that income inequality harms growth? }
\speak{Teacher}\colorbox{pink!25}{$\hookrightarrow$}
{ ``'' (increasing importance of human capital in development ) }
\\
\speak{Student}{\bf What has become the secret to economic growth? }
\speak{Teacher}\colorbox{pink!25}{$\hookrightarrow$}
{ ``'' (widespread education ) }
\\
\speak{Student}{\bf What did Stiglitz present in 2008 regarding global inequality? }
\speak{Teacher}\colorbox{pink!25}{$\hookrightarrow$}
{ ``'' (CANNOTANSWER ) }
\\
\speak{Student}{\bf How does inequality help growth? }
\speak{Teacher}\colorbox{pink!25}{$\hookrightarrow$}
{ ``'' (CANNOTANSWER ) }
\\
\speak{Student}{\bf What aren't both Branko Milanovic and Joseph Stiglitz? }
\speak{Teacher}\colorbox{pink!25}{$\hookrightarrow$}
{ ``'' (CANNOTANSWER ) }
\\
\speak{Student}{\bf What has been the main reason for the shift to the view that income inequality helps growth? }
\speak{Teacher}\colorbox{pink!25}{$\hookrightarrow$}
{ ``'' (CANNOTANSWER ) }
\\
 \end{dialogue}\end{tcolorbox}\end{figure}

\end{document}

