\documentclass[11pt,a4paper, onecolumn]{article}
\usepackage{times}
\usepackage{latexsym}
\usepackage{url}
\usepackage{textcomp}
\usepackage{bbm}
\usepackage{amsmath}
\usepackage{booktabs}
\usepackage{tabularx}
\usepackage{graphicx}
\usepackage{dialogue}
\usepackage{mathtools}
\usepackage{hyperref}
%\hypersetup{draft}

\usepackage{multirow}
\usepackage{mdframed}
\usepackage{tcolorbox}

\usepackage{xcolor,pifont}
%\newcommand{\cmark}{\ding{51}}
%\newcommand{\xmark}{\ding{55}}

\setcounter{topnumber}{2}
\setcounter{bottomnumber}{2}
\setcounter{totalnumber}{4}
\renewcommand{\topfraction}{0.75}
\renewcommand{\bottomfraction}{0.75}
\renewcommand{\textfraction}{0.05}
\renewcommand{\floatpagefraction}{0.6}

\newcommand\cmark {\textcolor{green}{\ding{52}}}
\newcommand\xmark {\textcolor{red}{\ding{55}}}
\mdfdefinestyle{dialogue}{
    backgroundcolor=yellow!20,
    innermargin=5pt
}
\usepackage{amssymb}
\usepackage{soul}
\makeatletter

\begin{document}

\hspace{15pt}{\textbf{Section}:Prime number12\\}
\\ Context: are prime. Prime numbers of this form are known as factorial primes. Other primes where either p + 1 or p − 1 is of a particular shape include the Sophie Germain primes (primes of the form 2p + 1 with p prime), primorial primes, Fermat primes and Mersenne primes, that is, prime numbers that are of the form 2p − 1, where p is an arbitrary prime. The Lucas–Lehmer test is particularly fast for numbers of this form. This is why the largest known prime has almost always been a Mersenne prime since the dawn of electronic computers. CANNOTANSWER

\begin{figure}[t] \small \begin{tcolorbox}[boxsep=0pt,left=5pt,right=0pt,top=2pt,colback = yellow!5] \begin{dialogue}
 \small 
 \speak{Student}{\bf Of what form are Sophie Germain primes? }
\speak{Teacher}\colorbox{pink!25}{$\hookrightarrow$}
{ ``'' (2p + 1 ) }
\\
\speak{Student}{\bf Of what form are Mersenne primes? }
\speak{Teacher}\colorbox{pink!25}{$\hookrightarrow$}
{ ``'' (2p − 1 ) }
\\
\speak{Student}{\bf What test is especially useful for numbers of the form 2p - 1? }
\speak{Teacher}\colorbox{pink!25}{$\hookrightarrow$}
{ ``'' (The Lucas–Lehmer test ) }
\\
\speak{Student}{\bf What is the name of one type of prime where p+1 or p-1 takes a certain shape? }
\speak{Teacher}\colorbox{pink!25}{$\hookrightarrow$}
{ ``'' (primorial primes ) }
\\
\speak{Student}{\bf What is the name of another type of prime here p+1 or p-1 takes a certain shape? }
\speak{Teacher}\colorbox{pink!25}{$\hookrightarrow$}
{ ``'' (Fermat primes ) }
\\
\speak{Student}{\bf Of what form are Sophie Germain tests? }
\speak{Teacher}\colorbox{pink!25}{$\hookrightarrow$}
{ ``'' (CANNOTANSWER ) }
\\
\speak{Student}{\bf Of what form are Mersenne tests? }
\speak{Teacher}\colorbox{pink!25}{$\hookrightarrow$}
{ ``'' (CANNOTANSWER ) }
\\
\speak{Student}{\bf What test is especially useful for tests of the form 2p-1? }
\speak{Teacher}\colorbox{pink!25}{$\hookrightarrow$}
{ ``'' (CANNOTANSWER ) }
\\
\speak{Student}{\bf What is the name of one type of test where p+1 or p-1 takes a certain shape? }
\speak{Teacher}\colorbox{pink!25}{$\hookrightarrow$}
{ ``'' (CANNOTANSWER ) }
\\
\speak{Student}{\bf What is the name of another type of test where p+1 or p-1 takes a certain shape? }
\speak{Teacher}\colorbox{pink!25}{$\hookrightarrow$}
{ ``'' (CANNOTANSWER ) }
\\
 \end{dialogue}\end{tcolorbox}\end{figure}

\end{document}

