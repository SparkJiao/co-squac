\documentclass[11pt,a4paper, onecolumn]{article}
\usepackage{times}
\usepackage{latexsym}
\usepackage{url}
\usepackage{textcomp}
\usepackage{bbm}
\usepackage{amsmath}
\usepackage{booktabs}
\usepackage{tabularx}
\usepackage{graphicx}
\usepackage{dialogue}
\usepackage{mathtools}
\usepackage{hyperref}
%\hypersetup{draft}

\usepackage{multirow}
\usepackage{mdframed}
\usepackage{tcolorbox}

\usepackage{xcolor,pifont}
%\newcommand{\cmark}{\ding{51}}
%\newcommand{\xmark}{\ding{55}}

\setcounter{topnumber}{2}
\setcounter{bottomnumber}{2}
\setcounter{totalnumber}{4}
\renewcommand{\topfraction}{0.75}
\renewcommand{\bottomfraction}{0.75}
\renewcommand{\textfraction}{0.05}
\renewcommand{\floatpagefraction}{0.6}

\newcommand\cmark {\textcolor{green}{\ding{52}}}
\newcommand\xmark {\textcolor{red}{\ding{55}}}
\mdfdefinestyle{dialogue}{
    backgroundcolor=yellow!20,
    innermargin=5pt
}
\usepackage{amssymb}
\usepackage{soul}
\makeatletter

\begin{document}

\hspace{15pt}{\textbf{Section}:Harvard University6\\}
\\ Context: In 1846, the natural history lectures of Louis Agassiz were acclaimed both in New York and on the campus at Harvard College. Agassiz's approach was distinctly idealist and posited Americans' ''participation in the Divine Nature'' and the possibility of understanding ''intellectual existences''. Agassiz's perspective on science combined observation with intuition and the assumption that a person can grasp the ''divine plan'' in all phenomena. When it came to explaining life-forms, Agassiz resorted to matters of shape based on a presumed archetype for his evidence. This dual view of knowledge was in concert with the teachings of Common Sense Realism derived from Scottish philosophers Thomas Reid and Dugald Stewart, whose works were part of the Harvard curriculum at the time. The popularity of Agassiz's efforts to ''soar with Plato'' probably also derived from other writings to which Harvard students were exposed, including Platonic treatises by Ralph Cudworth, John Norrisand, in a Romantic vein, Samuel Coleridge. The library records at Harvard reveal that the writings of Plato and his early modern and Romantic followers were almost as regularly read during the 19th century as those of the ''official philosophy'' of the more empirical and more deistic Scottish school. CANNOTANSWER

\begin{figure}[t] \small \begin{tcolorbox}[boxsep=0pt,left=5pt,right=0pt,top=2pt,colback = yellow!5] \begin{dialogue}
 \small 
 \speak{Student}{\bf in 1846 who's natural history lectures were acclaimed in New York and Harvard? }
\speak{Teacher}\colorbox{pink!25}{$\hookrightarrow$}
{ ``'' (Louis Agassiz ) }
\\
\speak{Student}{\bf Agassiz's approach to science combined observation and what? }
\speak{Teacher}\colorbox{pink!25}{$\hookrightarrow$}
{ ``'' (intuition ) }
\\
\speak{Student}{\bf Common Sense Realism of what Scottish philosophers did Agassiz incorporate in his dual view of knowedge? }
\speak{Teacher}\colorbox{pink!25}{$\hookrightarrow$}
{ ``'' (Thomas Reid and Dugald Stewart ) }
\\
\speak{Student}{\bf Where were the writings of Plato acclaimed in 1846? }
\speak{Teacher}\colorbox{pink!25}{$\hookrightarrow$}
{ ``'' (CANNOTANSWER ) }
\\
\speak{Student}{\bf What was Plato's approach considered? }
\speak{Teacher}\colorbox{pink!25}{$\hookrightarrow$}
{ ``'' (CANNOTANSWER ) }
\\
\speak{Student}{\bf What did Plato's perspective on science combine? }
\speak{Teacher}\colorbox{pink!25}{$\hookrightarrow$}
{ ``'' (CANNOTANSWER ) }
\\
\speak{Student}{\bf What was the assumption behind Plato's writings? }
\speak{Teacher}\colorbox{pink!25}{$\hookrightarrow$}
{ ``'' (CANNOTANSWER ) }
\\
\speak{Student}{\bf What evidence did Plato use to explain life-forms? }
\speak{Teacher}\colorbox{pink!25}{$\hookrightarrow$}
{ ``'' (CANNOTANSWER ) }
\\
 \end{dialogue}\end{tcolorbox}\end{figure}

\end{document}

