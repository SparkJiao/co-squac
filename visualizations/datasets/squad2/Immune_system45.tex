\documentclass[11pt,a4paper, onecolumn]{article}
\usepackage{times}
\usepackage{latexsym}
\usepackage{url}
\usepackage{textcomp}
\usepackage{bbm}
\usepackage{amsmath}
\usepackage{booktabs}
\usepackage{tabularx}
\usepackage{graphicx}
\usepackage{dialogue}
\usepackage{mathtools}
\usepackage{hyperref}
%\hypersetup{draft}

\usepackage{multirow}
\usepackage{mdframed}
\usepackage{tcolorbox}

\usepackage{xcolor,pifont}
%\newcommand{\cmark}{\ding{51}}
%\newcommand{\xmark}{\ding{55}}

\setcounter{topnumber}{2}
\setcounter{bottomnumber}{2}
\setcounter{totalnumber}{4}
\renewcommand{\topfraction}{0.75}
\renewcommand{\bottomfraction}{0.75}
\renewcommand{\textfraction}{0.05}
\renewcommand{\floatpagefraction}{0.6}

\newcommand\cmark {\textcolor{green}{\ding{52}}}
\newcommand\xmark {\textcolor{red}{\ding{55}}}
\mdfdefinestyle{dialogue}{
    backgroundcolor=yellow!20,
    innermargin=5pt
}
\usepackage{amssymb}
\usepackage{soul}
\makeatletter

\begin{document}

\hspace{15pt}{\textbf{Section}:Immune system45\\}
\\ Context: Larger drugs (>500 Da) can provoke a neutralizing immune response, particularly if the drugs are administered repeatedly, or in larger doses. This limits the effectiveness of drugs based on larger peptides and proteins (which are typically larger than 6000 Da). In some cases, the drug itself is not immunogenic, but may be co-administered with an immunogenic compound, as is sometimes the case for Taxol. Computational methods have been developed to predict the immunogenicity of peptides and proteins, which are particularly useful in designing therapeutic antibodies, assessing likely virulence of mutations in viral coat particles, and validation of proposed peptide-based drug treatments. Early techniques relied mainly on the observation that hydrophilic amino acids are overrepresented in epitope regions than hydrophobic amino acids; however, more recent developments rely on machine learning techniques using databases of existing known epitopes, usually on well-studied virus proteins, as a training set. A publicly accessible database has been established for the cataloguing of epitopes from pathogens known to be recognizable by B cells. The emerging field of bioinformatics-based studies of immunogenicity is referred to as immunoinformatics. Immunoproteomics is the study of large sets of proteins (proteomics) involved in the immune response. CANNOTANSWER

\begin{figure}[t] \small \begin{tcolorbox}[boxsep=0pt,left=5pt,right=0pt,top=2pt,colback = yellow!5] \begin{dialogue}
 \small 
 \speak{Student}{\bf At what size and larger can drugs elicit a neutralizing immune response? }
\speak{Teacher}\colorbox{pink!25}{$\hookrightarrow$}
{ ``'' (>500 Da ) }
\\
\speak{Student}{\bf What kind of amino acids are overrepresented in epitope regions? }
\speak{Teacher}\colorbox{pink!25}{$\hookrightarrow$}
{ ``'' (hydrophilic amino acids ) }
\\
\speak{Student}{\bf What is the study of proteins involved in immune response known as? }
\speak{Teacher}\colorbox{pink!25}{$\hookrightarrow$}
{ ``'' (Immunoproteomics ) }
\\
\speak{Student}{\bf There is a public database of epitopes for pathogens known to be recognizable by what cells? }
\speak{Teacher}\colorbox{pink!25}{$\hookrightarrow$}
{ ``'' (B cells ) }
\\
\speak{Student}{\bf What is the field of studying immunogenicity through bioinformatics known as? }
\speak{Teacher}\colorbox{pink!25}{$\hookrightarrow$}
{ ``'' (immunoinformatics ) }
\\
\speak{Student}{\bf What kind of immune response do smaller drugs provoke? }
\speak{Teacher}\colorbox{pink!25}{$\hookrightarrow$}
{ ``'' (CANNOTANSWER ) }
\\
\speak{Student}{\bf What does the neutralizing immune response mean for drugs based on smaller peptides and proteins? }
\speak{Teacher}\colorbox{pink!25}{$\hookrightarrow$}
{ ``'' (CANNOTANSWER ) }
\\
\speak{Student}{\bf What is not a way that is used to predict the immunogenicity of peptides and proteins? }
\speak{Teacher}\colorbox{pink!25}{$\hookrightarrow$}
{ ``'' (CANNOTANSWER ) }
\\
\speak{Student}{\bf What is underrepresented in epitope regions? }
\speak{Teacher}\colorbox{pink!25}{$\hookrightarrow$}
{ ``'' (CANNOTANSWER ) }
\\
\speak{Student}{\bf What is the study of small sets of proteins? }
\speak{Teacher}\colorbox{pink!25}{$\hookrightarrow$}
{ ``'' (CANNOTANSWER ) }
\\
 \end{dialogue}\end{tcolorbox}\end{figure}

\end{document}

