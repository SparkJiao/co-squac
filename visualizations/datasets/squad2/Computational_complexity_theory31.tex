\documentclass[11pt,a4paper, onecolumn]{article}
\usepackage{times}
\usepackage{latexsym}
\usepackage{url}
\usepackage{textcomp}
\usepackage{bbm}
\usepackage{amsmath}
\usepackage{booktabs}
\usepackage{tabularx}
\usepackage{graphicx}
\usepackage{dialogue}
\usepackage{mathtools}
\usepackage{hyperref}
%\hypersetup{draft}

\usepackage{multirow}
\usepackage{mdframed}
\usepackage{tcolorbox}

\usepackage{xcolor,pifont}
%\newcommand{\cmark}{\ding{51}}
%\newcommand{\xmark}{\ding{55}}

\setcounter{topnumber}{2}
\setcounter{bottomnumber}{2}
\setcounter{totalnumber}{4}
\renewcommand{\topfraction}{0.75}
\renewcommand{\bottomfraction}{0.75}
\renewcommand{\textfraction}{0.05}
\renewcommand{\floatpagefraction}{0.6}

\newcommand\cmark {\textcolor{green}{\ding{52}}}
\newcommand\xmark {\textcolor{red}{\ding{55}}}
\mdfdefinestyle{dialogue}{
    backgroundcolor=yellow!20,
    innermargin=5pt
}
\usepackage{amssymb}
\usepackage{soul}
\makeatletter

\begin{document}

\hspace{15pt}{\textbf{Section}:Computational complexity theory31\\}
\\ Context: This motivates the concept of a problem being hard for a complexity class. A problem X is hard for a class of problems C if every problem in C can be reduced to X. Thus no problem in C is harder than X, since an algorithm for X allows us to solve any problem in C. Of course, the notion of hard problems depends on the type of reduction being used. For complexity classes larger than P, polynomial-time reductions are commonly used. In particular, the set of problems that are hard for NP is the set of NP-hard problems. CANNOTANSWER

\begin{figure}[t] \small \begin{tcolorbox}[boxsep=0pt,left=5pt,right=0pt,top=2pt,colback = yellow!5] \begin{dialogue}
 \small 
 \speak{Student}{\bf The complexity of problems often depends on what? }
\speak{Teacher}\colorbox{pink!25}{$\hookrightarrow$}
{ ``'' (the type of reduction being used ) }
\\
\speak{Student}{\bf What would create a conflict between a problem X and problem C within the context of reduction?  }
\speak{Teacher}\colorbox{pink!25}{$\hookrightarrow$}
{ ``'' (if every problem in C can be reduced to X ) }
\\
\speak{Student}{\bf An algorithm for X which reduces to C would us to do what? }
\speak{Teacher}\colorbox{pink!25}{$\hookrightarrow$}
{ ``'' (solve any problem in C ) }
\\
\speak{Student}{\bf A problem set that that is hard for the expression NP can also be stated how? }
\speak{Teacher}\colorbox{pink!25}{$\hookrightarrow$}
{ ``'' (NP-hard ) }
\\
\speak{Student}{\bf What does the complexity of problems not often depend on? }
\speak{Teacher}\colorbox{pink!25}{$\hookrightarrow$}
{ ``'' (CANNOTANSWER ) }
\\
\speak{Student}{\bf What would not create a conflict between a problem X and problem C within the context of reduction? }
\speak{Teacher}\colorbox{pink!25}{$\hookrightarrow$}
{ ``'' (CANNOTANSWER ) }
\\
\speak{Student}{\bf What problem in C is harder than X? }
\speak{Teacher}\colorbox{pink!25}{$\hookrightarrow$}
{ ``'' (CANNOTANSWER ) }
\\
\speak{Student}{\bf How is a problem set that is hard for expression QP be stated? }
\speak{Teacher}\colorbox{pink!25}{$\hookrightarrow$}
{ ``'' (CANNOTANSWER ) }
\\
 \end{dialogue}\end{tcolorbox}\end{figure}

\end{document}

