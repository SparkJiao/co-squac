\documentclass[11pt,a4paper, onecolumn]{article}
\usepackage{times}
\usepackage{latexsym}
\usepackage{url}
\usepackage{textcomp}
\usepackage{bbm}
\usepackage{amsmath}
\usepackage{booktabs}
\usepackage{tabularx}
\usepackage{graphicx}
\usepackage{dialogue}
\usepackage{mathtools}
\usepackage{hyperref}
%\hypersetup{draft}

\usepackage{multirow}
\usepackage{mdframed}
\usepackage{tcolorbox}

\usepackage{xcolor,pifont}
%\newcommand{\cmark}{\ding{51}}
%\newcommand{\xmark}{\ding{55}}

\setcounter{topnumber}{2}
\setcounter{bottomnumber}{2}
\setcounter{totalnumber}{4}
\renewcommand{\topfraction}{0.75}
\renewcommand{\bottomfraction}{0.75}
\renewcommand{\textfraction}{0.05}
\renewcommand{\floatpagefraction}{0.6}

\newcommand\cmark {\textcolor{green}{\ding{52}}}
\newcommand\xmark {\textcolor{red}{\ding{55}}}
\mdfdefinestyle{dialogue}{
    backgroundcolor=yellow!20,
    innermargin=5pt
}
\usepackage{amssymb}
\usepackage{soul}
\makeatletter

\begin{document}

\hspace{15pt}{\textbf{Section}:Mongol Empire.txt0\\}
\\ Context: The Mongol Empire (Mongolian: ''Mongolyn Ezent Güren'' ; Mongolian Cyrillic: Монголын эзэнт гүрэн; ; also (''Horde'') in Russian chronicles) existed during the 13th and 14th centuries and was the largest contiguous land empire in history. Originating in the steppes of Central Asia, the Mongol Empire eventually stretched from Eastern Europe to the Sea of Japan, extending northwards into Siberia, eastwards and southwards into the Indian subcontinent, Indochina, and the Iranian plateau, and westwards as far as the Levant. The Mongol Empire emerged from the unification of nomadic tribes in the Mongol homeland under the leadership of Genghis Khan, whom a council proclaimed ruler of all the Mongols in 1206. The empire grew rapidly under his rule and that of his descendants, who sent invasions in every direction. The vast transcontinental empire connected the east with the west with an enforced ''Pax Mongolica'', allowing the dissemination and exchange of trade, technologies, commodities, and ideologies across Eurasia. The empire began to split due to wars over succession, as the grandchildren of Genghis Khan disputed whether the royal line should follow from his son and initial heir Ögedei or from one of his other sons, such as Tolui, Chagatai, or Jochi. The Toluids prevailed after a bloody purge of Ögedeid and Chagataid factions, but disputes continued even among the descendants of Tolui. A key reason for the split was the dispute over whether the Mongol Empire would become a sedentary, cosmopolitan empire, or would stay true to their nomadic and steppe lifestyle. After Möngke Khan died (1259), rival kurultai councils simultaneously elected different successors, the brothers Ariq Böke and Kublai Khan, who then not only fought each other in the Toluid Civil War (1260–1264), but also dealt with challenges from descendants of other sons of Genghis. Kublai successfully took power, but civil war ensued as Kublai sought unsuccessfully to regain control of the Chagatayid and Ögedeid families. CANNOTANSWER

\begin{figure}[t] \small \begin{tcolorbox}[boxsep=0pt,left=5pt,right=0pt,top=2pt,colback = yellow!5] \begin{dialogue}
 \small 
 \speak{Student}{\bf When did Genghis Khan become ruler of Mongol? }
\speak{Teacher}\colorbox{pink!25}{$\hookrightarrow$}
{ ``1206'' (1206 ) }
\\
\speak{Student}{\bf Did his empire grow when he and his family ruled? }
\speak{Teacher}\colorbox{pink!25}{$\hookrightarrow$}
\colorbox{red!25}{Yes,}
{ ``Yes.'' (1206 ) }
\\
\speak{Student}{\bf What allowed trade with the east and west? }
\speak{Teacher}\colorbox{pink!25}{$\hookrightarrow$}
{ ``Pax Mongolica'' (Pax Mongolica ) }
\\
\speak{Student}{\bf How many sons did Khan have? }
\speak{Teacher}\colorbox{pink!25}{$\hookrightarrow$}
{ ``at least 4'' (or ) }
\\
\speak{Student}{\bf Who was his original heir? }
\speak{Teacher}\colorbox{pink!25}{$\hookrightarrow$}
{ ``Ögedei'' (Ögedei ) }
\\
\speak{Student}{\bf What were the grandchildren arguing over? }
\speak{Teacher}\colorbox{pink!25}{$\hookrightarrow$}
{ ``If royal line should follow from his son and initial heir Ögedei, or another.'' (royal line should follow from his son and initial heir Ögedei ) }
\\
\speak{Student}{\bf What did their bickering cause to happen to the empire? }
\speak{Teacher}\colorbox{pink!25}{$\hookrightarrow$}
{ ``civil war'' (civil war ) }
\\
\speak{Student}{\bf When was the Toluid Civil War? }
\speak{Teacher}\colorbox{pink!25}{$\hookrightarrow$}
{ ``1260–1264'' (1260–1264 ) }
\\
\speak{Student}{\bf Did it involve family versus family? }
\speak{Teacher}\colorbox{pink!25}{$\hookrightarrow$}
\colorbox{red!25}{Yes,}
{ ``yes.'' (1260–1264 ) }
\\
\speak{Student}{\bf Who won? }
\speak{Teacher}\colorbox{pink!25}{$\hookrightarrow$}
{ ``Kublai'' (Kublai ) }
\\
\speak{Student}{\bf Did it last, though? }
\speak{Teacher}\colorbox{pink!25}{$\hookrightarrow$}
\colorbox{red!25}{No,}
{ ``no'' (Kublai ) }
\\
 \end{dialogue}\end{tcolorbox}\end{figure}

\end{document}

