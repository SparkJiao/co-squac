\documentclass[11pt,a4paper, onecolumn]{article}
\usepackage{times}
\usepackage{latexsym}
\usepackage{url}
\usepackage{textcomp}
\usepackage{bbm}
\usepackage{amsmath}
\usepackage{booktabs}
\usepackage{tabularx}
\usepackage{graphicx}
\usepackage{dialogue}
\usepackage{mathtools}
\usepackage{hyperref}
%\hypersetup{draft}

\usepackage{multirow}
\usepackage{mdframed}
\usepackage{tcolorbox}

\usepackage{xcolor,pifont}
%\newcommand{\cmark}{\ding{51}}
%\newcommand{\xmark}{\ding{55}}

\setcounter{topnumber}{2}
\setcounter{bottomnumber}{2}
\setcounter{totalnumber}{4}
\renewcommand{\topfraction}{0.75}
\renewcommand{\bottomfraction}{0.75}
\renewcommand{\textfraction}{0.05}
\renewcommand{\floatpagefraction}{0.6}

\newcommand\cmark {\textcolor{green}{\ding{52}}}
\newcommand\xmark {\textcolor{red}{\ding{55}}}
\mdfdefinestyle{dialogue}{
    backgroundcolor=yellow!20,
    innermargin=5pt
}
\usepackage{amssymb}
\usepackage{soul}
\makeatletter

\begin{document}

\hspace{15pt}{\textbf{Section}:high4578.txt0\\}
\\ Context: Robots are smart. With their computer brains, they help people work in dangerous places or do difficult jobs. Some robots do regular jobs. Bobby, the robot mail carrier, brings mail to a large office building in Washington, D.C. He is one of 250 robot mail carriers in the United States. Mr. Leachim, who weights two hundred pounds and is six feet tall, has some advantages as a teacher. One is that he does not forget details. He knows each child's name, their parents' names, and what each child knows and needs to know. In addition, he knows each child's pets and hobbies. Mr. Leachim does not make mistakes. Each child goes and tells him his or her name, then dials an identification number. His computer brain puts the child's voice and number together. He identifies the child with no mistakes. Another advantage is that Mr. Leachim is flexible. If the children need more time to do their lessons they can move switches. In this way they can repeat Mr. Leachim's lesson over and over again. When the children do a good job, he tells them something interesting about their hobbies. At the end of the lesson the children switch Mr. Leachim off. CANNOTANSWER

\begin{figure}[t] \small \begin{tcolorbox}[boxsep=0pt,left=5pt,right=0pt,top=2pt,colback = yellow!5] \begin{dialogue}
 \small 
 \speak{Student}{\bf how tall is Mr. Leachim? }
\speak{Teacher}\colorbox{pink!25}{$\hookrightarrow$}
{ ``Six feet.'' (six feet ) }
\\
\speak{Student}{\bf and how much does he weigh? }
\speak{Teacher}\colorbox{pink!25}{$\hookrightarrow$}
{ ``200 pounds.'' (pounds ) }
\\
\speak{Student}{\bf does he keep track of all the details about the children? }
\speak{Teacher}\colorbox{pink!25}{$\hookrightarrow$}
\colorbox{red!25}{Yes,}
{ ``Yes.'' (pounds ) }
\\
\speak{Student}{\bf is he a robot? }
\speak{Teacher}\colorbox{pink!25}{$\hookrightarrow$}
\colorbox{red!25}{Yes,}
{ ``Yes.'' (pounds ) }
\\
\speak{Student}{\bf what does he have for a brain? }
\speak{Teacher}\colorbox{pink!25}{$\hookrightarrow$}
{ ``A computer.'' (computer ) }
\\
\speak{Student}{\bf are robots smart? }
\speak{Teacher}\colorbox{pink!25}{$\hookrightarrow$}
\colorbox{red!25}{Yes,}
{ ``Yes.'' (computer ) }
\\
\speak{Student}{\bf what does bobby do for work? }
\speak{Teacher}\colorbox{pink!25}{$\hookrightarrow$}
{ ``Mail carrier,'' (mail carrier, ) }
\\
\speak{Student}{\bf how many robots are in the story? }
\speak{Teacher}\colorbox{pink!25}{$\hookrightarrow$}
{ ``Two.'' (a ) }
\\
\speak{Student}{\bf how many mail carrying robots are there? }
\speak{Teacher}\colorbox{pink!25}{$\hookrightarrow$}
{ ``250.'' (250 ) }
\\
\speak{Student}{\bf where does bobby live? }
\speak{Teacher}\colorbox{pink!25}{$\hookrightarrow$}
{ ``Washington, D.C.'' (Washington, D.C. ) }
\\
\speak{Student}{\bf where does he take the mail? }
\speak{Teacher}\colorbox{pink!25}{$\hookrightarrow$}
{ ``Large office building.'' (large office building ) }
\\
\speak{Student}{\bf what does Mr. Leachim tell the kids when they do a good job? }
\speak{Teacher}\colorbox{pink!25}{$\hookrightarrow$}
{ ``Something interesting about their hobbies..'' (something interesting about their hobbies. ) }
\\
\speak{Student}{\bf does Mr. Leachim ever get powered down? }
\speak{Teacher}\colorbox{pink!25}{$\hookrightarrow$}
\colorbox{red!25}{Yes,}
{ ``Yes.'' (something interesting about their hobbies. ) }
\\
\speak{Student}{\bf when? }
\speak{Teacher}\colorbox{pink!25}{$\hookrightarrow$}
{ ``At the end of the lesson.'' (At the end of the lesson ) }
\\
\speak{Student}{\bf how do robots help people for example? }
\speak{Teacher}\colorbox{pink!25}{$\hookrightarrow$}
{ ``Do difficult jobs.'' (do difficult jobs. ) }
\\
 \end{dialogue}\end{tcolorbox}\end{figure}\begin{figure}[t] \small \begin{tcolorbox}[boxsep=0pt,left=5pt,right=0pt,top=2pt,colback = yellow!5] \begin{dialogue}
 \small 
 \speak{Student}{\bf do they help other ways? }
\speak{Teacher}\colorbox{pink!25}{$\hookrightarrow$}
{ ``Work in dangerous places.'' (work in dangerous places ) }
\\
 \end{dialogue}\end{tcolorbox}\end{figure}

\end{document}

