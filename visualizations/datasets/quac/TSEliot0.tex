\documentclass[11pt,a4paper, onecolumn]{article}
\usepackage{times}
\usepackage{latexsym}
\usepackage{url}
\usepackage{textcomp}
\usepackage{bbm}
\usepackage{amsmath}
\usepackage{booktabs}
\usepackage{tabularx}
\usepackage{graphicx}
\usepackage{dialogue}
\usepackage{mathtools}
\usepackage{hyperref}
%\hypersetup{draft}

\usepackage{multirow}
\usepackage{mdframed}
\usepackage{tcolorbox}

\usepackage{xcolor,pifont}
%\newcommand{\cmark}{\ding{51}}
%\newcommand{\xmark}{\ding{55}}

\setcounter{topnumber}{2}
\setcounter{bottomnumber}{2}
\setcounter{totalnumber}{4}
\renewcommand{\topfraction}{0.75}
\renewcommand{\bottomfraction}{0.75}
\renewcommand{\textfraction}{0.05}
\renewcommand{\floatpagefraction}{0.6}

\newcommand\cmark {\textcolor{green}{\ding{52}}}
\newcommand\xmark {\textcolor{red}{\ding{55}}}
\mdfdefinestyle{dialogue}{
    backgroundcolor=yellow!20,
    innermargin=5pt
}
\usepackage{amssymb}
\usepackage{soul}
\makeatletter

\begin{document}

\hspace{15pt}{\textbf{Section}:T. S. Eliot0\\}
\\ Context: In a letter to Aiken late in December 1914, Eliot, aged 26, wrote, ''I am very dependent upon women (I mean female society).'' Less than four months later, Thayer introduced Eliot to Vivienne Haigh-Wood, a Cambridge governess. They were married at Hampstead Register Office on 26 June 1915. After a short visit alone to his family in the United States, Eliot returned to London and took several teaching jobs, such as lecturing at Birkbeck College, University of London. The philosopher Bertrand Russell took an interest in Vivienne while the newlyweds stayed in his flat. Some scholars have suggested that she and Russell had an affair, but the allegations were never confirmed. The marriage was markedly unhappy, in part because of Vivienne's health issues. In a letter addressed to Ezra Pound, she covers an extensive list of her symptoms, which included a habitually high temperature, fatigue, insomnia, migraines, and colitis. This, coupled with apparent mental instability, meant that she was often sent away by Eliot and her doctors for extended periods of time in the hope of improving her health, and as time went on, he became increasingly detached from her. The couple formally separated in 1933 and in 1938 Vivienne's brother, Maurice, had her committed to a lunatic asylum, against her will, where she remained until her death of heart disease in 1947. Their relationship became the subject of a 1984 play Tom & Viv, which in 1994 was adapted as a film. In a private paper written in his sixties, Eliot confessed: ''I came to persuade myself that I was in love with Vivienne simply because I wanted to burn my boats and commit myself to staying in England. And she persuaded herself (also under the influence of [Ezra] Pound) that she would save the poet by keeping him in England. To her, the marriage brought no happiness. To me, it brought the state of mind out of which came The Waste Land.'' CANNOTANSWER

\begin{figure}[t] \small \begin{tcolorbox}[boxsep=0pt,left=5pt,right=0pt,top=2pt,colback = yellow!5] \begin{dialogue}
 \small 
 \speak{Student}{\bf When did Eliot get married? }
\speak{Teacher}\colorbox{pink!25}{$\hookrightarrow$}
{ ``'' (on 26 June 1915. ) }
\\
\speak{Student}{\bf Who did he marry? }
\speak{Teacher}\colorbox{pink!25}{$\hookrightarrow$}
{ ``'' (Vivienne Haigh-Wood, ) }
\\
\speak{Student}{\bf Where did he meet Vivienne? }
\speak{Teacher}\colorbox{pink!25}{$\hookrightarrow$}
{ ``'' (Cambridge ) }
\\
\speak{Student}{\bf Did they have a happy marriage? }
\speak{Teacher}\colorbox{pink!25}{$\hookrightarrow$}
\colorbox{red!25}{No,}
{ ``'' (I was in love with Vivienne simply because I wanted to burn my boats and commit myself to staying in England. ) }
\\
\speak{Student}{\bf Did they have any children? }
\speak{Teacher}\colorbox{pink!25}{$\hookrightarrow$}
{ ``'' (CANNOTANSWER ) }
\\
\speak{Student}{\bf Where did the couple live? }
\speak{Teacher}\colorbox{pink!25}{$\hookrightarrow$}
{ ``'' (England. ) }
\\
\speak{Student}{\bf What did Vivienne do? }
\speak{Teacher}\colorbox{pink!25}{$\hookrightarrow$}
{ ``'' (CANNOTANSWER ) }
\\
 \end{dialogue}\end{tcolorbox}\end{figure}

\end{document}

