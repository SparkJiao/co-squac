\documentclass[11pt,a4paper, onecolumn]{article}
\usepackage{times}
\usepackage{latexsym}
\usepackage{url}
\usepackage{textcomp}
\usepackage{bbm}
\usepackage{amsmath}
\usepackage{booktabs}
\usepackage{tabularx}
\usepackage{graphicx}
\usepackage{dialogue}
\usepackage{mathtools}
\usepackage{hyperref}
%\hypersetup{draft}

\usepackage{multirow}
\usepackage{mdframed}
\usepackage{tcolorbox}

\usepackage{xcolor,pifont}
%\newcommand{\cmark}{\ding{51}}
%\newcommand{\xmark}{\ding{55}}

\setcounter{topnumber}{2}
\setcounter{bottomnumber}{2}
\setcounter{totalnumber}{4}
\renewcommand{\topfraction}{0.75}
\renewcommand{\bottomfraction}{0.75}
\renewcommand{\textfraction}{0.05}
\renewcommand{\floatpagefraction}{0.6}

\newcommand\cmark {\textcolor{green}{\ding{52}}}
\newcommand\xmark {\textcolor{red}{\ding{55}}}
\mdfdefinestyle{dialogue}{
    backgroundcolor=yellow!20,
    innermargin=5pt
}
\usepackage{amssymb}
\usepackage{soul}
\makeatletter

\begin{document}

\hspace{15pt}{\textbf{Section}:John C. Frémont0\\}
\\ Context: On January 16, 1847, Commodore Stockton appointed Fremont military governor of California following the Treaty of Cahuenga, and then left Los Angeles. Fremont functioned for a few weeks without controversy, but he had little money to administer his duties as governor. Previously, unknown to Stockton and Fremont, the Navy Department had sent orders for Sloat and his successors to establish military rule over California. These orders, however, postdated Kearny's orders to establish military control over California, but Kearny did not have the troop strength to enforce the orders, relying on Stockton and Fremont's California Battalion. Kearny, a veteran of the War of 1812, was a jealous officer, a grim martinet, who despised the rapid advancement, popularity, and success of Fremont, and was determined to humiliate him. On February 13, specific orders were sent from Washington through Commanding General Winfield Scott giving Kearny the authority to be military governor of California. Kearny, however, did not directly inform Fremont of these orders from Scott. Kearny ordered that Fremont's California Battalion be enlisted into the U.S. Army and Fremont send his archives to California. Fremont delayed these orders hoping Washington would send instructions for Fremont to be military governor. Also, the California Battalion refused to join the U.S. Army. Fremont gave orders for the California Battalion not to surrender arms, and rode to Monterey to talk to Kearny, and told Kearny he would obey orders. Kearny sent Col. Richard B. Mason to Los Angeles, who was to succeed Kearny as military governor of California, to inspect troops and give Fremont further orders. Fremont and Mason however were at odds with each other and Fremont challenged Mason to a duel. After an arrangement to postpone the duel, Kearny rode to Los Angeles and refused Fremont's request to join troops in Mexico. Ordered to march with Kearny's army back east, Fremont was arrested on August 22, 1847 when they arrived at Fort Leavenworth. He was charged with mutiny, disobedience of orders, assumption of powers, along with several other military offenses. Ordered by Kearny to report to the adjutant general in Washington to stand for court-martial, Fremont was convicted of mutiny, disobedience of a superior officer and military misconduct on January 31, 1848. While approving the court's decision, President James K. Polk quickly commuted Fremont's sentence of dishonorable discharge and reinstated him into the Army, due to his war services. Polk felt that Fremont was guilty of disobeying orders and misconduct, but he did not believe Fremont was guilty of mutiny. Additionally, Polk wished to placate Thomas Hart Benton, a powerful Senator and Fremont's father in law who felt that Fremont was innocent. Fremont, only gaining a partial pardon from Polk, resigned his commission in protest and settled in California. Despite the court-martial Fremont remained popular among the American public. CANNOTANSWER

\begin{figure}[t] \small \begin{tcolorbox}[boxsep=0pt,left=5pt,right=0pt,top=2pt,colback = yellow!5] \begin{dialogue}
 \small 
 \speak{Student}{\bf What did he get courtmartialed? }
\speak{Teacher}\colorbox{pink!25}{$\hookrightarrow$}
{ ``'' (Fremont was convicted of mutiny, disobedience of a superior officer and military misconduct ) }
\\
\speak{Student}{\bf Did he resign after this? }
\speak{Teacher}\colorbox{pink!25}{$\hookrightarrow$}
{ ``'' (CANNOTANSWER ) }
\\
\speak{Student}{\bf Why did he end up resigning? }
\speak{Teacher}\colorbox{pink!25}{$\hookrightarrow$}
{ ``'' (Polk felt that Fremont was guilty of disobeying orders and misconduct, but he did not believe Fremont was guilty of mutiny. ) }
\\
\speak{Student}{\bf What did he do after he left the military? }
\speak{Teacher}\colorbox{pink!25}{ $\bar{\hookrightarrow}$}
{ ``'' (Fremont, only gaining a partial pardon from Polk, resigned his commission in protest and settled in California. ) }
\\
 \end{dialogue}\end{tcolorbox}\end{figure}

\end{document}

