\documentclass[11pt,a4paper, onecolumn]{article}
\usepackage{times}
\usepackage{latexsym}
\usepackage{url}
\usepackage{textcomp}
\usepackage{bbm}
\usepackage{amsmath}
\usepackage{booktabs}
\usepackage{tabularx}
\usepackage{graphicx}
\usepackage{dialogue}
\usepackage{mathtools}
\usepackage{hyperref}
%\hypersetup{draft}

\usepackage{multirow}
\usepackage{mdframed}
\usepackage{tcolorbox}

\usepackage{xcolor,pifont}
%\newcommand{\cmark}{\ding{51}}
%\newcommand{\xmark}{\ding{55}}

\setcounter{topnumber}{2}
\setcounter{bottomnumber}{2}
\setcounter{totalnumber}{4}
\renewcommand{\topfraction}{0.75}
\renewcommand{\bottomfraction}{0.75}
\renewcommand{\textfraction}{0.05}
\renewcommand{\floatpagefraction}{0.6}

\newcommand\cmark {\textcolor{green}{\ding{52}}}
\newcommand\xmark {\textcolor{red}{\ding{55}}}
\mdfdefinestyle{dialogue}{
    backgroundcolor=yellow!20,
    innermargin=5pt
}
\usepackage{amssymb}
\usepackage{soul}
\makeatletter

\begin{document}

\hspace{15pt}{\textbf{Section}:Louis Armstrong0\\}
\\ Context: Throughout his riverboat experience, Armstrong's musicianship began to mature and expand. At twenty, he could read music and started to be featured in extended trumpet solos, one of the first jazz men to do this, injecting his own personality and style into his solo turns. He had learned how to create a unique sound and also started using singing and patter in his performances. In 1922, Armstrong joined the exodus to Chicago, where he had been invited by his mentor, Joe ''King'' Oliver, to join his Creole Jazz Band and where he could make a sufficient income so that he no longer needed to supplement his music with day labor jobs. It was a boom time in Chicago and though race relations were poor, the city was teeming with jobs available for black people, who were making good wages in factories and had plenty to spend on entertainment. Oliver's band was among the most influential jazz bands in Chicago in the early 1920s, at a time when Chicago was the center of the jazz universe. Armstrong lived luxuriously in Chicago, in his own apartment with his own private bath (his first). Excited as he was to be in Chicago, he began his career-long pastime of writing nostalgic letters to friends in New Orleans. Unusually, Armstrong could blow two hundred high Cs in a row. As his reputation grew, he was challenged to instrumental ''cutting contests'' by hornmen trying to displace him. Armstrong made his first recordings on the Gennett and Okeh labels (jazz records were starting to boom across the country), including taking some solos and breaks, while playing second cornet in Oliver's band in 1923. At this time, he met Hoagy Carmichael (with whom he would collaborate later) who was introduced by friend Bix Beiderbecke, who now had his own Chicago band. Armstrong enjoyed working with Oliver, but Louis' second wife, pianist Lil Hardin Armstrong, urged him to seek more prominent billing and develop his newer style away from the influence of Oliver. Lil had her husband play classical music in church concerts to broaden his skill and improve his solo play and she prodded him into wearing more stylish attire to make him look sharp and to better offset his growing girth. Lil's influence eventually undermined Armstrong's relationship with his mentor, especially concerning his salary and additional moneys that Oliver held back from Armstrong and other band members. CANNOTANSWER

\begin{figure}[t] \small \begin{tcolorbox}[boxsep=0pt,left=5pt,right=0pt,top=2pt,colback = yellow!5] \begin{dialogue}
 \small 
 \speak{Student}{\bf When did Louis Armstrong move to Chicago? }
\speak{Teacher}\colorbox{pink!25}{$\hookrightarrow$}
{ ``'' (In 1922, Armstrong joined the exodus to Chicago, where he had been invited by his mentor, Joe ''King'' Oliver, to join his Creole Jazz Band ) }
\\
\speak{Student}{\bf What was his first gig in Chicago? }
\speak{Teacher}\colorbox{pink!25}{$\hookrightarrow$}
{ ``'' (Oliver's band was among the most influential jazz bands in Chicago in the early 1920s, at a time when Chicago was the center of the jazz universe. ) }
\\
\speak{Student}{\bf What did the critics say of Armstrong's playing? }
\speak{Teacher}\colorbox{pink!25}{$\not\hookrightarrow$}
{ ``'' (CANNOTANSWER ) }
\\
\speak{Student}{\bf What songs was he famous for? }
\speak{Teacher}\colorbox{pink!25}{$\not\hookrightarrow$}
{ ``'' (CANNOTANSWER ) }
\\
\speak{Student}{\bf Are there any other interesting aspects about this article? }
\speak{Teacher}\colorbox{pink!25}{$\hookrightarrow$}
{ ``'' (Armstrong enjoyed working with Oliver, but Louis' second wife, pianist Lil Hardin Armstrong, urged him to seek more prominent billing and develop his newer style ) }
\\
\speak{Student}{\bf When did he start his own band? }
\speak{Teacher}\colorbox{pink!25}{$\hookrightarrow$}
{ ``'' (Lil had her husband play classical music in church concerts to broaden his skill and improve his solo play and she prodded him into wearing more stylish attire ) }
\\
\speak{Student}{\bf What was Armstrong's style? }
\speak{Teacher}\colorbox{pink!25}{$\hookrightarrow$}
{ ``'' (she prodded him into wearing more stylish attire to make him look sharp and to better offset his growing girth. ) }
\\
\speak{Student}{\bf Are there any other interesting aspects about this article? }
\speak{Teacher}\colorbox{pink!25}{$\hookrightarrow$}
{ ``'' (Excited as he was to be in Chicago, he began his career-long pastime of writing nostalgic letters to friends in New Orleans. ) }
\\
\speak{Student}{\bf When did he leave Chicago? }
\speak{Teacher}\colorbox{pink!25}{$\not\hookrightarrow$}
{ ``'' (CANNOTANSWER ) }
\\
 \end{dialogue}\end{tcolorbox}\end{figure}

\end{document}

