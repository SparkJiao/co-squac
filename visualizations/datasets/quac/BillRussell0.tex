\documentclass[11pt,a4paper, onecolumn]{article}
\usepackage{times}
\usepackage{latexsym}
\usepackage{url}
\usepackage{textcomp}
\usepackage{bbm}
\usepackage{amsmath}
\usepackage{booktabs}
\usepackage{tabularx}
\usepackage{graphicx}
\usepackage{dialogue}
\usepackage{mathtools}
\usepackage{hyperref}
%\hypersetup{draft}

\usepackage{multirow}
\usepackage{mdframed}
\usepackage{tcolorbox}

\usepackage{xcolor,pifont}
%\newcommand{\cmark}{\ding{51}}
%\newcommand{\xmark}{\ding{55}}

\setcounter{topnumber}{2}
\setcounter{bottomnumber}{2}
\setcounter{totalnumber}{4}
\renewcommand{\topfraction}{0.75}
\renewcommand{\bottomfraction}{0.75}
\renewcommand{\textfraction}{0.05}
\renewcommand{\floatpagefraction}{0.6}

\newcommand\cmark {\textcolor{green}{\ding{52}}}
\newcommand\xmark {\textcolor{red}{\ding{55}}}
\mdfdefinestyle{dialogue}{
    backgroundcolor=yellow!20,
    innermargin=5pt
}
\usepackage{amssymb}
\usepackage{soul}
\makeatletter

\begin{document}

\hspace{15pt}{\textbf{Section}:Bill Russell0\\}
\\ Context: Russell was driven by ''a neurotic need to win'', as his Celtic teammate Heinsohn observed. He was so tense before every game that he regularly vomited in the locker room; early in his career it happened so frequently that his fellow Celtics were more worried when it did not happen. Later in Russell's career, Havlicek said of his teammate and coach that he threw up less often than early in his career, only doing so ''when it's an important game or an important challenge for him--someone like Chamberlain, or someone coming up that everyone's touting. [The sound of Russell throwing up] is a welcome sound, too, because it means he's keyed up for the game, and around the locker room we grin and say, ''Man, we're going to be all right tonight.'' In a retrospective interview, Russell described the state of mind he felt he needed to enter in order to be able to play basketball as, ''I had to almost be in a rage. Nothing went on outside the borders[] of the court. I could hear anything, I could see anything, and nothing mattered. And I could anticipate every move that every player made.'' Russell was also known for his natural authority. When he became player-coach in 1966, Russell bluntly said to his teammates that ''he intended to cut all personal ties to other players'', and seamlessly made the transition from their peer to their superior. Russell, at the time his additional role of coach was announced, publicly stated he believed Auerbach's (who he regarded as the greatest of all coaches) impact as a coach confined every or almost every relationship with each Celtic player to a strictly professional one.) CANNOTANSWER

\begin{figure}[t] \small \begin{tcolorbox}[boxsep=0pt,left=5pt,right=0pt,top=2pt,colback = yellow!5] \begin{dialogue}
 \small 
 \speak{Student}{\bf How was Russell as a competitor? }
\speak{Teacher}\colorbox{pink!25}{$\hookrightarrow$}
{ ``'' (Russell was driven by ''a neurotic need to win'', ) }
\\
\speak{Student}{\bf How did other athletes like him? }
\speak{Teacher}\colorbox{pink!25}{ $\bar{\hookrightarrow}$}
{ ``'' (CANNOTANSWER ) }
\\
\speak{Student}{\bf What aspect of the game was he the best at? }
\speak{Teacher}\colorbox{pink!25}{ $\bar{\hookrightarrow}$}
{ ``'' (Russell was also known for his natural authority. ) }
\\
\speak{Student}{\bf Did he perform well with any other players? }
\speak{Teacher}\colorbox{pink!25}{$\not\hookrightarrow$}
{ ``'' (CANNOTANSWER ) }
\\
\speak{Student}{\bf Did he win any major games? }
\speak{Teacher}\colorbox{pink!25}{$\not\hookrightarrow$}
{ ``'' (CANNOTANSWER ) }
\\
 \end{dialogue}\end{tcolorbox}\end{figure}

\end{document}

