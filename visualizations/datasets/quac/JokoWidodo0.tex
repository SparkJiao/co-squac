\documentclass[11pt,a4paper, onecolumn]{article}
\usepackage{times}
\usepackage{latexsym}
\usepackage{url}
\usepackage{textcomp}
\usepackage{bbm}
\usepackage{amsmath}
\usepackage{booktabs}
\usepackage{tabularx}
\usepackage{graphicx}
\usepackage{dialogue}
\usepackage{mathtools}
\usepackage{hyperref}
%\hypersetup{draft}

\usepackage{multirow}
\usepackage{mdframed}
\usepackage{tcolorbox}

\usepackage{xcolor,pifont}
%\newcommand{\cmark}{\ding{51}}
%\newcommand{\xmark}{\ding{55}}

\setcounter{topnumber}{2}
\setcounter{bottomnumber}{2}
\setcounter{totalnumber}{4}
\renewcommand{\topfraction}{0.75}
\renewcommand{\bottomfraction}{0.75}
\renewcommand{\textfraction}{0.05}
\renewcommand{\floatpagefraction}{0.6}

\newcommand\cmark {\textcolor{green}{\ding{52}}}
\newcommand\xmark {\textcolor{red}{\ding{55}}}
\mdfdefinestyle{dialogue}{
    backgroundcolor=yellow!20,
    innermargin=5pt
}
\usepackage{amssymb}
\usepackage{soul}
\makeatletter

\begin{document}

\hspace{15pt}{\textbf{Section}:Joko Widodo0\\}
\\ Context: In the first quarter of 2015, year-on-year GDP grew 4.92 percent. In the second quarter it grew 4.6 , the lowest figure since 2009. Anything less than 6 per cent and Indonesia cannot absorb the new entrants to its labour market each year. Furthermore since most of 2017 economic growth remained above the 5.2 percent mark, the Indonesian government has projected its economic growth in 2018 to be at least at 5.4 percent, which is still .6 percent below what is considered healthy economic growth mark of 6 percent . In combination of many factors such as international trade war initiated between the U.S. and China, U.S. Federal Reserves' tightening of monetary policy and the general elections in Indonesia in 2019, much of the anemic nature of Indonesia's economic growth stems from income inequality. Whereas current government and its robust economic and ''overly ambitious'' infrastructure policies aim to stimulate economic growth, it hasn't been felt across the board. Only the middle- and upper- socio-economic households have been positively affected by the current government's policies. This is consistent with what the current administration attempts to do by improving infrastructure, transportation and connectivity across the country. By providing infrastructure such as roads, highways, bridges, railroads and airports across the country and into rural Indonesia, this will reduce the costs of goods. As a result, not only will this increase the purchasing power parity and reduce economic inequality, but also improving the quality of living across the board. The rupiah weakened further, with its exchange rate per US dollar, falling to Rp 14,000 in August 2015, the lowest level in the last 17 years. On 24 September 2015, it closed at 14797. The rupiah appreciated by 2.28 percent to Rp13.473/USD on 31 December 2016. The year-on-year inflation in June 2015 was 7.26 percent, higher than in May (7.15 percent) and June the year before (6.7 percent). CANNOTANSWER

\begin{figure}[t] \small \begin{tcolorbox}[boxsep=0pt,left=5pt,right=0pt,top=2pt,colback = yellow!5] \begin{dialogue}
 \small 
 \speak{Student}{\bf in what areas was Joko successful? }
\speak{Teacher}\colorbox{pink!25}{$\not\hookrightarrow$}
{ ``'' (In the second quarter it grew 4.6 , the lowest figure since 2009. ) }
\\
\speak{Student}{\bf what grew 4.6%? }
\speak{Teacher}\colorbox{pink!25}{$\not\hookrightarrow$}
{ ``'' (GDP ) }
\\
\speak{Student}{\bf What happened as a result of GDP growing? }
\speak{Teacher}\colorbox{pink!25}{$\not\hookrightarrow$}
{ ``'' (Anything less than 6 per cent and Indonesia cannot absorb the new entrants to its labour market each year. ) }
\\
\speak{Student}{\bf How did the economy perform during other periods? }
\speak{Teacher}\colorbox{pink!25}{$\not\hookrightarrow$}
{ ``'' (the Indonesian government has projected its economic growth in 2018 to be at least at 5.4 percent, ) }
\\
\speak{Student}{\bf What impact did Joko have on his country? }
\speak{Teacher}\colorbox{pink!25}{$\not\hookrightarrow$}
{ ``'' (trade war initiated between the U.S. and China, U.S. Federal Reserves ) }
\\
\speak{Student}{\bf What legacy did Joko leave behind to the economy? }
\speak{Teacher}\colorbox{pink!25}{$\not\hookrightarrow$}
{ ``'' (Indonesia's economic growth stems from income inequality. ) }
\\
\speak{Student}{\bf Is income inequality still an issue? }
\speak{Teacher}\colorbox{pink!25}{$\not\hookrightarrow$}
\colorbox{red!25}{Yes,}
{ ``'' (''overly ambitious'' ) }
\\
\speak{Student}{\bf What should people know about Joko's current economic performance? }
\speak{Teacher}\colorbox{pink!25}{$\not\hookrightarrow$}
{ ``'' (policies aim to stimulate economic growth, ) }
\\
\speak{Student}{\bf Are there any policies that hurt the economy? }
\speak{Teacher}\colorbox{pink!25}{$\not\hookrightarrow$}
\colorbox{red!25}{Yes,}
{ ``'' (Only the middle- and upper- socio-economic households have been positively affected ) }
\\
\speak{Student}{\bf what happened to the country when the economy improved? }
\speak{Teacher}\colorbox{pink!25}{$\not\hookrightarrow$}
{ ``'' (households have been positively affected ) }
\\
 \end{dialogue}\end{tcolorbox}\end{figure}

\end{document}

