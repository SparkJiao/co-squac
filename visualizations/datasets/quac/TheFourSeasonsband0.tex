\documentclass[11pt,a4paper, onecolumn]{article}
\usepackage{times}
\usepackage{latexsym}
\usepackage{url}
\usepackage{textcomp}
\usepackage{bbm}
\usepackage{amsmath}
\usepackage{booktabs}
\usepackage{tabularx}
\usepackage{graphicx}
\usepackage{dialogue}
\usepackage{mathtools}
\usepackage{hyperref}
%\hypersetup{draft}

\usepackage{multirow}
\usepackage{mdframed}
\usepackage{tcolorbox}

\usepackage{xcolor,pifont}
%\newcommand{\cmark}{\ding{51}}
%\newcommand{\xmark}{\ding{55}}

\setcounter{topnumber}{2}
\setcounter{bottomnumber}{2}
\setcounter{totalnumber}{4}
\renewcommand{\topfraction}{0.75}
\renewcommand{\bottomfraction}{0.75}
\renewcommand{\textfraction}{0.05}
\renewcommand{\floatpagefraction}{0.6}

\newcommand\cmark {\textcolor{green}{\ding{52}}}
\newcommand\xmark {\textcolor{red}{\ding{55}}}
\mdfdefinestyle{dialogue}{
    backgroundcolor=yellow!20,
    innermargin=5pt
}
\usepackage{amssymb}
\usepackage{soul}
\makeatletter

\begin{document}

\hspace{15pt}{\textbf{Section}:The Four Seasons (band)0\\}
\\ Context: By 1969, the band's popularity had declined, with public interest moving towards rock with a harder edge and music with more socially conscious lyrics. Aware of that, Bob Gaudio partnered with folk-rock songwriter Jake Holmes to write a concept album titled The Genuine Imitation Life Gazette, which discussed contemporary issues from the band's standpoint, including divorce (''Saturday's Father''), and Kinks-style satirical looks at modern life (e.g., ''American Crucifixion and Resurrection'', ''Mrs. Stately's Garden'', ''Genuine Imitation Life''). The album cover was designed to resemble the front page of a newspaper, pre-dating Jethro Tull's Thick as a Brick by several years. The record was a commercial failure and led to band's departure from Philips shortly thereafter, but it did catch the attention of Frank Sinatra, whose 1969 album, Watertown, involved Gaudio, Holmes and Calello. The Seasons' last single on Philips, 1970's ''Patch of Blue'', featured the band's name as ''Frankie Valli & the Four Seasons'', but the change in billing did not revive the band's fortunes. Reverting to the ''Four Seasons'' billing without Valli's name up front, the band issued a single on Crewe's eponymous label, ''And That Reminds Me'', which peaked at number 45 on the Billboard chart. After leaving Philips, the Four Seasons recorded a one-off single for the Warner Bros. label in England, ''Sleeping Man'', backed by ''Whatever You Say'', which was never released in the USA. John Stefan, the band's lead trumpeter, arranged the horn parts. Following that single, the band signed to Motown. The first LP, Chameleon, released by Motown subsidiary label MoWest Records in 1972, failed to sell. A 1971 Frankie Valli solo single on Motown,''Love Isn't Here'', and three Four Seasons singles, ''Walk On, Don't Look Back'' on MoWest in 1972, ''How Come'' and ''Hickory'' on Motown in 1973, sank without a trace. A song from Chameleon, ''The Night'', later became a Northern Soul hit and reached the top 10 of the UK Singles Chart, but was not commercially released in the United States as a single, although promotional copies were distributed in 1972, showing the artist as Frankie Valli. In late 1973 and early 1974, the Four Seasons recorded eight songs for a second Motown album, which the company refused to release, and later in 1974, the label and the band parted ways. On behalf of the Four Seasons Partnership, Valli tried to purchase the entire collection of master recordings the band had made for Motown. After hearing the amount needed to buy them all, Valli arranged to purchase ''My Eyes Adored You'' for  US4000. He took the tape to Larry Uttal, the owner and founder of Private Stock Records, who wanted to release it as a Frankie Valli solo single. Although the band remained unsigned in the later part of 1974, Valli had a new label--and a new solo career. CANNOTANSWER

\begin{figure}[t] \small \begin{tcolorbox}[boxsep=0pt,left=5pt,right=0pt,top=2pt,colback = yellow!5] \begin{dialogue}
 \small 
 \speak{Student}{\bf What happened at the end of the 1960s }
\speak{Teacher}\colorbox{pink!25}{$\hookrightarrow$}
{ ``'' (By 1969, the band's popularity had declined, with public interest moving towards rock with a harder edge and music with more socially conscious lyrics. ) }
\\
\speak{Student}{\bf Did they transition to motown }
\speak{Teacher}\colorbox{pink!25}{$\hookrightarrow$}
{ ``'' (Following that single, the band signed to Motown. ) }
\\
\speak{Student}{\bf What single prompted the move to motown }
\speak{Teacher}\colorbox{pink!25}{$\hookrightarrow$}
{ ``'' (''Sleeping Man'', backed by ''Whatever You Say'', which was never released in the USA. ) }
\\
\speak{Student}{\bf Did they win any awards for these singles }
\speak{Teacher}\colorbox{pink!25}{$\not\hookrightarrow$}
{ ``'' (reached the top 10 ) }
\\
 \end{dialogue}\end{tcolorbox}\end{figure}

\end{document}

