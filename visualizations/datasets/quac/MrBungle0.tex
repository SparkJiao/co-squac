\documentclass[11pt,a4paper, onecolumn]{article}
\usepackage{times}
\usepackage{latexsym}
\usepackage{url}
\usepackage{textcomp}
\usepackage{bbm}
\usepackage{amsmath}
\usepackage{booktabs}
\usepackage{tabularx}
\usepackage{graphicx}
\usepackage{dialogue}
\usepackage{mathtools}
\usepackage{hyperref}
%\hypersetup{draft}

\usepackage{multirow}
\usepackage{mdframed}
\usepackage{tcolorbox}

\usepackage{xcolor,pifont}
%\newcommand{\cmark}{\ding{51}}
%\newcommand{\xmark}{\ding{55}}

\setcounter{topnumber}{2}
\setcounter{bottomnumber}{2}
\setcounter{totalnumber}{4}
\renewcommand{\topfraction}{0.75}
\renewcommand{\bottomfraction}{0.75}
\renewcommand{\textfraction}{0.05}
\renewcommand{\floatpagefraction}{0.6}

\newcommand\cmark {\textcolor{green}{\ding{52}}}
\newcommand\xmark {\textcolor{red}{\ding{55}}}
\mdfdefinestyle{dialogue}{
    backgroundcolor=yellow!20,
    innermargin=5pt
}
\usepackage{amssymb}
\usepackage{soul}
\makeatletter

\begin{document}

\hspace{15pt}{\textbf{Section}:Mr. Bungle0\\}
\\ Context: Due to artwork delays and the band members' many side-projects, it was four years before Disco Volante was released, in October 1995. The new album displayed musical development and a shift in tone from their earlier recordings. While the self-titled album was described as ''funk metal'', with Disco Volante this label was replaced with ''avant-garde'' or ''experimental''. The music was complex and unpredictable, with the band continuing with their shifts of musical style. Some of the tracks were in foreign languages and would radically change genres mid-song. Featuring lyrics about death, suicide and child abuse, along with children's songs and a Middle Eastern techno number, music critic Greg Prato described the album as having ''a totally original and new musical style that sounds like nothing that currently exists''. Not all critics were impressed with the album, with The Washington Post describing it as ''an album of cheesy synthesizers, mangled disco beats, virtuosic playing and juvenile noises'', calling it ''self-indulgent'' and adding that ''Mr. Bungle's musicians like to show off their classical, jazz and world-beat influences in fast, difficult passages which are technically impressive but never seem to go anywhere''. Additionally, writer Scott McGaughey described it as ''difficult'', and was critical of its ''lack of actual songs''. Disco Volante included influences from contemporary classical music, avant-garde jazz, electronic music pioneer Pierre Henry, Edgar Allan Poe, John Zorn, Krzysztof Penderecki and European film music of the 1960s and 1970s, such as those composed by Ennio Morricone and Peter Thomas. The album notes also contained an invitation to participate in an ''unusual scam'' - if  2 was sent to the band's address, participants would receive additional artwork, lyrics to the songs ''Ma Meeshka Mow Skwoz'' and ''Chemical Marriage'' and some stickers. The vinyl release of this album shipped with a 7'' by the then-unknown Secret Chiefs 3. In 1996, Theo Lengyel retired as Bungle's original sax player and keyboardist due to creative differences. Mr. Bungle supported this record with their first world tour, performing across North America, Europe and Australia during 1995 and 1996. In 1997, the Seattle-based Rastacore Records started distributing CDs of Bowel of Chiley (incorrectly labelled Bowl of Chiley on the Rastacore release). This was done without official authorization from Mr. Bungle or Warner Bros., and as such production was halted, with only a limited number of CDs surviving. CANNOTANSWER

\begin{figure}[t] \small \begin{tcolorbox}[boxsep=0pt,left=5pt,right=0pt,top=2pt,colback = yellow!5] \begin{dialogue}
 \small 
 \speak{Student}{\bf how did the disco volante begin? }
\speak{Teacher}\colorbox{pink!25}{$\hookrightarrow$}
{ ``'' (Due to artwork delays and the band members' many side-projects, it was four years before Disco Volante was released, in October 1995. ) }
\\
\speak{Student}{\bf how did the perform commercially? }
\speak{Teacher}\colorbox{pink!25}{$\not\hookrightarrow$}
{ ``'' (Not all critics were impressed with the album, with The Washington Post describing it as ''an album of cheesy synthesizers, ) }
\\
\speak{Student}{\bf what followed this album? }
\speak{Teacher}\colorbox{pink!25}{$\hookrightarrow$}
{ ``'' (Mr. Bungle supported this record with their first world tour, performing across North America, Europe and Australia during 1995 and 1996. In 1997, ) }
\\
\speak{Student}{\bf what is the highlight of this period in his career? }
\speak{Teacher}\colorbox{pink!25}{$\hookrightarrow$}
{ ``'' (The music was complex and unpredictable, with the band continuing with their shifts of musical style. ) }
\\
 \end{dialogue}\end{tcolorbox}\end{figure}

\end{document}

