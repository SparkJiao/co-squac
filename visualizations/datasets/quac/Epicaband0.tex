\documentclass[11pt,a4paper, onecolumn]{article}
\usepackage{times}
\usepackage{latexsym}
\usepackage{url}
\usepackage{textcomp}
\usepackage{bbm}
\usepackage{amsmath}
\usepackage{booktabs}
\usepackage{tabularx}
\usepackage{graphicx}
\usepackage{dialogue}
\usepackage{mathtools}
\usepackage{hyperref}
%\hypersetup{draft}

\usepackage{multirow}
\usepackage{mdframed}
\usepackage{tcolorbox}

\usepackage{xcolor,pifont}
%\newcommand{\cmark}{\ding{51}}
%\newcommand{\xmark}{\ding{55}}

\setcounter{topnumber}{2}
\setcounter{bottomnumber}{2}
\setcounter{totalnumber}{4}
\renewcommand{\topfraction}{0.75}
\renewcommand{\bottomfraction}{0.75}
\renewcommand{\textfraction}{0.05}
\renewcommand{\floatpagefraction}{0.6}

\newcommand\cmark {\textcolor{green}{\ding{52}}}
\newcommand\xmark {\textcolor{red}{\ding{55}}}
\mdfdefinestyle{dialogue}{
    backgroundcolor=yellow!20,
    innermargin=5pt
}
\usepackage{amssymb}
\usepackage{soul}
\makeatletter

\begin{document}

\hspace{15pt}{\textbf{Section}:Epica (band)0\\}
\\ Context: Their second release, entitled Consign to Oblivion, was influenced by the culture of the Maya civilization, which can be noticed on songs in the ''A New Age Dawns'' series. ''A New Age Dawns'' refers to the time system of the Mayan people, which extends up to 2012, and makes no reference of what may happen past said year. Consign to Oblivion was composed with film scores as a basis, with Hans Zimmer and Danny Elfman cited as major inspirations. The album features guest singing by Roy Khan (from Kamelot) on the song ''Trois Vierges''. Epica also joined Kamelot as a support band on parts of their tour for promotion of The Black Halo album, to which Simons had contributed her vocals on the track ''The Haunting (Somewhere in Time)''. Two singles were released from the album, ''Solitary Ground'' and ''Quietus''. Epica's non-metal album The Score - An Epic Journey was released in September 2005 and is the soundtrack for a Dutch movie called Joyride, though it could also be considered to be their third album. Mark Jansen describes the album as typical Epica, ''only without the singing, without the guitars, no bass and no drums''. In 2005 and 2006 Epica went on their first tour throughout North America with Kamelot. After the tour, drummer Jeroen Simons left the band because of his wish to pursue other musical interests. In Fall 2006, Simone once again contributed vocals to an album of Kamelot, this time on the tracks ''Blucher'' and ''Season's End'' on the album Ghost Opera. In December, Arien van Weesenbeek from God Dethroned was announced via Epica's official website as the guest drummer for their new album, but not as a permanent band member. CANNOTANSWER

\begin{figure}[t] \small \begin{tcolorbox}[boxsep=0pt,left=5pt,right=0pt,top=2pt,colback = yellow!5] \begin{dialogue}
 \small 
 \speak{Student}{\bf what was a song on this album? }
\speak{Teacher}\colorbox{pink!25}{$\not\hookrightarrow$}
{ ``'' (''A New Age Dawns'' ) }
\\
\speak{Student}{\bf who was a singer on one of these albums? }
\speak{Teacher}\colorbox{pink!25}{$\hookrightarrow$}
{ ``'' (The album features guest singing by Roy Khan ( ) }
\\
\speak{Student}{\bf were there any other singers? }
\speak{Teacher}\colorbox{pink!25}{$\not\hookrightarrow$}
{ ``'' (CANNOTANSWER ) }
\\
\speak{Student}{\bf when was one of the albums released? }
\speak{Teacher}\colorbox{pink!25}{$\hookrightarrow$}
{ ``'' (September 2005 ) }
\\
\speak{Student}{\bf did it sell well? }
\speak{Teacher}\colorbox{pink!25}{$\not\hookrightarrow$}
{ ``'' (CANNOTANSWER ) }
\\
 \end{dialogue}\end{tcolorbox}\end{figure}

\end{document}

