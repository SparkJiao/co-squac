\documentclass[11pt,a4paper, onecolumn]{article}
\usepackage{times}
\usepackage{latexsym}
\usepackage{url}
\usepackage{textcomp}
\usepackage{bbm}
\usepackage{amsmath}
\usepackage{booktabs}
\usepackage{tabularx}
\usepackage{graphicx}
\usepackage{dialogue}
\usepackage{mathtools}
\usepackage{hyperref}
%\hypersetup{draft}

\usepackage{multirow}
\usepackage{mdframed}
\usepackage{tcolorbox}

\usepackage{xcolor,pifont}
%\newcommand{\cmark}{\ding{51}}
%\newcommand{\xmark}{\ding{55}}

\setcounter{topnumber}{2}
\setcounter{bottomnumber}{2}
\setcounter{totalnumber}{4}
\renewcommand{\topfraction}{0.75}
\renewcommand{\bottomfraction}{0.75}
\renewcommand{\textfraction}{0.05}
\renewcommand{\floatpagefraction}{0.6}

\newcommand\cmark {\textcolor{green}{\ding{52}}}
\newcommand\xmark {\textcolor{red}{\ding{55}}}
\mdfdefinestyle{dialogue}{
    backgroundcolor=yellow!20,
    innermargin=5pt
}
\usepackage{amssymb}
\usepackage{soul}
\makeatletter

\begin{document}

\hspace{15pt}{\textbf{Section}:Louis XVIII of France0\\}
\\ Context: Louis brought his wife and queen, Marie Josephine, from mainland Europe in 1808. His stay at Gosfield Hall did not last long; he soon moved to Hartwell House in Buckinghamshire, where over one hundred courtiers were housed. The King paid PS500 in rent each year to the owner of the estate, Sir George Lee. The Prince of Wales (the future George IV of Great Britain) was very charitable to the exiled Bourbons. As Prince Regent, he granted them permanent right of asylum and extremely generous allowances. The Count of Artois did not join the court-in-exile in Hartwell, preferring to continue his frivolous life in London. Louis's friend the Count of Avaray left Hartwell for Madeira in 1809, and died there in 1811. Louis replaced Avaray with the Comte de Blacas as his principal political advisor. Queen Marie Josephine died on 13 November 1810. That same winter, Louis suffered a particularly severe attack of gout, which was a recurring problem for him at Hartwell, and he had to take to a wheelchair. Napoleon I embarked on an invasion of Russia in 1812. This war would prove to be the turning point in his fortunes, as the expedition failed miserably, and Napoleon was forced to retreat with an army in tatters. In 1813, Louis XVIII issued another declaration from Hartwell. The Declaration of Hartwell was even more liberal than his Declaration of 1805, asserting that all those who served Napoleon or the Republic would not suffer repercussions for their acts, and that the original owners of the Biens nationaux (lands confiscated from the nobility and clergy during the Revolution) were to be compensated for their losses. Allied troops entered Paris on 31 March 1814. Louis, however, was unable to walk, and so he had sent the Count of Artois to France in January 1814. Louis issued letters patent appointing Artois as Lieutenant-General of the Kingdom in the event of his being restored as king, and on 11 April, five days after the French Senate had invited Louis to resume the throne of France, the Emperor Napoleon I abdicated. CANNOTANSWER

\begin{figure}[t] \small \begin{tcolorbox}[boxsep=0pt,left=5pt,right=0pt,top=2pt,colback = yellow!5] \begin{dialogue}
 \small 
 \speak{Student}{\bf what did he do in england? }
\speak{Teacher}\colorbox{pink!25}{$\hookrightarrow$}
{ ``'' (The King paid PS500 in rent each year to the owner of the estate, Sir George Lee. ) }
\\
\speak{Student}{\bf why did he do this? }
\speak{Teacher}\colorbox{pink!25}{$\not\hookrightarrow$}
{ ``'' (granted them permanent right of asylum and extremely generous allowances. ) }
\\
\speak{Student}{\bf what did they need asylum from? }
\speak{Teacher}\colorbox{pink!25}{ $\bar{\hookrightarrow}$}
{ ``'' (CANNOTANSWER ) }
\\
\speak{Student}{\bf was louis liked in london? }
\speak{Teacher}\colorbox{pink!25}{$\not\hookrightarrow$}
{ ``'' (CANNOTANSWER ) }
\\
 \end{dialogue}\end{tcolorbox}\end{figure}

\end{document}

