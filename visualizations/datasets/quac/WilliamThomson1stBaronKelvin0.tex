\documentclass[11pt,a4paper, onecolumn]{article}
\usepackage{times}
\usepackage{latexsym}
\usepackage{url}
\usepackage{textcomp}
\usepackage{bbm}
\usepackage{amsmath}
\usepackage{booktabs}
\usepackage{tabularx}
\usepackage{graphicx}
\usepackage{dialogue}
\usepackage{mathtools}
\usepackage{hyperref}
%\hypersetup{draft}

\usepackage{multirow}
\usepackage{mdframed}
\usepackage{tcolorbox}

\usepackage{xcolor,pifont}
%\newcommand{\cmark}{\ding{51}}
%\newcommand{\xmark}{\ding{55}}

\setcounter{topnumber}{2}
\setcounter{bottomnumber}{2}
\setcounter{totalnumber}{4}
\renewcommand{\topfraction}{0.75}
\renewcommand{\bottomfraction}{0.75}
\renewcommand{\textfraction}{0.05}
\renewcommand{\floatpagefraction}{0.6}

\newcommand\cmark {\textcolor{green}{\ding{52}}}
\newcommand\xmark {\textcolor{red}{\ding{55}}}
\mdfdefinestyle{dialogue}{
    backgroundcolor=yellow!20,
    innermargin=5pt
}
\usepackage{amssymb}
\usepackage{soul}
\makeatletter

\begin{document}

\hspace{15pt}{\textbf{Section}:William Thomson, 1st Baron Kelvin0\\}
\\ Context: Thomson's fears were realized when Whitehouse's apparatus proved insufficiently sensitive and had to be replaced by Thomson's mirror galvanometer. Whitehouse continued to maintain that it was his equipment that was providing the service and started to engage in desperate measures to remedy some of the problems. He succeeded only in fatally damaging the cable by applying 2,000 V. When the cable failed completely Whitehouse was dismissed, though Thomson objected and was reprimanded by the board for his interference. Thomson subsequently regretted that he had acquiesced too readily to many of Whitehouse's proposals and had not challenged him with sufficient energy. A joint committee of inquiry was established by the Board of Trade and the Atlantic Telegraph Company. Most of the blame for the cable's failure was found to rest with Whitehouse. The committee found that, though underwater cables were notorious in their lack of reliability, most of the problems arose from known and avoidable causes. Thomson was appointed one of a five-member committee to recommend a specification for a new cable. The committee reported in October 1863. In July 1865, Thomson sailed on the cable-laying expedition of the SS Great Eastern but the voyage was again dogged by technical problems. The cable was lost after 1,200 miles (1,900 km) had been laid and the expedition had to be abandoned. A further expedition in 1866 managed to lay a new cable in two weeks and then go on to recover and complete the 1865 cable. The enterprise was now feted as a triumph by the public and Thomson enjoyed a large share of the adulation. Thomson, along with the other principals of the project, was knighted on 10 November 1866. To exploit his inventions for signalling on long submarine cables, Thomson now entered into a partnership with C.F. Varley and Fleeming Jenkin. In conjunction with the latter, he also devised an automatic curb sender, a kind of telegraph key for sending messages on a cable. CANNOTANSWER

\begin{figure}[t] \small \begin{tcolorbox}[boxsep=0pt,left=5pt,right=0pt,top=2pt,colback = yellow!5] \begin{dialogue}
 \small 
 \speak{Student}{\bf What was the disaster? }
\speak{Teacher}\colorbox{pink!25}{$\hookrightarrow$}
{ ``'' (The cable was lost after 1,200 miles (1,900 km) had been laid and the expedition had to be abandoned. ) }
\\
\speak{Student}{\bf In what year did this happen? }
\speak{Teacher}\colorbox{pink!25}{$\hookrightarrow$}
{ ``'' (In July 1865, ) }
\\
\speak{Student}{\bf What happened after the expedition was abandoned? }
\speak{Teacher}\colorbox{pink!25}{ $\bar{\hookrightarrow}$}
{ ``'' (A joint committee of inquiry was established by the Board of Trade and the Atlantic Telegraph Company. ) }
\\
\speak{Student}{\bf What was the triumph? }
\speak{Teacher}\colorbox{pink!25}{ $\bar{\hookrightarrow}$}
{ ``'' (he also devised an automatic curb sender, a kind of telegraph key for sending messages on a cable. ) }
\\
 \end{dialogue}\end{tcolorbox}\end{figure}

\end{document}

