\documentclass[11pt,a4paper, onecolumn]{article}
\usepackage{times}
\usepackage{latexsym}
\usepackage{url}
\usepackage{textcomp}
\usepackage{bbm}
\usepackage{amsmath}
\usepackage{booktabs}
\usepackage{tabularx}
\usepackage{graphicx}
\usepackage{dialogue}
\usepackage{mathtools}
\usepackage{hyperref}
%\hypersetup{draft}

\usepackage{multirow}
\usepackage{mdframed}
\usepackage{tcolorbox}

\usepackage{xcolor,pifont}
%\newcommand{\cmark}{\ding{51}}
%\newcommand{\xmark}{\ding{55}}

\setcounter{topnumber}{2}
\setcounter{bottomnumber}{2}
\setcounter{totalnumber}{4}
\renewcommand{\topfraction}{0.75}
\renewcommand{\bottomfraction}{0.75}
\renewcommand{\textfraction}{0.05}
\renewcommand{\floatpagefraction}{0.6}

\newcommand\cmark {\textcolor{green}{\ding{52}}}
\newcommand\xmark {\textcolor{red}{\ding{55}}}
\mdfdefinestyle{dialogue}{
    backgroundcolor=yellow!20,
    innermargin=5pt
}
\usepackage{amssymb}
\usepackage{soul}
\makeatletter

\begin{document}

\hspace{15pt}{\textbf{Section}:Bernard Lewis0\\}
\\ Context: In 1936, Lewis graduated from the School of Oriental Studies (now School of Oriental and African Studies, SOAS) at the University of London with a BA in history with special reference to the Near and Middle East. He earned his PhD three years later, also from SOAS, specializing in the history of Islam. Lewis also studied law, going part of the way toward becoming a solicitor, but returned to study Middle Eastern history. He undertook post-graduate studies at the University of Paris, where he studied with the orientalist Louis Massignon and earned the ''Diplome des Etudes Semitiques'' in 1937. He returned to SOAS in 1938 as an assistant lecturer in Islamic History. During the Second World War, Lewis served in the British Army in the Royal Armoured Corps and as a Corporal in the Intelligence Corps in 1940-41 before being seconded to the Foreign Office. After the war, he returned to SOAS. In 1949, at the age of 33, he was appointed to the new chair in Near and Middle Eastern History. In 1974, aged 57, Lewis accepted a joint position at Princeton University and the Institute for Advanced Study, also located in Princeton, New Jersey. The terms of his appointment were such that Lewis taught only one semester per year, and being free from administrative responsibilities, he could devote more time to research than previously. Consequently, Lewis's arrival at Princeton marked the beginning of the most prolific period in his research career during which he published numerous books and articles based on previously accumulated materials. After retiring from Princeton in 1986, Lewis served at Cornell University until 1990. In 1966, Lewis was a founding member of the learned society, Middle East Studies Association of North America (MESA), but in 2007 he broke away and founded Association for the Study of the Middle East and Africa (ASMEA) to challenge MESA, which the New York Sun noted as ''dominated by academics who have been critical of Israel and of America's role in the Middle East.'' The organization was formed as an academic society dedicated to promoting high standards of research and teaching in Middle Eastern and African studies and other related fields, with Lewis as Chairman of its academic council. In 1990, the National Endowment for the Humanities selected Lewis for the Jefferson Lecture, the U.S. federal government's highest honor for achievement in the humanities. His lecture, entitled ''Western Civilization: A View from the East'', was revised and reprinted in The Atlantic Monthly under the title ''The Roots of Muslim Rage.'' His 2007 Irving Kristol Lecture, given to the American Enterprise Institute, was published as Europe and Islam. Lewis' influence extends beyond academia to the general public. He is a pioneer of the social and economic history of the Middle East and is famous for his extensive research of the Ottoman archives. He began his research career with the study of medieval Arab, especially Syrian, history. His first article, dedicated to professional guilds of medieval Islam, had been widely regarded as the most authoritative work on the subject for about thirty years. However, after the establishment of the state of Israel in 1948, scholars of Jewish origin found it more and more difficult to conduct archival and field research in the Arab countries, where they were suspected of espionage. Therefore, Lewis switched to the study of the Ottoman Empire, while continuing to research Arab history through the Ottoman archives which had only recently been opened to Western researchers. A series of articles that Lewis published over the next several years revolutionized the history of the Middle East by giving a broad picture of Islamic society, including its government, economy, and demographics. Lewis argues that the Middle East is currently backward and its decline was a largely self-inflicted condition resulting from both culture and religion, as opposed to the post-colonialist view which posits the problems of the region as economic and political maldevelopment mainly due to the 19th-century European colonization. In his 1982 work Muslim Discovery of Europe, Lewis argues that Muslim societies could not keep pace with the West and that ''Crusader successes were due in no small part to Muslim weakness.'' Further, he suggested that as early as the 11th century Islamic societies were decaying, primarily the byproduct of internal problems like ''cultural arrogance,'' which was a barrier to creative borrowing, rather than external pressures like the Crusades. In the wake of Soviet and Arab attempts to delegitimize Israel as a racist country, Lewis wrote a study of anti-Semitism, Semites and Anti-Semites (1986). In other works he argued Arab rage against Israel was disproportionate to other tragedies or injustices in the Muslim world, such as the Soviet invasion of Afghanistan and control of Muslim-majority land in Central Asia, the bloody and destructive fighting during the Hama uprising in Syria (1982), the Algerian civil war (1992-98), and the Iran-Iraq War (1980-88). In addition to his scholarly works, Lewis wrote several influential books accessible to the general public: The Arabs in History (1950), The Middle East and the West (1964), and The Middle East (1995). In the wake of the September 11, 2001 attacks, the interest in Lewis's work surged, especially his 1990 essay The Roots of Muslim Rage. Three of his books were published after 9/11: What Went Wrong? (written before the attacks), which explored the reasons of the Muslim world's apprehension of (and sometimes outright hostility to) modernization; The Crisis of Islam; and Islam: The Religion and the People. In the mid-1960s, Lewis emerged as a commentator on the issues of the modern Middle East and his analysis of the Israeli-Palestinian conflict and the rise of militant Islam brought him publicity and aroused significant controversy. American historian Joel Beinin has called him ''perhaps the most articulate and learned Zionist advocate in the North American Middle East academic community''. Lewis's policy advice has particular weight thanks to this scholarly authority. U.S. Vice President Dick Cheney remarked ''in this new century, his wisdom is sought daily by policymakers, diplomats, fellow academics, and the news media.'' A harsh critic of the Soviet Union, Lewis continued the liberal tradition in Islamic historical studies. Although his early Marxist views had a bearing on his first book The Origins of Ismailism, Lewis subsequently discarded Marxism. His later works are a reaction against the left-wing current of Third-worldism which came to be a significant current in Middle Eastern studies. Lewis advocated closer Western ties with Israel and Turkey, which he saw as especially important in light of the extension of the Soviet influence in the Middle East. Modern Turkey holds a special place in Lewis's view of the region due to the country's efforts to become a part of the West. He is an Honorary Fellow of the Institute of Turkish Studies, an honor which is given ''on the basis of generally recognized scholarly distinction and ... long and devoted service to the field of Turkish Studies.'' Lewis views Christendom and Islam as civilizations that have been in perpetual collision since the advent of Islam in the 7th century. In his essay The Roots of Muslim Rage (1990), he argued that the struggle between the West and Islam was gathering strength. According to one source, this essay (and Lewis' 1990 Jefferson Lecture on which the article was based) first introduced the term ''Islamic fundamentalism'' to North America. This essay has been credited with coining the phrase ''clash of civilizations'', which received prominence in the eponymous book by Samuel Huntington. However, another source indicates that Lewis first used the phrase ''clash of civilizations'' at a 1957 meeting in Washington where it was recorded in the transcript. In 1998, Lewis read in a London-based newspaper Al-Quds Al-Arabi a declaration of war on the United States by Osama bin Laden. In his essay ''A License to Kill'', Lewis indicated he considered bin Laden's language as the ''ideology of jihad'' and warned that bin Laden would be a danger to the West. The essay was published after the Clinton administration and the US intelligence community had begun its hunt for bin Laden in Sudan and then in Afghanistan. CANNOTANSWER

\begin{figure}[t] \small \begin{tcolorbox}[boxsep=0pt,left=5pt,right=0pt,top=2pt,colback = yellow!5] \begin{dialogue}
 \small 
 \speak{Student}{\bf What was Bernard's view and influence on contemporary politics? }
\speak{Teacher}\colorbox{pink!25}{$\hookrightarrow$}
{ ``'' (Lewis argues that the Middle East is currently backward and its decline was a largely self-inflicted condition resulting from both culture and religion, ) }
\\
\speak{Student}{\bf Who is the person hat oppose this view? }
\speak{Teacher}\colorbox{pink!25}{$\not\hookrightarrow$}
{ ``'' (CANNOTANSWER ) }
\\
\speak{Student}{\bf Does the view has any oposition? }
\speak{Teacher}\colorbox{pink!25}{$\not\hookrightarrow$}
{ ``'' (CANNOTANSWER ) }
\\
\speak{Student}{\bf Are there any other interesting aspects about this article? }
\speak{Teacher}\colorbox{pink!25}{$\hookrightarrow$}
\colorbox{red!25}{Yes,}
{ ``'' (In addition to his scholarly works, Lewis wrote several influential books accessible to the general public: ) }
\\
 \end{dialogue}\end{tcolorbox}\end{figure}

\end{document}

