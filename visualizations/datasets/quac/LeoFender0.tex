\documentclass[11pt,a4paper, onecolumn]{article}
\usepackage{times}
\usepackage{latexsym}
\usepackage{url}
\usepackage{textcomp}
\usepackage{bbm}
\usepackage{amsmath}
\usepackage{booktabs}
\usepackage{tabularx}
\usepackage{graphicx}
\usepackage{dialogue}
\usepackage{mathtools}
\usepackage{hyperref}
%\hypersetup{draft}

\usepackage{multirow}
\usepackage{mdframed}
\usepackage{tcolorbox}

\usepackage{xcolor,pifont}
%\newcommand{\cmark}{\ding{51}}
%\newcommand{\xmark}{\ding{55}}

\setcounter{topnumber}{2}
\setcounter{bottomnumber}{2}
\setcounter{totalnumber}{4}
\renewcommand{\topfraction}{0.75}
\renewcommand{\bottomfraction}{0.75}
\renewcommand{\textfraction}{0.05}
\renewcommand{\floatpagefraction}{0.6}

\newcommand\cmark {\textcolor{green}{\ding{52}}}
\newcommand\xmark {\textcolor{red}{\ding{55}}}
\mdfdefinestyle{dialogue}{
    backgroundcolor=yellow!20,
    innermargin=5pt
}
\usepackage{amssymb}
\usepackage{soul}
\makeatletter

\begin{document}

\hspace{15pt}{\textbf{Section}:Leo Fender0\\}
\\ Context: Clarence Leonidas Fender (''Leo'') was born on August 10, 1909, to Clarence Monte Fender and Harriet Elvira Wood, owners of a successful orange grove located between Anaheim and Fullerton, California. From an early age, Fender showed an interest in tinkering with electronics. When he was 13 years old, his uncle, who ran an automotive-electric shop, sent him a box filled with discarded car radio parts, and a battery. The following year, Leo visited his uncle's shop in Santa Maria, California, and was fascinated by a radio his uncle had built from spare parts and placed on display in the front of the shop. Leo later claimed that the loud music coming from the speaker of that radio made a lasting impression on him. Soon thereafter, Leo began repairing radios in a small shop in his parents' home. In the spring of 1928, Fender graduated from Fullerton Union High School, and entered Fullerton Junior College that fall, as an accounting major. While he was studying to be an accountant, he continued to teach himself electronics, and tinker with radios and other electrical items but never took any kind of electronics course. After college, Fender took a job as a delivery man for Consolidated Ice and Cold Storage Company in Anaheim, where he was later made the bookkeeper. It was around this time that a local band leader approached Leo, asking him if he could build a public address system for use by the band at dances in Hollywood. Fender was contracted to build six of these PA systems. In 1933, Fender met Esther Klosky, and they were married in 1934. About that time, he took a job as an accountant for the California Highway Department in San Luis Obispo. In a depression government change, his job was eliminated, and he then took a job in the accounting department of a tire company. After working there for six months, Leo lost his job along with the other accountants in the company. CANNOTANSWER

\begin{figure}[t] \small \begin{tcolorbox}[boxsep=0pt,left=5pt,right=0pt,top=2pt,colback = yellow!5] \begin{dialogue}
 \small 
 \speak{Student}{\bf Is there a written biography? }
\speak{Teacher}\colorbox{pink!25}{$\not\hookrightarrow$}
{ ``'' (CANNOTANSWER ) }
\\
\speak{Student}{\bf Where was Leo Fender born? }
\speak{Teacher}\colorbox{pink!25}{$\hookrightarrow$}
{ ``'' (CANNOTANSWER ) }
\\
\speak{Student}{\bf Does he have any siblings? }
\speak{Teacher}\colorbox{pink!25}{$\not\hookrightarrow$}
{ ``'' (CANNOTANSWER ) }
\\
\speak{Student}{\bf When was he born? }
\speak{Teacher}\colorbox{pink!25}{$\hookrightarrow$}
{ ``'' (was born on August 10, 1909, ) }
\\
\speak{Student}{\bf What were his parents names? }
\speak{Teacher}\colorbox{pink!25}{$\hookrightarrow$}
{ ``'' (Clarence Monte Fender and Harriet Elvira ) }
\\
 \end{dialogue}\end{tcolorbox}\end{figure}

\end{document}

