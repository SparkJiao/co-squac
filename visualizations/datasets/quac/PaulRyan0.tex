\documentclass[11pt,a4paper, onecolumn]{article}
\usepackage{times}
\usepackage{latexsym}
\usepackage{url}
\usepackage{textcomp}
\usepackage{bbm}
\usepackage{amsmath}
\usepackage{booktabs}
\usepackage{tabularx}
\usepackage{graphicx}
\usepackage{dialogue}
\usepackage{mathtools}
\usepackage{hyperref}
%\hypersetup{draft}

\usepackage{multirow}
\usepackage{mdframed}
\usepackage{tcolorbox}

\usepackage{xcolor,pifont}
%\newcommand{\cmark}{\ding{51}}
%\newcommand{\xmark}{\ding{55}}

\setcounter{topnumber}{2}
\setcounter{bottomnumber}{2}
\setcounter{totalnumber}{4}
\renewcommand{\topfraction}{0.75}
\renewcommand{\bottomfraction}{0.75}
\renewcommand{\textfraction}{0.05}
\renewcommand{\floatpagefraction}{0.6}

\newcommand\cmark {\textcolor{green}{\ding{52}}}
\newcommand\xmark {\textcolor{red}{\ding{55}}}
\mdfdefinestyle{dialogue}{
    backgroundcolor=yellow!20,
    innermargin=5pt
}
\usepackage{amssymb}
\usepackage{soul}
\makeatletter

\begin{document}

\hspace{15pt}{\textbf{Section}:Paul Ryan0\\}
\\ Context: On October 8, 2015, a push by congressional Republicans to recruit Ryan to run to succeed John Boehner as Speaker of the House was initiated. Boehner had recently announced his resignation and stated his support for Kevin McCarthy to be his replacement, which received wide support among Republicans, including Ryan, who was set to officially nominate him. McCarthy withdrew his name from consideration on October 8 when it was apparent that the Freedom Caucus, a caucus of staunchly conservative House Republicans, would not support him. This led many Republicans to turn to Ryan as a compromise candidate. The push included a plea from Boehner, who reportedly told Ryan that he was the only person who could unite the House Republicans at a time of turmoil. Ryan released a statement that said, ''While I am grateful for the encouragement I've received, I will not be a candidate.'' But on October 9, close aides of Ryan confirmed that Ryan had reconsidered, and was considering the possibility of a run. Ryan confirmed on October 22 that he would seek the speakership after receiving the endorsements of two factions of House Republicans, including the conservative Freedom Caucus. Ryan, upon confirming his bid for the speakership, stated, ''I never thought I'd be speaker. But I pledged to you that if I could be a unifying figure, then I would serve -- I would go all in. After talking with so many of you, and hearing your words of encouragement, I believe we are ready to move forward as one, united team. And I am ready and eager to be our speaker.'' On October 29, Ryan was elected Speaker with 236 votes. He is the youngest Speaker since James G. Blaine in 1875. CANNOTANSWER

\begin{figure}[t] \small \begin{tcolorbox}[boxsep=0pt,left=5pt,right=0pt,top=2pt,colback = yellow!5] \begin{dialogue}
 \small 
 \speak{Student}{\bf When did he run for speaker of the house? }
\speak{Teacher}\colorbox{pink!25}{$\hookrightarrow$}
{ ``'' (On October 8, 2015, a push by congressional Republicans to recruit Ryan to run to succeed John Boehner as Speaker of the House was initiated. ) }
\\
\speak{Student}{\bf What policies did he help implement? }
\speak{Teacher}\colorbox{pink!25}{$\not\hookrightarrow$}
{ ``'' (CANNOTANSWER ) }
\\
\speak{Student}{\bf Can you tell me more about his bid for the seat? }
\speak{Teacher}\colorbox{pink!25}{$\hookrightarrow$}
{ ``'' (Boehner had recently announced his resignation and stated his support for Kevin McCarthy to be his replacement, ) }
\\
\speak{Student}{\bf Why did he resign? }
\speak{Teacher}\colorbox{pink!25}{$\not\hookrightarrow$}
{ ``'' (CANNOTANSWER ) }
\\
\speak{Student}{\bf What did he do while in office? }
\speak{Teacher}\colorbox{pink!25}{$\not\hookrightarrow$}
{ ``'' (CANNOTANSWER ) }
\\
 \end{dialogue}\end{tcolorbox}\end{figure}

\end{document}

