\documentclass[11pt,a4paper, onecolumn]{article}
\usepackage{times}
\usepackage{latexsym}
\usepackage{url}
\usepackage{textcomp}
\usepackage{bbm}
\usepackage{amsmath}
\usepackage{booktabs}
\usepackage{tabularx}
\usepackage{graphicx}
\usepackage{dialogue}
\usepackage{mathtools}
\usepackage{hyperref}
%\hypersetup{draft}

\usepackage{multirow}
\usepackage{mdframed}
\usepackage{tcolorbox}

\usepackage{xcolor,pifont}
%\newcommand{\cmark}{\ding{51}}
%\newcommand{\xmark}{\ding{55}}

\setcounter{topnumber}{2}
\setcounter{bottomnumber}{2}
\setcounter{totalnumber}{4}
\renewcommand{\topfraction}{0.75}
\renewcommand{\bottomfraction}{0.75}
\renewcommand{\textfraction}{0.05}
\renewcommand{\floatpagefraction}{0.6}

\newcommand\cmark {\textcolor{green}{\ding{52}}}
\newcommand\xmark {\textcolor{red}{\ding{55}}}
\mdfdefinestyle{dialogue}{
    backgroundcolor=yellow!20,
    innermargin=5pt
}
\usepackage{amssymb}
\usepackage{soul}
\makeatletter

\begin{document}

\hspace{15pt}{\textbf{Section}:Katherine Harris0\\}
\\ Context: The Pensacola News Journal suggested that Harris might withdraw from the Senate race after winning a primary victory, thereby allowing the Republicans to nominate another candidate, such as Tom Gallagher, to run against Bill Nelson (politician). In August, Katherine Harris touted political endorsements from fellow Republican lawmakers on her campaign web site. However, some of those cited claim that they never endorsed her. This conflict resulted in several Republican congressmen calling the Harris campaign to complain after the St. Petersburg Times notified them of the endorsements listed on Harris's Web site. A short time later, their names were removed without comment from Harris's Web site. Of Harris's three primary opponents, only Will McBride endorsed her candidacy for the general election. In the first few days after the primary, a number of Republican nominees such as Charlie Crist and Tom Lee went on a statewide unity tour with Gov. Bush. Harris was not invited; Republicans said the tour was only for nominees to statewide offices. Harris claimed Bush would campaign with her sometime in the two months before the election, but the governor's office denied this. President Bush did not make public appearances or private meetings with Harris before the primary. He did, however, appear with her at a fundraiser on September 21 in Tampa. When it came time for newspapers to make their op-ed endorsements, all 22 of Florida's major daily newspapers supported Senator Nelson. The only endorsement Harris received was from the Polk County Democrat, a newspaper in Bartow which publishes four days out of the week. CANNOTANSWER

\begin{figure}[t] \small \begin{tcolorbox}[boxsep=0pt,left=5pt,right=0pt,top=2pt,colback = yellow!5] \begin{dialogue}
 \small 
 \speak{Student}{\bf Why does Harris not have the Republican support? }
\speak{Teacher}\colorbox{pink!25}{$\hookrightarrow$}
{ ``'' (Harris touted political endorsements from fellow Republican lawmakers on her campaign web site. However, some of those cited claim that they never endorsed her. ) }
\\
\speak{Student}{\bf Who she did claim did support her, but didn't? }
\speak{Teacher}\colorbox{pink!25}{ $\bar{\hookrightarrow}$}
{ ``'' (several Republican congressmen calling the Harris campaign to complain ) }
\\
\speak{Student}{\bf Did she issue a statement? }
\speak{Teacher}\colorbox{pink!25}{$\not\hookrightarrow$}
\colorbox{red!25}{No,}
{ ``'' (A short time later, their names were removed without comment from Harris's Web site. ) }
\\
\speak{Student}{\bf Has she done anything illegal? }
\speak{Teacher}\colorbox{pink!25}{$\not\hookrightarrow$}
{ ``'' (CANNOTANSWER ) }
\\
\speak{Student}{\bf Did she run for office anyway? }
\speak{Teacher}\colorbox{pink!25}{$\hookrightarrow$}
\colorbox{red!25}{Yes,}
{ ``'' (Of Harris's three primary opponents, only Will McBride endorsed her candidacy for the general election. ) }
\\
 \end{dialogue}\end{tcolorbox}\end{figure}

\end{document}

