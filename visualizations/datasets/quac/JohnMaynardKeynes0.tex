\documentclass[11pt,a4paper, onecolumn]{article}
\usepackage{times}
\usepackage{latexsym}
\usepackage{url}
\usepackage{textcomp}
\usepackage{bbm}
\usepackage{amsmath}
\usepackage{booktabs}
\usepackage{tabularx}
\usepackage{graphicx}
\usepackage{dialogue}
\usepackage{mathtools}
\usepackage{hyperref}
%\hypersetup{draft}

\usepackage{multirow}
\usepackage{mdframed}
\usepackage{tcolorbox}

\usepackage{xcolor,pifont}
%\newcommand{\cmark}{\ding{51}}
%\newcommand{\xmark}{\ding{55}}

\setcounter{topnumber}{2}
\setcounter{bottomnumber}{2}
\setcounter{totalnumber}{4}
\renewcommand{\topfraction}{0.75}
\renewcommand{\bottomfraction}{0.75}
\renewcommand{\textfraction}{0.05}
\renewcommand{\floatpagefraction}{0.6}

\newcommand\cmark {\textcolor{green}{\ding{52}}}
\newcommand\xmark {\textcolor{red}{\ding{55}}}
\mdfdefinestyle{dialogue}{
    backgroundcolor=yellow!20,
    innermargin=5pt
}
\usepackage{amssymb}
\usepackage{soul}
\makeatletter

\begin{document}

\hspace{15pt}{\textbf{Section}:John Maynard Keynes0\\}
\\ Context: During the Second World War, Keynes argued in How to Pay for the War, published in 1940, that the war effort should be largely financed by higher taxation and especially by compulsory saving (essentially workers lending money to the government), rather than deficit spending, in order to avoid inflation. Compulsory saving would act to dampen domestic demand, assist in channelling additional output towards the war efforts, would be fairer than punitive taxation and would have the advantage of helping to avoid a post war slump by boosting demand once workers were allowed to withdraw their savings. In September 1941 he was proposed to fill a vacancy in the Court of Directors of the Bank of England, and subsequently carried out a full term from the following April. In June 1942, Keynes was rewarded for his service with a hereditary peerage in the King's Birthday Honours. On 7 July his title was gazetted as ''Baron Keynes, of Tilton, in the County of Sussex'' and he took his seat in the House of Lords on the Liberal Party benches. As the Allied victory began to look certain, Keynes was heavily involved, as leader of the British delegation and chairman of the World Bank commission, in the mid-1944 negotiations that established the Bretton Woods system. The Keynes-plan, concerning an international clearing-union, argued for a radical system for the management of currencies. He proposed the creation of a common world unit of currency, the bancor, and new global institutions - a world central bank and the International Clearing Union. Keynes envisaged these institutions managing an international trade and payments system with strong incentives for countries to avoid substantial trade deficits or surpluses. The USA's greater negotiating strength, however, meant that the final outcomes accorded more closely to the more conservative plans of Harry Dexter White. According to US economist J. Bradford DeLong, on almost every point where he was overruled by the Americans, Keynes was later proved correct by events. The two new institutions, later known as the World Bank and the International Monetary Fund (IMF), were founded as a compromise that primarily reflected the American vision. There would be no incentives for states to avoid a large trade surplus; instead, the burden for correcting a trade imbalance would continue to fall only on the deficit countries, which Keynes had argued were least able to address the problem without inflicting economic hardship on their populations. Yet, Keynes was still pleased when accepting the final agreement, saying that if the institutions stayed true to their founding principles, ''the brotherhood of man will have become more than a phrase.'' CANNOTANSWER

\begin{figure}[t] \small \begin{tcolorbox}[boxsep=0pt,left=5pt,right=0pt,top=2pt,colback = yellow!5] \begin{dialogue}
 \small 
 \speak{Student}{\bf What role did Keynes play in the Second World War? }
\speak{Teacher}\colorbox{pink!25}{ $\bar{\hookrightarrow}$}
{ ``'' (Keynes argued in How to Pay for the War, ) }
\\
\speak{Student}{\bf What position did he hold? }
\speak{Teacher}\colorbox{pink!25}{ $\bar{\hookrightarrow}$}
{ ``'' (Baron Keynes, of Tilton, ) }
\\
\speak{Student}{\bf Was he instrumental in helping the government pay for the Second World War? }
\speak{Teacher}\colorbox{pink!25}{ $\bar{\hookrightarrow}$}
\colorbox{red!25}{Yes,}
{ ``'' (Compulsory saving would act to dampen domestic demand, ) }
\\
\speak{Student}{\bf What method did he use to pay for the Second World War? }
\speak{Teacher}\colorbox{pink!25}{ $\bar{\hookrightarrow}$}
{ ``'' (that the war effort should be largely financed by higher taxation and especially by compulsory saving ) }
\\
\speak{Student}{\bf Was he praised for his work during the Second World War? }
\speak{Teacher}\colorbox{pink!25}{ $\bar{\hookrightarrow}$}
\colorbox{red!25}{Yes,}
{ ``'' (Keynes was rewarded for his service with a hereditary peerage in the King's Birthday Honours. ) }
\\
\speak{Student}{\bf Was he appointed to a special position? }
\speak{Teacher}\colorbox{pink!25}{ $\bar{\hookrightarrow}$}
\colorbox{red!25}{Yes,}
{ ``'' (On 7 July his title was gazetted as ''Baron Keynes, of Tilton, in the County of Sussex'' ) }
\\
\speak{Student}{\bf Is there anything interesting about his role in the Second World War? }
\speak{Teacher}\colorbox{pink!25}{$\not\hookrightarrow$}
{ ``'' (He proposed the creation of a common world unit of currency, the bancor, and new global institutions - ) }
\\
 \end{dialogue}\end{tcolorbox}\end{figure}

\end{document}

