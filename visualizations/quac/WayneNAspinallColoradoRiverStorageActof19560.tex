\documentclass[11pt,a4paper, onecolumn]{article}
\usepackage{times}
\usepackage{latexsym}
\usepackage{url}
\usepackage{textcomp}
\usepackage{bbm}
\usepackage{amsmath}
\usepackage{booktabs}
\usepackage{tabularx}
\usepackage{graphicx}
\usepackage{dialogue}
\usepackage{mathtools}
\usepackage{hyperref}
%\hypersetup{draft}

\usepackage{multirow}
\usepackage{mdframed}
\usepackage{tcolorbox}

\usepackage{xcolor,pifont}
%\newcommand{\cmark}{\ding{51}}
%\newcommand{\xmark}{\ding{55}}

\setcounter{topnumber}{2}
\setcounter{bottomnumber}{2}
\setcounter{totalnumber}{4}
\renewcommand{\topfraction}{0.75}
\renewcommand{\bottomfraction}{0.75}
\renewcommand{\textfraction}{0.05}
\renewcommand{\floatpagefraction}{0.6}

\newcommand\cmark {\textcolor{green}{\ding{52}}}
\newcommand\xmark {\textcolor{red}{\ding{55}}}
\mdfdefinestyle{dialogue}{
    backgroundcolor=yellow!20,
    innermargin=5pt
}
\usepackage{amssymb}
\usepackage{soul}
\makeatletter

\begin{document}

\hspace{15pt}{\textbf{Section}:Wayne N. Aspinall -- Colorado River Storage Act of 19560\\}
\\ Context: Aspinall favored dams and water reclamation projects for several reasons: (1) the power they generated; (2) general recreational use; and (3) he felt the key to Western economic prosperity lay in obtaining permanent stored supply of water for economic purposes. In Aspinall's mind, Americans had many opportunities to enjoy scenic areas, so damming a few of them would not hurt the country. After his career, he boasted that he had brought over  1 billion worth of water projects to his district. According to his observers, he ''never met a dam he didn't like.'' The Colorado River Storage Project (CRSP) came before Congress in the early to mid-1950s. The bill, sponsored by Wayne Aspinall and several western allies, called for damming several areas in the Upper Basin of the Colorado River. It included the Echo Park Dam proposal, located within Dinosaur National Monument. This became a volatile issue between environmentalists and water project advocates. Located just below the Green and Yampa Rivers, the proposed 525-foot (160 m) high dam would have created a 43,000 acre (170 km2) lake, flooding the Green River Canyon for 63 miles (101 km) and the Yampa River Canyon for 44 miles (71 km). However, because of increased environmental awareness around the country, the majority of the public opposed the project. In 1954 alone, the Department of the Interior received 20,000 pieces of mail about the plan, and one insider estimated the letters ran 80 to 1 in opposition of the project. In 1954 and 1955, environmentalists defeated the controversial aspects of the CRSP. In '55, Aspinall conceded that for the bill to pass, he needed to sacrifice the Echo Park plan. Several more compromises between the two sides gave way to the Colorado River Storage Act of 1956 (Public Law 485, in chapter 203), which called for the creation of other irrigation projects and several large dams, including the Glen Canyon Dam in Utah, which created Lake Powell. Other projects created from the CRSP included the Flaming Gorge, Navajo, and Curecanti reservoirs. For his compromise, Aspinall gained five other reclamation projects and three hydroelectric dams for the Fourth District. The bill passed the House 256-136 in March 1956 and then-President Dwight Eisenhower signed it into law soon thereafter. CANNOTANSWER

\begin{figure}[t] \small \begin{tcolorbox}[boxsep=0pt,left=5pt,right=0pt,top=2pt,colback = yellow!5] \begin{dialogue}
 \small 
 \speak{Student}{\bf What was the Colorado River Storage Act of 1956? }
\speak{Teacher}\colorbox{pink!25}{$\hookrightarrow$}
{ The bill, sponsored by Wayne Aspinall and several western allies, called for damming several areas in the Upper Basin of the Colorado River. }
\\
\speak{Student}{\bf Was the bill easy to pass? }
\speak{Teacher}\colorbox{pink!25}{ $\bar{\hookrightarrow}$}
\colorbox{red!25}{No,}
{ because of increased environmental awareness around the country, the majority of the public opposed the project. }
\\
\speak{Student}{\bf Was Aspinall eventually able to get the bill to pass through Congress? }
\speak{Teacher}\colorbox{pink!25}{ $\bar{\hookrightarrow}$}
\colorbox{red!25}{Yes,}
{ The bill passed the House 256-136 in March 1956 and then-President Dwight Eisenhower signed it into law soon thereafter. }
\\
\speak{Student}{\bf What was Aspinall political position during this time period? }
\speak{Teacher}\colorbox{pink!25}{$\not\hookrightarrow$}
{ Aspinall gained five other reclamation projects and three hydroelectric dams for the Fourth District. }
\\
\speak{Student}{\bf Why did the people oppose the Act? }
\speak{Teacher}\colorbox{pink!25}{$\not\hookrightarrow$}
{ This became a volatile issue between environmentalists and water project advocates. }
\\
\speak{Student}{\bf Is there anything else significant about Aspinall's role in the Colorado River Storage Act? }
\speak{Teacher}\colorbox{pink!25}{$\hookrightarrow$}
\colorbox{red!25}{Yes,}
{ In 1954 alone, the Department of the Interior received 20,000 pieces of mail about the plan, }
\\
\speak{Student}{\bf What were the nature of the mail received? }
\speak{Teacher}\colorbox{pink!25}{$\not\hookrightarrow$}
{ public opposed the project. }
\\
 \end{dialogue}\end{tcolorbox}\end{figure}

\end{document}

