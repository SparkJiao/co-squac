\documentclass[11pt,a4paper, onecolumn]{article}
\usepackage{times}
\usepackage{latexsym}
\usepackage{url}
\usepackage{textcomp}
\usepackage{bbm}
\usepackage{amsmath}
\usepackage{booktabs}
\usepackage{tabularx}
\usepackage{graphicx}
\usepackage{dialogue}
\usepackage{mathtools}
\usepackage{hyperref}
%\hypersetup{draft}

\usepackage{multirow}
\usepackage{mdframed}
\usepackage{tcolorbox}

\usepackage{xcolor,pifont}
%\newcommand{\cmark}{\ding{51}}
%\newcommand{\xmark}{\ding{55}}

\setcounter{topnumber}{2}
\setcounter{bottomnumber}{2}
\setcounter{totalnumber}{4}
\renewcommand{\topfraction}{0.75}
\renewcommand{\bottomfraction}{0.75}
\renewcommand{\textfraction}{0.05}
\renewcommand{\floatpagefraction}{0.6}

\newcommand\cmark {\textcolor{green}{\ding{52}}}
\newcommand\xmark {\textcolor{red}{\ding{55}}}
\mdfdefinestyle{dialogue}{
    backgroundcolor=yellow!20,
    innermargin=5pt
}
\usepackage{amssymb}
\usepackage{soul}
\makeatletter

\begin{document}

\hspace{15pt}{\textbf{Section}:Oprah Winfrey -- ''Oprahfication''0\\}
\\ Context: In 1993, Winfrey hosted a rare prime-time interview with Michael Jackson, which became the fourth most-watched event in American television history as well as the most watched interview ever, with an audience of 36.5 million. On December 1, 2005, Winfrey appeared on the Late Show with David Letterman to promote the new Broadway musical The Color Purple, of which she was a producer, joining the host for the first time in 16 years. The episode was hailed by some as the ''television event of the decade'' and helped Letterman attract his largest audience in more than 11 years: 13.45 million viewers. Although a much-rumored feud was said to have been the cause of the rift, both Winfrey and Letterman balked at such talk. ''I want you to know, it's really over, whatever you thought was happening'', said Winfrey. On September 10, 2007, Letterman made his first appearance on The Oprah Winfrey Show, as its season premiere was filmed in New York City. In 2006, rappers Ludacris, 50 Cent and Ice Cube criticized Winfrey for what they perceived as an anti-hip hop bias. In an interview with GQ magazine, Ludacris said that Winfrey gave him a ''hard time'' about his lyrics, and edited comments he made during an appearance on her show with the cast of the film Crash. He also said that he wasn't initially invited on the show with the rest of the cast. Winfrey responded by saying that she is opposed to rap lyrics that ''marginalize women'', but enjoys some artists, including Kanye West, who appeared on her show. She said she spoke with Ludacris backstage after his appearance to explain her position and said she understood that his music was for entertainment purposes, but that some of his listeners might take it literally. In September 2008, Winfrey received criticism after Matt Drudge of the Drudge Report reported that Winfrey refused to have Sarah Palin on her show, allegedly because of Winfrey's support for Barack Obama. Winfrey denied the report, maintaining that there never was a discussion regarding Palin's appearing on her show. She said that after she made public her support for Obama, she decided that she would not let her show be used as a platform for any of the candidates. Although Obama appeared twice on her show, those appearances were prior to his declaring himself a candidate. Winfrey added that Palin would make a fantastic guest and that she would love to have her on the show after the election, which she did on November 18, 2009. In 2009, Winfrey was criticized for allowing actress Suzanne Somers to appear on her show to discuss hormone treatments that are not accepted by mainstream medicine. Critics have also suggested that Winfrey is not tough enough when questioning celebrity guests or politicians whom she appears to like. Lisa de Moraes, a media columnist for The Washington Post, stated: ''Oprah doesn't do follow-up questions unless you're an author who's embarrassed her by fabricating portions of a supposed memoir she's plugged for her book club.'' In 1985, Winfrey co-starred in Steven Spielberg's The Color Purple as distraught housewife Sofia. She was nominated for an Academy Award for Best Supporting Actress for her performance. The Alice Walker novel went on to become a Broadway musical which opened in late 2005, with Winfrey credited as a producer. In October 1998, Winfrey produced and starred in the film Beloved, based on Toni Morrison's Pulitzer Prize-winning novel of the same name. To prepare for her role as Sethe, the protagonist and former slave, Winfrey experienced a 24-hour simulation of the experience of slavery, which included being tied up and blindfolded and left alone in the woods. Despite major advertising, including two episodes of her talk show dedicated solely to the film, and moderate to good critical reviews, Beloved opened to poor box-office results, losing approximately  30 million. While promoting the movie, co-star Thandie Newton described Winfrey as ''a very strong technical actress and it's because she's so smart. She's acute. She's got a mind like a razor blade.'' In 2005, Harpo Productions released a film adaptation of Zora Neale Hurston's 1937 novel Their Eyes Were Watching God. The made-for-television film was based upon a teleplay by Suzan-Lori Parks and starred Halle Berry in the lead female role. In late 2008, Winfrey's company Harpo Films signed an exclusive output pact to develop and produce scripted series, documentaries, and movies for HBO. Oprah voiced Gussie the goose in Charlotte's Web (2006) and voiced Judge Bumbleton in Bee Movie (2007), co-starring the voices of Jerry Seinfeld and Renee Zellweger. In 2009, Winfrey provided the voice for the character of Eudora, the mother of Princess Tiana, in Disney's The Princess and the Frog and in 2010, narrated the US version of the BBC nature program Life for Discovery. In 2018, Winfrey starred as Mrs. Which in the film adaptation of Madeleine L'Engle's novel A Wrinkle in Time. The Wall Street Journal coined the term ''Oprahfication'', meaning public confession as a form of therapy. By confessing intimate details about her weight problems, tumultuous love life, and sexual abuse, and crying alongside her guests, Time magazine credits Winfrey with creating a new form of media communication known as ''rapport talk'' as distinguished from the ''report talk'' of Phil Donahue: ''Winfrey saw television's power to blend public and private; while it links strangers and conveys information over public airwaves, TV is most often viewed in the privacy of our homes. Like a family member, it sits down to meals with us and talks to us in the lonely afternoons. Grasping this paradox, ... She makes people care because she cares. That is Winfrey's genius, and will be her legacy, as the changes she has wrought in the talk show continue to permeate our culture and shape our lives.'' Observers have also noted the ''Oprahfication'' of politics such as ''Oprah-style debates'' and Bill Clinton being described as ''the man who brought Oprah-style psychobabble and misty confessions to politics.'' Newsweek stated: ''Every time a politician lets his lip quiver or a cable anchor 'emotes' on TV, they nod to the cult of confession that Oprah helped create. The November 1988 Ms. observed that ''in a society where fat is taboo, she made it in a medium that worships thin and celebrates a bland, white-bread prettiness of body and personality [...] But Winfrey made fat sexy, elegant - damned near gorgeous - with her drop-dead wardrobe, easy body language, and cheerful sensuality.'' CANNOTANSWER

\begin{figure}[t] \small \begin{tcolorbox}[boxsep=0pt,left=5pt,right=0pt,top=2pt,colback = yellow!5] \begin{dialogue}
 \small 
 \speak{Student}{\bf What is Oprahfication? }
\speak{Teacher}\colorbox{pink!25}{$\hookrightarrow$}
{ The Wall Street Journal coined the term ''Oprahfication'', meaning }
\\
\speak{Student}{\bf what does this cause in society? }
\speak{Teacher}\colorbox{pink!25}{$\hookrightarrow$}
{ ... She makes people care because she cares. }
\\
\speak{Student}{\bf what is a cause she supports? }
\speak{Teacher}\colorbox{pink!25}{$\not\hookrightarrow$}
{ she is opposed to rap lyrics that ''marginalize women'', }
\\
\speak{Student}{\bf what did she bring to the viewers? }
\speak{Teacher}\colorbox{pink!25}{$\hookrightarrow$}
{ a new form of media communication known as ''rapport talk'' as distinguished from the ''report talk'' of Phil Donahue: }
\\
\speak{Student}{\bf what is a hallmark of Oprahfication? }
\speak{Teacher}\colorbox{pink!25}{ $\bar{\hookrightarrow}$}
{ She makes people care because she cares. That is Winfrey's genius, and will be her legacy, }
\\
\speak{Student}{\bf what topics did she cover? }
\speak{Teacher}\colorbox{pink!25}{$\hookrightarrow$}
{ intimate details about her weight problems, tumultuous love life, and sexual abuse, }
\\
\speak{Student}{\bf who did this effect the viewers? }
\speak{Teacher}\colorbox{pink!25}{ $\bar{\hookrightarrow}$}
{ the changes she has wrought in the talk show continue to permeate our culture and shape our lives. }
\\
\speak{Student}{\bf what do the critics think about her style? }
\speak{Teacher}\colorbox{pink!25}{ $\bar{\hookrightarrow}$}
{ ''Winfrey saw television's power to blend public and private; }
 \end{dialogue}\end{tcolorbox}\end{figure}

\end{document}

