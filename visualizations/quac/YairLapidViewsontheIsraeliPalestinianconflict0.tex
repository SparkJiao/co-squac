\documentclass[11pt,a4paper, onecolumn]{article}
\usepackage{times}
\usepackage{latexsym}
\usepackage{url}
\usepackage{textcomp}
\usepackage{bbm}
\usepackage{amsmath}
\usepackage{booktabs}
\usepackage{tabularx}
\usepackage{graphicx}
\usepackage{dialogue}
\usepackage{mathtools}
\usepackage{hyperref}
%\hypersetup{draft}

\usepackage{multirow}
\usepackage{mdframed}
\usepackage{tcolorbox}

\usepackage{xcolor,pifont}
%\newcommand{\cmark}{\ding{51}}
%\newcommand{\xmark}{\ding{55}}

\setcounter{topnumber}{2}
\setcounter{bottomnumber}{2}
\setcounter{totalnumber}{4}
\renewcommand{\topfraction}{0.75}
\renewcommand{\bottomfraction}{0.75}
\renewcommand{\textfraction}{0.05}
\renewcommand{\floatpagefraction}{0.6}

\newcommand\cmark {\textcolor{green}{\ding{52}}}
\newcommand\xmark {\textcolor{red}{\ding{55}}}
\mdfdefinestyle{dialogue}{
    backgroundcolor=yellow!20,
    innermargin=5pt
}
\usepackage{amssymb}
\usepackage{soul}
\makeatletter

\begin{document}

\hspace{15pt}{\textbf{Section}:Yair Lapid -- Views on the Israeli-Palestinian conflict0\\}
\\ Context: Lapid said that he would demand a resumption of negotiations between Israel and the Palestinian Authority. His party's platform calls for an outline of ''two states for two peoples'', while maintaining the large Israeli settlement blocks and ensuring the safety of Israel. In January 2013, just days before the election, Lapid said he won't join a cabinet that stalls peace talks with the Palestinian Authority, and added that the idea of a single country for both Israelis and Palestinians without a peace agreement would endanger the Jewish character of Israel. He said, ''We're not looking for a happy marriage with the Palestinians, but for a divorce agreement we can live with.''  As part of a future peace agreement, Lapid said that the Palestinians would have to recognize that the large West Bank settlement blocs of Ariel, Gush Etzion and Ma'aleh Adumim would remain within the State of Israel. According to Lapid, only granting Palestinians their own state could end the conflict and Jews and Arabs should live apart in two states, while Jerusalem should remain undivided under Israeli rule. Regarding the diplomatic stalemate in the Israeli-Palestinian peace process, Lapid said that ''Most of the blame belongs to the Palestinian side, and I am not sure that they as a people are ready to make peace with us.'' He has, however, dismissed as unrealistic the possibility of a comprehensive peace deal with the Palestinians. In June 2015, after the March 2015 elections, Yair Lapid visited the United States and after an hour long interview, American journalist Jeffrey Goldberg wrote that, ''Lapid is a leader of the great mass of disillusioned centrists in Israeli politics. He could conceivably be prime minister one day, assuming Benjamin Netanyahu, in whose previous cabinet he served, ever stops being prime minister. Now functioning as a kind of shadow foreign minister, Lapid argues that Israel must seize the diplomatic initiative with the Palestinians if it is to continue existing as a Jewish-majority democracy, and he is proposing a regional summit somewhat along the lines of the earlier Arab Peace Initiative. Lapid is not a left-winger--he has a particular sort of contempt for the Israeli left, born of the belief that leftists don't recognize the nature of the region in which they live. But he is also for territorial compromise as a political and moral necessity, and he sees Netanyahu leading Israel inexorably toward the abyss.'' In September 2015 Yair Lapid laid out his diplomatic vision in a major speech at Bar Ilan University  in which he said ''Israel's strategic goal needs to be a regional agreement that will lead to full and normal relations with the Arab world and the creation of a demilitarized independent Palestinian state alongside Israel. That's where Israel needs to head. Separation from the Palestinians with strict security measures will save the Jewish character of the state.'' CANNOTANSWER

\begin{figure}[t] \small \begin{tcolorbox}[boxsep=0pt,left=5pt,right=0pt,top=2pt,colback = yellow!5] \begin{dialogue}
 \small 
 \speak{Student}{\bf What were some of the views? }
\speak{Teacher}\colorbox{pink!25}{$\hookrightarrow$}
{ Lapid said that he would demand a resumption of negotiations between Israel and the Palestinian Authority. }
\\
\speak{Student}{\bf Was this view well received? }
\speak{Teacher}\colorbox{pink!25}{ $\bar{\hookrightarrow}$}
{ CANNOTANSWER }
\\
\speak{Student}{\bf What else did he have to say? }
\speak{Teacher}\colorbox{pink!25}{$\hookrightarrow$}
{ His party's platform calls for an outline of ''two states for two peoples'', while maintaining the large Israeli settlement blocks and ensuring the safety of Israel. }
\\
\speak{Student}{\bf What other views did he hold? }
\speak{Teacher}\colorbox{pink!25}{$\hookrightarrow$}
{ He said, ''We're not looking for a happy marriage with the Palestinians, but for a divorce agreement we can live with.'' }
\\
\speak{Student}{\bf How did he do in the polls with those views? }
\speak{Teacher}\colorbox{pink!25}{$\not\hookrightarrow$}
{ CANNOTANSWER }
\\
\speak{Student}{\bf Are there any other interesting aspects about this article? }
\speak{Teacher}\colorbox{pink!25}{$\hookrightarrow$}
\colorbox{red!25}{Yes,}
{ he has a particular sort of contempt for the Israeli left, born of the belief that leftists don't recognize the nature of the region in which they live. }
\\
 \end{dialogue}\end{tcolorbox}\end{figure}

\end{document}

