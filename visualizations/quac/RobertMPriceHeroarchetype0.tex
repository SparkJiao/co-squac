\documentclass[11pt,a4paper, onecolumn]{article}
\usepackage{times}
\usepackage{latexsym}
\usepackage{url}
\usepackage{textcomp}
\usepackage{bbm}
\usepackage{amsmath}
\usepackage{booktabs}
\usepackage{tabularx}
\usepackage{graphicx}
\usepackage{dialogue}
\usepackage{mathtools}
\usepackage{hyperref}
%\hypersetup{draft}

\usepackage{multirow}
\usepackage{mdframed}
\usepackage{tcolorbox}

\usepackage{xcolor,pifont}
%\newcommand{\cmark}{\ding{51}}
%\newcommand{\xmark}{\ding{55}}

\setcounter{topnumber}{2}
\setcounter{bottomnumber}{2}
\setcounter{totalnumber}{4}
\renewcommand{\topfraction}{0.75}
\renewcommand{\bottomfraction}{0.75}
\renewcommand{\textfraction}{0.05}
\renewcommand{\floatpagefraction}{0.6}

\newcommand\cmark {\textcolor{green}{\ding{52}}}
\newcommand\xmark {\textcolor{red}{\ding{55}}}
\mdfdefinestyle{dialogue}{
    backgroundcolor=yellow!20,
    innermargin=5pt
}
\usepackage{amssymb}
\usepackage{soul}
\makeatletter

\begin{document}

\hspace{15pt}{\textbf{Section}:Robert M. Price -- Hero archetype0\\}
\\ Context: He views Jesus of Nazareth as an invented figure conforming to the Rank-Raglan mythotype.  In the documentary The God Who Wasn't There, Price supports a version of the Christ myth theory, suggesting that the early Christians adopted the model for the figure of Jesus from the popular Mediterranean dying-rising saviour myths of the time, such as that of Dionysus. He argues that the comparisons were known at the time, as early church father Justin Martyr had admitted the similarities. Price suggests that Christianity simply adopted themes from the dying-rising god stories of the day and supplemented them with themes (escaping crosses, empty tombs, children being persecuted by tyrants, etc.) from the popular stories of the day in order to come up with the narratives about Christ. [Per the Kyrios Christos Cult] The ancient Mediterranean world was hip-deep in religions centering on the death and resurrection of a savior god. [...] It is very hard not to see extensive and basic similarities between these religions and the Christian religion. But somehow Christian scholars have managed not to see it, and this, one must suspect, for dogmatic reasons. [...] But it seems to me that the definitive proof that the resurrection of the Mystery Religion saviors preceded Christianity is the fact that ancient Christian apologists did not deny it! [...] A Christ religion modeled after a Mystery cult is a Mystery cult, [and against Mack's Christ cult] a Christ cult worthy of the name. This is what we expect Burton Mack to be talking about when he talks about Christ cults. Price notes that historians of classical antiquity approached mythical figures such as Heracles by rejecting supernatural tales while doggedly assuming that ''a genuine historical figure'' could be identified at the root of the legend. He describes this general approach as Euhemerism, and argues that most historical Jesus research today is also Euhemerist. Price argues that Jesus is like other ancient mythic figures, in that no mundane, secular information seems to have survived. Accordingly, Jesus also should be regarded as a mythic figure. But, Price admits to some uncertainty in this regard. He writes at the conclusion of his 2000 book Deconstructing Jesus: ''There may have been a real figure there, but there is simply no longer any way of being sure.'' CANNOTANSWER

\begin{figure}[t] \small \begin{tcolorbox}[boxsep=0pt,left=5pt,right=0pt,top=2pt,colback = yellow!5] \begin{dialogue}
 \small 
 \speak{Student}{\bf why was he a hero? }
\speak{Teacher}\colorbox{pink!25}{$\not\hookrightarrow$}
{ CANNOTANSWER }
\\
\speak{Student}{\bf Are there any other interesting aspects about this article? }
\speak{Teacher}\colorbox{pink!25}{$\hookrightarrow$}
\colorbox{red!25}{Yes,}
{ He views Jesus of Nazareth as an invented figure conforming to the Rank-Raglan mythotype. }
\\
\speak{Student}{\bf why were these his views? }
\speak{Teacher}\colorbox{pink!25}{$\not\hookrightarrow$}
{ suggesting that the early Christians adopted the model for the figure of Jesus from the popular Mediterranean dying-rising saviour myths of the time, }
\\
\speak{Student}{\bf did they adopt it? }
\speak{Teacher}\colorbox{pink!25}{$\not\hookrightarrow$}
{ CANNOTANSWER }
\\
\speak{Student}{\bf was he married? }
\speak{Teacher}\colorbox{pink!25}{$\not\hookrightarrow$}
{ CANNOTANSWER }
\\
\speak{Student}{\bf did he have a family? }
\speak{Teacher}\colorbox{pink!25}{$\not\hookrightarrow$}
{ CANNOTANSWER }
\\
 \end{dialogue}\end{tcolorbox}\end{figure}

\end{document}

