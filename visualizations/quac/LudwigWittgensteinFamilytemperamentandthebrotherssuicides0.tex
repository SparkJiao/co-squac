\documentclass[11pt,a4paper, onecolumn]{article}
\usepackage{times}
\usepackage{latexsym}
\usepackage{url}
\usepackage{textcomp}
\usepackage{bbm}
\usepackage{amsmath}
\usepackage{booktabs}
\usepackage{tabularx}
\usepackage{graphicx}
\usepackage{dialogue}
\usepackage{mathtools}
\usepackage{hyperref}
%\hypersetup{draft}

\usepackage{multirow}
\usepackage{mdframed}
\usepackage{tcolorbox}

\usepackage{xcolor,pifont}
%\newcommand{\cmark}{\ding{51}}
%\newcommand{\xmark}{\ding{55}}

\setcounter{topnumber}{2}
\setcounter{bottomnumber}{2}
\setcounter{totalnumber}{4}
\renewcommand{\topfraction}{0.75}
\renewcommand{\bottomfraction}{0.75}
\renewcommand{\textfraction}{0.05}
\renewcommand{\floatpagefraction}{0.6}

\newcommand\cmark {\textcolor{green}{\ding{52}}}
\newcommand\xmark {\textcolor{red}{\ding{55}}}
\mdfdefinestyle{dialogue}{
    backgroundcolor=yellow!20,
    innermargin=5pt
}
\usepackage{amssymb}
\usepackage{soul}
\makeatletter

\begin{document}

\hspace{15pt}{\textbf{Section}:Ludwig Wittgenstein -- Family temperament and the brothers' suicides0\\}
\\ Context: Ray Monk writes that Karl's aim was to turn his sons into captains of industry; they were not sent to school lest they acquire bad habits, but were educated at home to prepare them for work in Karl's industrial empire. Three of the five brothers would later commit suicide. Psychiatrist Michael Fitzgerald argues that Karl was a harsh perfectionist who lacked empathy, and that Wittgenstein's mother was anxious and insecure, unable to stand up to her husband. Johannes Brahms said of the family, whom he visited regularly: ''They seemed to act towards one another as if they were at court.'' The family appeared to have a strong streak of depression running through it. Anthony Gottlieb tells a story about Paul practicing on one of the pianos in the Wittgensteins' main family mansion, when he suddenly shouted at Ludwig in the next room: ''I cannot play when you are in the house, as I feel your scepticism seeping towards me from under the door!'' The family Palace housed seven grand pianos and each of the siblings pursued music ''with an enthusiasm that, at times, bordered on the pathological.'' The eldest brother, Hans, was hailed as a musical prodigy. At the age of four, writes Alexander Waugh, Hans could identify the Doppler effect in a passing siren as a quarter-tone drop in pitch, and at five started crying ''Wrong! Wrong!'' when two brass bands in a carnival played the same tune in different keys. But he died in mysterious circumstances in May 1902, when he ran away to America and disappeared from a boat in Chesapeake Bay, most likely having committed suicide. Two years later, aged 22 and studying chemistry at the Berlin Academy, the third eldest brother, Rudi, committed suicide in a Berlin bar. He had asked the pianist to play Thomas Koschat's ''Verlassen, verlassen, verlassen bin ich'' (''Forsaken, forsaken, forsaken am I''), before mixing himself a drink of milk and potassium cyanide. He had left several suicide notes, one to his parents that said he was grieving over the death of a friend, and another that referred to his ''perverted disposition''. It was reported at the time that he had sought advice from the Scientific-Humanitarian Committee, an organization that was campaigning against Paragraph 175 of the German Criminal Code, which prohibited homosexual sex. His father forbade the family from ever mentioning his name again. The second eldest brother, Kurt, an officer and company director, shot himself on 27 October 1918 at the end of World War I, when the Austrian troops he was commanding refused to obey his orders and deserted en masse. According to Gottlieb, Hermine had said Kurt seemed to carry ''...the germ of disgust for life within himself.'' Later Wittgenstein wrote: ''I ought to have... become a star in the sky. Instead of which I have remained stuck on earth.'' CANNOTANSWER

\begin{figure}[t] \small \begin{tcolorbox}[boxsep=0pt,left=5pt,right=0pt,top=2pt,colback = yellow!5] \begin{dialogue}
 \small 
 \speak{Student}{\bf Why did his brothers commit suicide? }
\speak{Teacher}\colorbox{pink!25}{$\hookrightarrow$}
{ The family appeared to have a strong streak of depression running through it. }
\\
\speak{Student}{\bf When did his brother commit suicide? }
\speak{Teacher}\colorbox{pink!25}{$\hookrightarrow$}
{ in May 1902, when he ran away to America and disappeared from a boat in Chesapeake Bay, most likely having committed suicide. }
\\
\speak{Student}{\bf When did his other brother commit suicide? }
\speak{Teacher}\colorbox{pink!25}{$\hookrightarrow$}
{ Two years later, aged 22 and studying chemistry at the Berlin Academy, the third eldest brother, Rudi, committed suicide in a Berlin bar. }
\\
\speak{Student}{\bf How did his family react? }
\speak{Teacher}\colorbox{pink!25}{ $\bar{\hookrightarrow}$}
{ His father forbade the family from ever mentioning his name again. }
\\
\speak{Student}{\bf How did their mother react? }
\speak{Teacher}\colorbox{pink!25}{ $\bar{\hookrightarrow}$}
{ CANNOTANSWER }
\\
\speak{Student}{\bf How did Ludwig deal with their deaths? }
\speak{Teacher}\colorbox{pink!25}{$\not\hookrightarrow$}
{ CANNOTANSWER }
\\
\speak{Student}{\bf What happened after their deaths? }
\speak{Teacher}\colorbox{pink!25}{$\hookrightarrow$}
{ The second eldest brother, Kurt, an officer and company director, shot himself on 27 October 1918 at the end of World War I, }
\\
\speak{Student}{\bf What happened after Kurt's death? }
\speak{Teacher}\colorbox{pink!25}{$\not\hookrightarrow$}
{ Later Wittgenstein wrote: ''I ought to have... become a star in the sky. Instead of which I have remained stuck on earth.'' }
 \end{dialogue}\end{tcolorbox}\end{figure}\begin{figure}[t] \small \begin{tcolorbox}[boxsep=0pt,left=5pt,right=0pt,top=2pt,colback = yellow!5] \begin{dialogue}
 \small 
 \speak{Student}{\bf Did their deaths influence Wittgenstein's work? }
\speak{Teacher}\colorbox{pink!25}{$\not\hookrightarrow$}
{ CANNOTANSWER }
\\
\speak{Student}{\bf Did he write anything else about his brother's deaths? }
\speak{Teacher}\colorbox{pink!25}{$\not\hookrightarrow$}
{ CANNOTANSWER }
\\
 \end{dialogue}\end{tcolorbox}\end{figure}

\end{document}

