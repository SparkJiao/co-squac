\documentclass[11pt,a4paper, onecolumn]{article}
\usepackage{times}
\usepackage{latexsym}
\usepackage{url}
\usepackage{textcomp}
\usepackage{bbm}
\usepackage{amsmath}
\usepackage{booktabs}
\usepackage{tabularx}
\usepackage{graphicx}
\usepackage{dialogue}
\usepackage{mathtools}
\usepackage{hyperref}
%\hypersetup{draft}

\usepackage{multirow}
\usepackage{mdframed}
\usepackage{tcolorbox}

\usepackage{xcolor,pifont}
%\newcommand{\cmark}{\ding{51}}
%\newcommand{\xmark}{\ding{55}}

\setcounter{topnumber}{2}
\setcounter{bottomnumber}{2}
\setcounter{totalnumber}{4}
\renewcommand{\topfraction}{0.75}
\renewcommand{\bottomfraction}{0.75}
\renewcommand{\textfraction}{0.05}
\renewcommand{\floatpagefraction}{0.6}

\newcommand\cmark {\textcolor{green}{\ding{52}}}
\newcommand\xmark {\textcolor{red}{\ding{55}}}
\mdfdefinestyle{dialogue}{
    backgroundcolor=yellow!20,
    innermargin=5pt
}
\usepackage{amssymb}
\usepackage{soul}
\makeatletter

\begin{document}

\hspace{15pt}{\textbf{Section}:Bertie Wooster -- Fictional biography0\\}
\\ Context: Bertie Wooster and his friend Bingo Little were born in the same village only a few days apart. Bertie's middle name, ''Wilberforce'', is the doing of his father, who won money on a horse named Wilberforce in the Grand National the day before Bertie's christening and insisted on his son carrying that name. The only other piece of information given about Bertie's father, aside from the fact that he had numerous relatives, is that he was a great friend of Lord Wickhammersley of Twing Hall. Bertie refers to his father as his ''guv'nor''. When he was around seven years of age, Bertie was sometimes compelled to recite ''The Charge of the Light Brigade'' for guests by his mother; she proclaimed that he recited nicely, but Bertie disagrees, and says that he and others found the experience unpleasant. Bertie also mentions reciting other poems as a child, including ''Ben Battle'' and works by poet Walter Scott. Like Jeeves, Bertie says that his mother thought him intelligent. Bertie makes no other mention of his mother, though he makes a remark about motherhood after being astounded by a friend telling a blatant lie: ''And this, mark you, a man who had had a good upbringing and had, no doubt, spent years at his mother's knee being taught to tell the truth''. When Bertie was eight years old, he took dancing lessons (alongside Corky Potter-Pirbright, sister of Bertie's friend Catsmeat Potter-Pirbright). It is established throughout the series that Bertie is an orphan who inherited a large fortune at some point, although the exact details and timing of his parents' deaths are never made clear. CANNOTANSWER

\begin{figure}[t] \small \begin{tcolorbox}[boxsep=0pt,left=5pt,right=0pt,top=2pt,colback = yellow!5] \begin{dialogue}
 \small 
 \speak{Student}{\bf What is the name of the fictional biography }
\speak{Teacher}\colorbox{pink!25}{$\not\hookrightarrow$}
{ CANNOTANSWER }
\\
\speak{Student}{\bf Was Bertie Wooster a character of a fictional biography }
\speak{Teacher}\colorbox{pink!25}{ $\bar{\hookrightarrow}$}
{ CANNOTANSWER }
\\
\speak{Student}{\bf What was the fictional biography }
\speak{Teacher}\colorbox{pink!25}{$\not\hookrightarrow$}
{ CANNOTANSWER }
\\
\speak{Student}{\bf Are there any other interesting aspects about this article? }
\speak{Teacher}\colorbox{pink!25}{$\hookrightarrow$}
\colorbox{red!25}{Yes,}
{ Bertie is an orphan who inherited a large fortune at some point, although the exact details and timing of his parents' deaths are never made clear. }
\\
 \end{dialogue}\end{tcolorbox}\end{figure}

\end{document}

