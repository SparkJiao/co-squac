\documentclass[11pt,a4paper, onecolumn]{article}
\usepackage{times}
\usepackage{latexsym}
\usepackage{url}
\usepackage{textcomp}
\usepackage{bbm}
\usepackage{amsmath}
\usepackage{booktabs}
\usepackage{tabularx}
\usepackage{graphicx}
\usepackage{dialogue}
\usepackage{mathtools}
\usepackage{hyperref}
%\hypersetup{draft}

\usepackage{multirow}
\usepackage{mdframed}
\usepackage{tcolorbox}

\usepackage{xcolor,pifont}
%\newcommand{\cmark}{\ding{51}}
%\newcommand{\xmark}{\ding{55}}

\setcounter{topnumber}{2}
\setcounter{bottomnumber}{2}
\setcounter{totalnumber}{4}
\renewcommand{\topfraction}{0.75}
\renewcommand{\bottomfraction}{0.75}
\renewcommand{\textfraction}{0.05}
\renewcommand{\floatpagefraction}{0.6}

\newcommand\cmark {\textcolor{green}{\ding{52}}}
\newcommand\xmark {\textcolor{red}{\ding{55}}}
\mdfdefinestyle{dialogue}{
    backgroundcolor=yellow!20,
    innermargin=5pt
}
\usepackage{amssymb}
\usepackage{soul}
\makeatletter

\begin{document}

\hspace{15pt}{\textbf{Section}:Seinfeld -- Themes0\\}
\\ Context: The series was often described as ''a show about nothing''. However, Seinfeld in 2014 stated ''the pitch for the show, the real pitch, when Larry and I went to NBC in 1988, was we want to show how a comedian gets his material. The show about nothing was just a joke in an episode many years later, and Larry and I to this day are surprised that it caught on as a way that people describe the show, because to us it's the opposite of that.'' Seinfeld broke several conventions of mainstream television. The show offers no growth or reconciliation to its characters. It eschews sentimentality. An episode is typically driven by humor interspersed with the superficial conflicts of characters with peculiar dispositions. Many episodes revolve around the characters' involvement in the lives of others with typically disastrous results. On the set, the notion that the characters should not develop or improve throughout the series was expressed as the ''no hugging, no learning'' rule. Also unlike most sitcoms, there are no moments of pathos; the audience is never made to feel sorry for any of the characters. Even Susan's death elicits no genuine emotions from anybody in the show. The characters are ''thirty-something singles with vague identities, no roots, and conscious indifference to morals''. Usual conventions, like isolating the characters from the actors playing them and separating the characters' world from that of the actors and audience, were broken. One such example is the story arc where the characters promote a TV sitcom series named Jerry. The show within a show, Jerry, was much like Seinfeld in that it was ''about nothing'' and Seinfeld played himself. The fictional Jerry was launched in the season four finale, but unlike Seinfeld, it wasn't picked up as a series. Jerry is one of many examples of metafiction in the show. There are no fewer than twenty-two fictional movies featured, like Rochelle, Rochelle. Because of these several elements, Seinfeld became the first TV series since Monty Python's Flying Circus to be widely described as postmodern. Jerry Seinfeld is an avid Abbott and Costello fan, and has cited the Abbott and Costello Show as an influence on Seinfeld. ''Everybody on the show knows I'm a fan. We're always joking about how we do stuff from their show. George and I will often get into a riff that has the rhythm from the old Abbott and Costello shows. And sometimes I'll hit George in the chest the way Abbott would hit Costello.'' The series includes numerous references to the team. George Costanza's middle name is ''Louis,'' after Costello. ''The Old Man'' episode featured a cantankerous character named ''Sid Fields'' as a tribute to the landlord on the team's TV show. Kramer's friend is named Mickey Abbott. A copywriter for the J. Peterman catalog is named Eddie Sherman, after the team's longtime agent. In Episode 30, Kramer hears the famous Abbott and Costello line, ''His father was a mudder. His mother was a mudder.'' CANNOTANSWER

\begin{figure}[t] \small \begin{tcolorbox}[boxsep=0pt,left=5pt,right=0pt,top=2pt,colback = yellow!5] \begin{dialogue}
 \small 
 \speak{Student}{\bf What is a theme in the show? }
\speak{Teacher}\colorbox{pink!25}{$\hookrightarrow$}
{ The series was often described as ''a show about nothing''. }
\\
\speak{Student}{\bf How did it get that description? }
\speak{Teacher}\colorbox{pink!25}{$\hookrightarrow$}
{ The show about nothing was just a joke in an episode many years later, }
\\
\speak{Student}{\bf What else is notable about the show? }
\speak{Teacher}\colorbox{pink!25}{$\hookrightarrow$}
{ '' Seinfeld broke several conventions of mainstream television. }
\\
\speak{Student}{\bf What conventions did they break? }
\speak{Teacher}\colorbox{pink!25}{$\hookrightarrow$}
{ The show offers no growth or reconciliation to its characters. It eschews sentimentality. }
\\
\speak{Student}{\bf Are there examples of this? }
\speak{Teacher}\colorbox{pink!25}{$\hookrightarrow$}
{ An episode is typically driven by humor interspersed with the superficial conflicts of characters with peculiar dispositions. }
\\
\speak{Student}{\bf anything else interesting? }
\speak{Teacher}\colorbox{pink!25}{$\hookrightarrow$}
{ Jerry Seinfeld is an avid Abbott and Costello fan, and has cited the Abbott and Costello Show as an influence on Seinfeld. }
\\
\speak{Student}{\bf What kind of influences? }
\speak{Teacher}\colorbox{pink!25}{$\hookrightarrow$}
{ George and I will often get into a riff that has the rhythm from the old Abbott and Costello shows. }
\\
\speak{Student}{\bf anything else? }
\speak{Teacher}\colorbox{pink!25}{$\not\hookrightarrow$}
{ And sometimes I'll hit George in the chest the way Abbott would hit Costello.'' }
 \end{dialogue}\end{tcolorbox}\end{figure}\begin{figure}[t] \small \begin{tcolorbox}[boxsep=0pt,left=5pt,right=0pt,top=2pt,colback = yellow!5] \begin{dialogue}
 \small 
 \speak{Student}{\bf Any other abbot and costello influences? }
\speak{Teacher}\colorbox{pink!25}{ $\bar{\hookrightarrow}$}
{ The Old Man'' episode featured a cantankerous character named ''Sid Fields'' as a tribute to the landlord on the team's TV show. Kramer's friend is named Mickey Abbott. }
\\
 \end{dialogue}\end{tcolorbox}\end{figure}

\end{document}

