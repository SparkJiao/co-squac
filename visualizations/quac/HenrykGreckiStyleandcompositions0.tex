\documentclass[11pt,a4paper, onecolumn]{article}
\usepackage{times}
\usepackage{latexsym}
\usepackage{url}
\usepackage{textcomp}
\usepackage{bbm}
\usepackage{amsmath}
\usepackage{booktabs}
\usepackage{tabularx}
\usepackage{graphicx}
\usepackage{dialogue}
\usepackage{mathtools}
\usepackage{hyperref}
%\hypersetup{draft}

\usepackage{multirow}
\usepackage{mdframed}
\usepackage{tcolorbox}

\usepackage{xcolor,pifont}
%\newcommand{\cmark}{\ding{51}}
%\newcommand{\xmark}{\ding{55}}

\setcounter{topnumber}{2}
\setcounter{bottomnumber}{2}
\setcounter{totalnumber}{4}
\renewcommand{\topfraction}{0.75}
\renewcommand{\bottomfraction}{0.75}
\renewcommand{\textfraction}{0.05}
\renewcommand{\floatpagefraction}{0.6}

\newcommand\cmark {\textcolor{green}{\ding{52}}}
\newcommand\xmark {\textcolor{red}{\ding{55}}}
\mdfdefinestyle{dialogue}{
    backgroundcolor=yellow!20,
    innermargin=5pt
}
\usepackage{amssymb}
\usepackage{soul}
\makeatletter

\begin{document}

\hspace{15pt}{\textbf{Section}:Henryk Górecki -- Style and compositions0\\}
\\ Context: Gorecki's music covers a variety of styles, but tends towards relative harmonic and rhythmical simplicity. He is considered to be a founder of the so-called New Polish School. Described by Terry Teachout, he said Gorecki has ''more conventional array of compositional techniques includes both elaborate counterpoint and the ritualistic repetition of melodic fragments and harmonic patterns.'' His first works, dating from the last half of the 1950s, were in the avant-garde style of Webern and other serialists of that time. Some of these twelve-tone and serial pieces include Epitaph (1958), First Symphony (1959), and Scontri (1960) (Mirka 2004, p. 305). At that time, Gorecki's reputation was not lagging behind that of his near-exact contemporary and his status was confirmed in 1960s when ''Monologhi'' won first prize. Even until 1962, he was firmly ensconced in the minds of the Warsaw Autumn public as a leader of the Polish Modern School, alongside Penderecki. Danuta Mirka has shown that Gorecki's compositional techniques in the 1960s were often based on geometry, including axes, figures, one- and two-dimensional patterns, and especially symmetry. Thus, she proposes the term ''geometrical period'' to refer to Gorecki's works between 1962 and 1970. Building on Krzysztof Droba's classifications, she further divides this period into two phases: (1962-63) ''the phase of sonoristic means''; and (1964-70) ''the phase of reductive constructicism'' (Mirka 2004, p. 329). During the middle 1960s and early 1970s, Gorecki progressively moved away from his early career as radical modernist, and began to compose with a more traditional, romantic mode of expression. His change of style was viewed as an affront to the then avant-garde establishment, and though he continued to receive commissions from various Polish agencies, by the mid-1970s Gorecki was no longer regarded as a composer that mattered. In the words of one critic, his ''new material was no longer cerebral and sparse; rather, it was intensely expressive, persistently rhythmic and often richly colored in the darkest of orchestral hues''. CANNOTANSWER

\begin{figure}[t] \small \begin{tcolorbox}[boxsep=0pt,left=5pt,right=0pt,top=2pt,colback = yellow!5] \begin{dialogue}
 \small 
 \speak{Student}{\bf What was his style like? }
\speak{Teacher}\colorbox{pink!25}{$\hookrightarrow$}
{ Gorecki's music covers a variety of styles, but tends towards relative harmonic and rhythmical simplicity. }
\\
\speak{Student}{\bf How about his compositions }
\speak{Teacher}\colorbox{pink!25}{$\hookrightarrow$}
{ His first works, dating from the last half of the 1950s, were in the avant-garde style of Webern and other serialists of that time. }
\\
\speak{Student}{\bf Are there any other interesting aspects about this article? }
\speak{Teacher}\colorbox{pink!25}{$\hookrightarrow$}
\colorbox{red!25}{Yes,}
{ Danuta Mirka has shown that Gorecki's compositional techniques in the 1960s }
\\
\speak{Student}{\bf What songs he made }
\speak{Teacher}\colorbox{pink!25}{$\hookrightarrow$}
{ Epitaph (1958), First Symphony (1959), and Scontri (1960) (Mirka 2004, p. 305). }
\\
\speak{Student}{\bf Did any hit the billboard }
\speak{Teacher}\colorbox{pink!25}{$\not\hookrightarrow$}
{ CANNOTANSWER }
\\
\speak{Student}{\bf What else was going on with Henryk }
\speak{Teacher}\colorbox{pink!25}{$\hookrightarrow$}
{ During the middle 1960s and early 1970s, Gorecki progressively moved away from his early career as radical modernist, }
\\
\speak{Student}{\bf Where did he move to }
\speak{Teacher}\colorbox{pink!25}{$\hookrightarrow$}
{ moved away from his early career as radical modernist, and began to compose with a more traditional, romantic mode of expression. }
\\
\speak{Student}{\bf Was he successful }
\speak{Teacher}\colorbox{pink!25}{$\hookrightarrow$}
\colorbox{red!25}{No,}
{ by the mid-1970s Gorecki was no longer regarded as a composer that mattered. }
 \end{dialogue}\end{tcolorbox}\end{figure}\begin{figure}[t] \small \begin{tcolorbox}[boxsep=0pt,left=5pt,right=0pt,top=2pt,colback = yellow!5] \begin{dialogue}
 \small 
 \speak{Student}{\bf What was he considered in the 1970s }
\speak{Teacher}\colorbox{pink!25}{$\hookrightarrow$}
{ his ''new material was no longer cerebral and sparse; rather, it was intensely expressive, persistently rhythmic and often richly colored in the darkest of orchestral hues''. }
\\
\speak{Student}{\bf How did the fans feel toward him }
\speak{Teacher}\colorbox{pink!25}{$\not\hookrightarrow$}
{ CANNOTANSWER }
\\
\speak{Student}{\bf Anything else you enjoyed reading }
\speak{Teacher}\colorbox{pink!25}{$\hookrightarrow$}
\colorbox{red!25}{Yes,}
{ Even until 1962, he was firmly ensconced in the minds of the Warsaw Autumn public as a leader of the Polish Modern School, alongside Penderecki. }
\\
 \end{dialogue}\end{tcolorbox}\end{figure}

\end{document}

