\documentclass[11pt,a4paper, onecolumn]{article}
\usepackage{times}
\usepackage{latexsym}
\usepackage{url}
\usepackage{textcomp}
\usepackage{bbm}
\usepackage{amsmath}
\usepackage{booktabs}
\usepackage{tabularx}
\usepackage{graphicx}
\usepackage{dialogue}
\usepackage{mathtools}
\usepackage{hyperref}
%\hypersetup{draft}

\usepackage{multirow}
\usepackage{mdframed}
\usepackage{tcolorbox}

\usepackage{xcolor,pifont}
%\newcommand{\cmark}{\ding{51}}
%\newcommand{\xmark}{\ding{55}}

\setcounter{topnumber}{2}
\setcounter{bottomnumber}{2}
\setcounter{totalnumber}{4}
\renewcommand{\topfraction}{0.75}
\renewcommand{\bottomfraction}{0.75}
\renewcommand{\textfraction}{0.05}
\renewcommand{\floatpagefraction}{0.6}

\newcommand\cmark {\textcolor{green}{\ding{52}}}
\newcommand\xmark {\textcolor{red}{\ding{55}}}
\mdfdefinestyle{dialogue}{
    backgroundcolor=yellow!20,
    innermargin=5pt
}
\usepackage{amssymb}
\usepackage{soul}
\makeatletter

\begin{document}

\hspace{15pt}{\textbf{Section}:Walter Winchell -- Professional career0\\}
\\ Context: Winchell was born in New York City, the son of Jennie (Bakst) and Jacob Winchell, a salesman; they were Russian Jewish immigrants. He left school in the sixth grade and started performing in Gus Edwards's vaudeville troupe known as the ''Newsboys Sextet'', which also included George Jessel. He began his career in journalism by posting notes about his acting troupe on backstage bulletin boards. He joined the Vaudeville News in 1920, then left the paper for the Evening Graphic in 1924, where his column was named Mainly About Mainstreeters. He was hired on June 10, 1929, by the New York Daily Mirror, where he finally became the author of the first syndicated gossip column, entitled On-Broadway. The column was syndicated by King Features Syndicate. He used connections in the entertainment, social, and governmental realms to expose exciting or embarrassing information about celebrities in those industries. This caused him to become very feared as a journalist because he routinely would affect the lives of famous or powerful people, exposing alleged information and rumors about them, using this as ammunition to attack his enemies and to blackmail influential people. He used this power, trading positive mention in his column (and later, his radio show) for more rumors and secrets. He made his radio debut over WABC in New York, a CBS affiliate, on May 12, 1930. The show, entitled Saks on Broadway, was a 15-minute feature that provided business news about Broadway. He switched to WJZ (later renamed WABC) and the NBC Blue (later ABC Radio) in 1932 for the Jergens Journal. CANNOTANSWER

\begin{figure}[t] \small \begin{tcolorbox}[boxsep=0pt,left=5pt,right=0pt,top=2pt,colback = yellow!5] \begin{dialogue}
 \small 
 \speak{Student}{\bf How was his professional career start? }
\speak{Teacher}\colorbox{pink!25}{$\hookrightarrow$}
{ He left school in the sixth grade and started performing in Gus Edwards's vaudeville troupe known as the ''Newsboys Sextet'', which also included George Jessel. }
\\
\speak{Student}{\bf How did he do with the troupe? }
\speak{Teacher}\colorbox{pink!25}{$\hookrightarrow$}
{ CANNOTANSWER }
\\
\speak{Student}{\bf What media outlet he joined to do this? }
\speak{Teacher}\colorbox{pink!25}{$\hookrightarrow$}
{ He joined the Vaudeville News in 1920, then left the paper for the Evening Graphic in 1924, where his column was named Mainly About Mainstreeters. }
\\
\speak{Student}{\bf What are some of the details of that column? }
\speak{Teacher}\colorbox{pink!25}{$\hookrightarrow$}
{ became the author of the first syndicated gossip column, entitled On-Broadway. The column was syndicated by King Features Syndicate. }
\\
 \end{dialogue}\end{tcolorbox}\end{figure}

\end{document}

