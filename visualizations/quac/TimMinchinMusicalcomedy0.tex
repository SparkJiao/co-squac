\documentclass[11pt,a4paper, onecolumn]{article}
\usepackage{times}
\usepackage{latexsym}
\usepackage{url}
\usepackage{textcomp}
\usepackage{bbm}
\usepackage{amsmath}
\usepackage{booktabs}
\usepackage{tabularx}
\usepackage{graphicx}
\usepackage{dialogue}
\usepackage{mathtools}
\usepackage{hyperref}
%\hypersetup{draft}

\usepackage{multirow}
\usepackage{mdframed}
\usepackage{tcolorbox}

\usepackage{xcolor,pifont}
%\newcommand{\cmark}{\ding{51}}
%\newcommand{\xmark}{\ding{55}}

\setcounter{topnumber}{2}
\setcounter{bottomnumber}{2}
\setcounter{totalnumber}{4}
\renewcommand{\topfraction}{0.75}
\renewcommand{\bottomfraction}{0.75}
\renewcommand{\textfraction}{0.05}
\renewcommand{\floatpagefraction}{0.6}

\newcommand\cmark {\textcolor{green}{\ding{52}}}
\newcommand\xmark {\textcolor{red}{\ding{55}}}
\mdfdefinestyle{dialogue}{
    backgroundcolor=yellow!20,
    innermargin=5pt
}
\usepackage{amssymb}
\usepackage{soul}
\makeatletter

\begin{document}

\hspace{15pt}{\textbf{Section}:Tim Minchin -- Musical comedy0\\}
\\ Context: Minchin describes his act as a ''funny cabaret show'' and sees himself primarily as a musician and songwriter as opposed to a comedian; he has said that his songs ''just happen to be funny.'' His reasoning for combining the disciplines of music and comedy was revealed in one interview when he said: ''I'm a good musician for a comedian and I'm a good comedian for a musician but if I had to do any of them in isolation I dunno.'' He draws on his background in theatre for his distinctive onstage appearance and persona. In his performances, he typically goes barefoot with wild hair and heavy eye makeup, which is juxtaposed with a crisp suit and tails, and a grand piano. According to Minchin, he likes going barefoot in his shows because it makes him feel more comfortable. He considers the eye makeup important because while he is playing the piano he is not able to use his arms and relies on his face for expressions and gestures; the eyeliner makes his features more distinguishable for the audience. He has said that much of his look and persona is about ''treading that line between mocking yourself and wanting to be an iconic figure. Mocking the ridiculousness and completely unrealistic dream of being an iconic figure.'' The shows consist largely of Minchin's comedic songs and poetry, with subjects including social satire, inflatable dolls, sex fetishes, and his own failed rock star ambitions. In between songs, he performs short stand-up routines. Several of his songs deal with religion, a subject with which Minchin--an atheist and a fan of Richard Dawkins--says he is ''a bit obsessed''. He argues that, as one of the most powerful and influential forces in the world, religion should never be off-limits to satirists. He says that his favourite song to perform is ''Peace Anthem for Palestine'', which reflects his feelings about religious conflict. In October 2010, he was made a Distinguished Supporter of the British Humanist Association. His comedy also deals with taboos more broadly. A prime example of this is the song ''Prejudice'', which parodies the power awarded to something as simple as a word. CANNOTANSWER

\begin{figure}[t] \small \begin{tcolorbox}[boxsep=0pt,left=5pt,right=0pt,top=2pt,colback = yellow!5] \begin{dialogue}
 \small 
 \speak{Student}{\bf when was tim born }
\speak{Teacher}\colorbox{pink!25}{$\not\hookrightarrow$}
{ CANNOTANSWER }
\\
\speak{Student}{\bf what was something tim enjoyed doing }
\speak{Teacher}\colorbox{pink!25}{ $\bar{\hookrightarrow}$}
{ I'm a good musician for a comedian and I'm a good comedian for a musician but if I had to do any of them in isolation I dunno.'' }
\\
\speak{Student}{\bf what was one of the 1st things he did in his career }
\speak{Teacher}\colorbox{pink!25}{$\hookrightarrow$}
{ CANNOTANSWER }
\\
\speak{Student}{\bf what was one of his performances }
\speak{Teacher}\colorbox{pink!25}{$\hookrightarrow$}
{ The shows consist largely of Minchin's comedic songs and poetry, with subjects including social satire, inflatable dolls, sex fetishes, and his own failed rock star ambitions. }
\\
\speak{Student}{\bf WHAT WAS THE NAME OF ONE OF HIS PERFORMANCES }
\speak{Teacher}\colorbox{pink!25}{$\hookrightarrow$}
{ He says that his favourite song to perform is ''Peace Anthem for Palestine'', which reflects his feelings about religious conflict. }
\\
 \end{dialogue}\end{tcolorbox}\end{figure}

\end{document}

