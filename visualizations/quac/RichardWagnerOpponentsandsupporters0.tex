\documentclass[11pt,a4paper, onecolumn]{article}
\usepackage{times}
\usepackage{latexsym}
\usepackage{url}
\usepackage{textcomp}
\usepackage{bbm}
\usepackage{amsmath}
\usepackage{booktabs}
\usepackage{tabularx}
\usepackage{graphicx}
\usepackage{dialogue}
\usepackage{mathtools}
\usepackage{hyperref}
%\hypersetup{draft}

\usepackage{multirow}
\usepackage{mdframed}
\usepackage{tcolorbox}

\usepackage{xcolor,pifont}
%\newcommand{\cmark}{\ding{51}}
%\newcommand{\xmark}{\ding{55}}

\setcounter{topnumber}{2}
\setcounter{bottomnumber}{2}
\setcounter{totalnumber}{4}
\renewcommand{\topfraction}{0.75}
\renewcommand{\bottomfraction}{0.75}
\renewcommand{\textfraction}{0.05}
\renewcommand{\floatpagefraction}{0.6}

\newcommand\cmark {\textcolor{green}{\ding{52}}}
\newcommand\xmark {\textcolor{red}{\ding{55}}}
\mdfdefinestyle{dialogue}{
    backgroundcolor=yellow!20,
    innermargin=5pt
}
\usepackage{amssymb}
\usepackage{soul}
\makeatletter

\begin{document}

\hspace{15pt}{\textbf{Section}:Richard Wagner -- Opponents and supporters0\\}
\\ Context: Not all reaction to Wagner was positive. For a time, German musical life divided into two factions, supporters of Wagner and supporters of Johannes Brahms; the latter, with the support of the powerful critic Eduard Hanslick (of whom Beckmesser in Meistersinger is in part a caricature) championed traditional forms and led the conservative front against Wagnerian innovations. They were supported by the conservative leanings of some German music schools, including the conservatories at Leipzig under Ignaz Moscheles and at Cologne under the direction of Ferdinand Hiller. Another Wagner detractor was the French composer Charles-Valentin Alkan, who wrote to Hiller after attending Wagner's Paris concert on 25 January 1860 at which Wagner conducted the overtures to Der fliegende Hollander and Tannhauser, the preludes to Lohengrin and Tristan und Isolde, and six other extracts from Tannhauser and Lohengrin: ''I had imagined that I was going to meet music of an innovative kind but was astonished to find a pale imitation of Berlioz ... I do not like all the music of Berlioz while appreciating his marvellous understanding of certain instrumental effects ... but here he was imitated and caricatured ... Wagner is not a musician, he is a disease.'' Even those who, like Debussy, opposed Wagner (''this old poisoner'') could not deny his influence. Indeed, Debussy was one of many composers, including Tchaikovsky, who felt the need to break with Wagner precisely because his influence was so unmistakable and overwhelming. ''Golliwogg's Cakewalk'' from Debussy's Children's Corner piano suite contains a deliberately tongue-in-cheek quotation from the opening bars of Tristan. Others who proved resistant to Wagner's operas included Gioachino Rossini, who said ''Wagner has wonderful moments, and dreadful quarters of an hour.'' In the 20th century Wagner's music was parodied by Paul Hindemith and Hanns Eisler, among others. Wagner's followers (known as Wagnerians or Wagnerites) have formed many societies dedicated to Wagner's life and work. CANNOTANSWER

\begin{figure}[t] \small \begin{tcolorbox}[boxsep=0pt,left=5pt,right=0pt,top=2pt,colback = yellow!5] \begin{dialogue}
 \small 
 \speak{Student}{\bf Who was an opponent of wagner }
\speak{Teacher}\colorbox{pink!25}{$\hookrightarrow$}
{ like Debussy, }
\\
\speak{Student}{\bf any one else }
\speak{Teacher}\colorbox{pink!25}{$\hookrightarrow$}
{ Gioachino Rossini, }
\\
\speak{Student}{\bf who supported him }
\speak{Teacher}\colorbox{pink!25}{$\hookrightarrow$}
{ was positive. For a time, German musical life divided into two factions, supporters of Wagner and supporters of Johannes Brahms; }
\\
\speak{Student}{\bf was brahms one of his competitors }
\speak{Teacher}\colorbox{pink!25}{$\hookrightarrow$}
\colorbox{red!25}{Yes,}
{ supporters of Johannes Brahms; the latter, with the support of the powerful critic Eduard Hanslick (of whom Beckmesser in }
\\
\speak{Student}{\bf what else can you tell me }
\speak{Teacher}\colorbox{pink!25}{$\hookrightarrow$}
{ Wagner's followers (known as Wagnerians or Wagnerites) have formed many societies dedicated to Wagner's life and work. }
\\
\speak{Student}{\bf what is one of the societies }
\speak{Teacher}\colorbox{pink!25}{$\not\hookrightarrow$}
{ CANNOTANSWER }
\\
\speak{Student}{\bf What kind of music did he make }
\speak{Teacher}\colorbox{pink!25}{$\not\hookrightarrow$}
{ For a time, German musical life }
\\
\speak{Student}{\bf Did he work with other people or solo }
\speak{Teacher}\colorbox{pink!25}{$\hookrightarrow$}
{ to Der fliegende Hollander and Tannhauser, the preludes to Lohengrin and Tristan und Isolde, and six other extracts from Tannhauser }
 \end{dialogue}\end{tcolorbox}\end{figure}\begin{figure}[t] \small \begin{tcolorbox}[boxsep=0pt,left=5pt,right=0pt,top=2pt,colback = yellow!5] \begin{dialogue}
 \small 
 \speak{Student}{\bf were these included in his supporters }
\speak{Teacher}\colorbox{pink!25}{$\hookrightarrow$}
\colorbox{red!25}{Yes,}
{ They were supported by the conservative leanings of some German music schools, including }
\\
\speak{Student}{\bf what was the school }
\speak{Teacher}\colorbox{pink!25}{ $\bar{\hookrightarrow}$}
{ including the conservatories at Leipzig under Ignaz Moscheles and at Cologne under the direction }
\\
\speak{Student}{\bf what countries did he work in }
\speak{Teacher}\colorbox{pink!25}{$\hookrightarrow$}
{ Paris concert }
\\
\speak{Student}{\bf were there others }
\speak{Teacher}\colorbox{pink!25}{ $\bar{\hookrightarrow}$}
{ CANNOTANSWER }
\\
 \end{dialogue}\end{tcolorbox}\end{figure}

\end{document}

