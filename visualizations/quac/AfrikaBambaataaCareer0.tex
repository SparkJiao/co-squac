\documentclass[11pt,a4paper, onecolumn]{article}
\usepackage{times}
\usepackage{latexsym}
\usepackage{url}
\usepackage{textcomp}
\usepackage{bbm}
\usepackage{amsmath}
\usepackage{booktabs}
\usepackage{tabularx}
\usepackage{graphicx}
\usepackage{dialogue}
\usepackage{mathtools}
\usepackage{hyperref}
%\hypersetup{draft}

\usepackage{multirow}
\usepackage{mdframed}
\usepackage{tcolorbox}

\usepackage{xcolor,pifont}
%\newcommand{\cmark}{\ding{51}}
%\newcommand{\xmark}{\ding{55}}

\setcounter{topnumber}{2}
\setcounter{bottomnumber}{2}
\setcounter{totalnumber}{4}
\renewcommand{\topfraction}{0.75}
\renewcommand{\bottomfraction}{0.75}
\renewcommand{\textfraction}{0.05}
\renewcommand{\floatpagefraction}{0.6}

\newcommand\cmark {\textcolor{green}{\ding{52}}}
\newcommand\xmark {\textcolor{red}{\ding{55}}}
\mdfdefinestyle{dialogue}{
    backgroundcolor=yellow!20,
    innermargin=5pt
}
\usepackage{amssymb}
\usepackage{soul}
\makeatletter

\begin{document}

\hspace{15pt}{\textbf{Section}:Afrika Bambaataa -- Career0\\}
\\ Context: Inspired by DJ Kool Herc and Kool DJ Dee, Bambaataa began hosting hip-hop parties beginning in 1976. He vowed to use hip-hop to draw angry kids out of gangs and form the Universal Zulu Nation. Robert Keith Wiggins, a.k.a. ''Cowboy'' of Grandmaster Flash and the Furious Five, is credited with naming hip-hop; the term became a common phrase used by MCs as part of a scat-inspired style of rhyming. In the documentary film Just to Get a Rep, the writer Steven Hager claims that the first time ''hip-hop'' was used in print was in his Village Voice article where he was quoting Bambaataa who had called the culture ''hip-hop'' in an interview. In 1982, Bambaataa and his followers, a group of dancers, artists, and DJs, went outside the United States on the first hip hop tour. He saw that the hip hop tours would be the key to help expand hip hop and his Universal Zulu Nation. In addition it would help promote the values of hip hop that he believed are based on peace, unity, love, and having fun. He brought peace to the gangs; many artists and gang members say that ''hip hop saved a lot of lives.'' His influence inspired many overseas artists like the French rapper MC Solaar. He was a popular DJ in The South Bronx rap scene and became known not only as Afrika Bambaataa but also as the ''Master of Records.'' He established two rap crews: the Jazzy 5 including MCs Master Ice, Mr. Freeze, Master Bee, Master D.E.E, and AJ Les, and the second crew referred to as Soulsonic Force including Mr. Biggs, Pow Wow and Emcee G.L.O.B.E. In 1982, Taylor, who was inspired by Kraftwerk's futuristic electronic music, debuted at The Roxy a test cassette of EBN-OZN's ground breaking, 12-inch white rap/spoken word ''AEIOU Sometimes Y,'' the first commercially released American single ever made on a computer, a Fairlight CMI, ushering in the era of music computer sampling. In that same year Bambaataa and Soulsonic Force dropped the live band to go high-tech. Bambaataa credited the pioneering Japanese electropop group Yellow Magic Orchestra, whose work he sampled, as an inspiration. He also borrowed an eerie keyboard hook from German electronic pioneers Kraftwerk and was provided the electronic Roland TR-808 ''beat-box'' by producer Arthur Baker and synthesizer player John Robie. That resulted in ''Planet Rock,'' which went to gold status and generated an entire school of ''electro-boogie'' rap and dance music. Bambaataa formed his own label to release the Time Zone Compilation. He created ''turntablism'' as its own subgenre and the ratification of ''electronica'' as an industry-certified trend in the late 1990s. CANNOTANSWER

\begin{figure}[t] \small \begin{tcolorbox}[boxsep=0pt,left=5pt,right=0pt,top=2pt,colback = yellow!5] \begin{dialogue}
 \small 
 \speak{Student}{\bf When did he first start singing }
\speak{Teacher}\colorbox{pink!25}{ $\bar{\hookrightarrow}$}
{ Bambaataa began hosting hip-hop parties beginning in 1976. }
\\
\speak{Student}{\bf Did he have a lot of hits? }
\speak{Teacher}\colorbox{pink!25}{ $\bar{\hookrightarrow}$}
\colorbox{red!25}{Yes,}
{ beat-box }
\\
\speak{Student}{\bf What were some of his hits }
\speak{Teacher}\colorbox{pink!25}{ $\bar{\hookrightarrow}$}
{ Planet Rock, }
\\
\speak{Student}{\bf did he do concerts }
\speak{Teacher}\colorbox{pink!25}{ $\bar{\hookrightarrow}$}
\colorbox{red!25}{Yes,}
{ In 1982, Bambaataa and his followers, a group of dancers, artists, and DJs, went outside the United States on the first hip hop tour. }
\\
\speak{Student}{\bf did they make a lot of money }
\speak{Teacher}\colorbox{pink!25}{$\not\hookrightarrow$}
{ CANNOTANSWER }
\\
\speak{Student}{\bf did they do any charity work }
\speak{Teacher}\colorbox{pink!25}{ $\bar{\hookrightarrow}$}
{ CANNOTANSWER }
\\
\speak{Student}{\bf is he still singing }
\speak{Teacher}\colorbox{pink!25}{ $\bar{\hookrightarrow}$}
{ CANNOTANSWER }
\\
\speak{Student}{\bf what is the high tech ? }
\speak{Teacher}\colorbox{pink!25}{ $\bar{\hookrightarrow}$}
{ Bambaataa credited the pioneering Japanese electropop group Yellow Magic Orchestra, whose work he sampled, as an inspiration. }
 \end{dialogue}\end{tcolorbox}\end{figure}

\end{document}

