\documentclass[11pt,a4paper, onecolumn]{article}
\usepackage{times}
\usepackage{latexsym}
\usepackage{url}
\usepackage{textcomp}
\usepackage{bbm}
\usepackage{amsmath}
\usepackage{booktabs}
\usepackage{tabularx}
\usepackage{graphicx}
\usepackage{dialogue}
\usepackage{mathtools}
\usepackage{hyperref}
%\hypersetup{draft}

\usepackage{multirow}
\usepackage{mdframed}
\usepackage{tcolorbox}

\usepackage{xcolor,pifont}
%\newcommand{\cmark}{\ding{51}}
%\newcommand{\xmark}{\ding{55}}

\setcounter{topnumber}{2}
\setcounter{bottomnumber}{2}
\setcounter{totalnumber}{4}
\renewcommand{\topfraction}{0.75}
\renewcommand{\bottomfraction}{0.75}
\renewcommand{\textfraction}{0.05}
\renewcommand{\floatpagefraction}{0.6}

\newcommand\cmark {\textcolor{green}{\ding{52}}}
\newcommand\xmark {\textcolor{red}{\ding{55}}}
\mdfdefinestyle{dialogue}{
    backgroundcolor=yellow!20,
    innermargin=5pt
}
\usepackage{amssymb}
\usepackage{soul}
\makeatletter

\begin{document}

\hspace{15pt}{\textbf{Section}:Renaissance (band) -- UK hit single0\\}
\\ Context: Although commercial success was limited during this period, Renaissance scored a hit single in Britain with ''Northern Lights'', which reached No. 10 during the summer of 1978. The single was taken from the album A Song for All Seasons (a No. 58 album in the US), and received significant airplay in the US on both AOR and on radio stations adapting to a new format known as ''soft rock'', now known as adult contemporary. The band performed on a modestly successful tour of the US east of the Mississippi and drew significant crowds in State College, Pennsylvania and Cleveland in May and June 1979, promoting both A Song For All Seasons and a mix of old and new tracks. Additionally the band was able to get additional exposure via US television; performing ''Carpet of the Sun'' in 1977 on The Midnight Special (TV series) and being guests on the May 4 1978 edition of the Mike Douglas Show, where they played Northern Lights. These clips can currently be viewed on YouTube. Renaissance floundered following 1979's Azure d'Or, as many fans could not relate to a largely synthesizer-oriented sound. As a result, the band's fan base began to lose interest and the album only reached No. 125. Dunford and Camp assumed most of the band's songwriting. In the 1970s, Renaissance defined their work with folk rock and classical fusions. Their songs include quotations from and allusions to such composers as Alain, Bach, Chopin, Debussy, Giazotto, Maurice Jarre, Rachmaninoff, Rimsky-Korsakov, Prokofiev and Shostakovich. Renaissance records, especially Ashes Are Burning, were frequently played on American progressive rock radio stations such as WNEW-FM, WHFS-FM, WMMR-FM, KSHE 95 and WVBR. CANNOTANSWER

\begin{figure}[t] \small \begin{tcolorbox}[boxsep=0pt,left=5pt,right=0pt,top=2pt,colback = yellow!5] \begin{dialogue}
 \small 
 \speak{Student}{\bf What is the name of their hit single? }
\speak{Teacher}\colorbox{pink!25}{$\hookrightarrow$}
{ ''Northern Lights'', }
\\
\speak{Student}{\bf How did the single do in the charts? }
\speak{Teacher}\colorbox{pink!25}{$\hookrightarrow$}
{ which reached No. 10 during the summer of }
\\
\speak{Student}{\bf What label were they with when it was produced? }
\speak{Teacher}\colorbox{pink!25}{$\hookrightarrow$}
{ CANNOTANSWER }
\\
\speak{Student}{\bf Are there any other interesting aspects about this article? }
\speak{Teacher}\colorbox{pink!25}{$\hookrightarrow$}
\colorbox{red!25}{Yes,}
{ the 1970s, Renaissance defined their work with folk rock and classical fusions. }
\\
\speak{Student}{\bf Did they play with anyone else? }
\speak{Teacher}\colorbox{pink!25}{$\hookrightarrow$}
{ CANNOTANSWER }
\\
\speak{Student}{\bf How many albums did they sell? }
\speak{Teacher}\colorbox{pink!25}{$\hookrightarrow$}
{ CANNOTANSWER }
\\
\speak{Student}{\bf Did they win any awards with their single? }
\speak{Teacher}\colorbox{pink!25}{$\hookrightarrow$}
{ CANNOTANSWER }
\\
\speak{Student}{\bf Did it do better in the US? }
\speak{Teacher}\colorbox{pink!25}{ $\bar{\hookrightarrow}$}
{ oriented sound. As a result, the band's fan base began to lose interest and the album only reached No. 125. Dunford }
 \end{dialogue}\end{tcolorbox}\end{figure}\begin{figure}[t] \small \begin{tcolorbox}[boxsep=0pt,left=5pt,right=0pt,top=2pt,colback = yellow!5] \begin{dialogue}
 \small 
 \speak{Student}{\bf What fact is most interesting to you in this article? }
\speak{Teacher}\colorbox{pink!25}{$\hookrightarrow$}
{ Additionally the band was able to get additional exposure via US television; performing ''Carpet of the Sun'' in 1977 on }
\\
 \end{dialogue}\end{tcolorbox}\end{figure}

\end{document}

