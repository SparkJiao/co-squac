\documentclass[11pt,a4paper, onecolumn]{article}
\usepackage{times}
\usepackage{latexsym}
\usepackage{url}
\usepackage{textcomp}
\usepackage{bbm}
\usepackage{amsmath}
\usepackage{booktabs}
\usepackage{tabularx}
\usepackage{graphicx}
\usepackage{dialogue}
\usepackage{mathtools}
\usepackage{hyperref}
%\hypersetup{draft}

\usepackage{multirow}
\usepackage{mdframed}
\usepackage{tcolorbox}

\usepackage{xcolor,pifont}
%\newcommand{\cmark}{\ding{51}}
%\newcommand{\xmark}{\ding{55}}

\setcounter{topnumber}{2}
\setcounter{bottomnumber}{2}
\setcounter{totalnumber}{4}
\renewcommand{\topfraction}{0.75}
\renewcommand{\bottomfraction}{0.75}
\renewcommand{\textfraction}{0.05}
\renewcommand{\floatpagefraction}{0.6}

\newcommand\cmark {\textcolor{green}{\ding{52}}}
\newcommand\xmark {\textcolor{red}{\ding{55}}}
\mdfdefinestyle{dialogue}{
    backgroundcolor=yellow!20,
    innermargin=5pt
}
\usepackage{amssymb}
\usepackage{soul}
\makeatletter

\begin{document}

\hspace{15pt}{\textbf{Section}:Forrest Gump -- Visual effects0\\}
\\ Context: Ken Ralston and his team at Industrial Light & Magic were responsible for the film's visual effects. Using CGI techniques, it was possible to depict Gump meeting deceased personages and shaking their hands. Hanks was first shot against a blue screen along with reference markers so that he could line up with the archive footage. To record the voices of the historical figures, voice actors were filmed and special effects were used to alter lip-syncing for the new dialogue. Archival footage was used and with the help of such techniques as chroma key, image warping, morphing, and rotoscoping, Hanks was integrated into it. In one Vietnam War scene, Gump carries Bubba away from an incoming napalm attack. To create the effect, stunt actors were initially used for compositing purposes. Then, Hanks and Williamson were filmed, with Williamson supported by a cable wire as Hanks ran with him. The explosion was then filmed, and the actors were digitally added to appear just in front of the explosions. The jet fighters and napalm canisters were also added by CGI. The CGI removal of actor Gary Sinise's legs, after his character had them amputated, was achieved by wrapping his legs with a blue fabric, which later facilitated the work of the ''roto-paint'' team to paint out his legs from every single frame. At one point, while hoisting himself into his wheelchair, his legs are used for support. The scene where Forrest spots Jenny at a peace rally at the Lincoln Memorial and Reflecting Pool in Washington, D.C., required visual effects to create the large crowd of people. Over two days of filming, approximately 1,500 extras were used. At each successive take, the extras were rearranged and moved into a different quadrant away from the camera. With the help of computers, the extras were multiplied to create a crowd of several hundred thousand people. CANNOTANSWER

\begin{figure}[t] \small \begin{tcolorbox}[boxsep=0pt,left=5pt,right=0pt,top=2pt,colback = yellow!5] \begin{dialogue}
 \small 
 \speak{Student}{\bf What type of visual effects were used in this movie? }
\speak{Teacher}\colorbox{pink!25}{$\hookrightarrow$}
{ Using CGI techniques, it was possible to depict Gump meeting deceased personages and shaking their hands. }
\\
\speak{Student}{\bf What other techniques were used? }
\speak{Teacher}\colorbox{pink!25}{$\hookrightarrow$}
{ To record the voices of the historical figures, voice actors were filmed and special effects were used to alter lip-syncing for the new dialogue. }
\\
\speak{Student}{\bf Who was in charge of the visual effects? }
\speak{Teacher}\colorbox{pink!25}{$\hookrightarrow$}
{ Ken Ralston and his team at Industrial Light & Magic were responsible for the film's visual effects. }
\\
\speak{Student}{\bf Are there any notable scenes of the movie where you can see the use of the visual effects? }
\speak{Teacher}\colorbox{pink!25}{$\hookrightarrow$}
{ In one Vietnam War scene, Gump carries Bubba away from an incoming napalm attack. To create the effect, stunt actors were initially used for compositing purposes. }
\\
\speak{Student}{\bf Are their any other scenes? }
\speak{Teacher}\colorbox{pink!25}{$\hookrightarrow$}
\colorbox{red!25}{Yes,}
{ The scene where Forrest spots Jenny at a peace rally at the Lincoln Memorial and Reflecting Pool in Washington, D.C., required visual effects to create the large crowd of people. }
\\
\speak{Student}{\bf Is there anything else interesting about the visual effects that were used? }
\speak{Teacher}\colorbox{pink!25}{$\hookrightarrow$}
\colorbox{red!25}{Yes,}
{ The CGI removal of actor Gary Sinise's legs, after his character had them amputated, was achieved by wrapping his legs with a blue fabric, }
\\
 \end{dialogue}\end{tcolorbox}\end{figure}

\end{document}

