\documentclass[11pt,a4paper, onecolumn]{article}
\usepackage{times}
\usepackage{latexsym}
\usepackage{url}
\usepackage{textcomp}
\usepackage{bbm}
\usepackage{amsmath}
\usepackage{booktabs}
\usepackage{tabularx}
\usepackage{graphicx}
\usepackage{dialogue}
\usepackage{mathtools}
\usepackage{hyperref}
%\hypersetup{draft}

\usepackage{multirow}
\usepackage{mdframed}
\usepackage{tcolorbox}

\usepackage{xcolor,pifont}
%\newcommand{\cmark}{\ding{51}}
%\newcommand{\xmark}{\ding{55}}

\setcounter{topnumber}{2}
\setcounter{bottomnumber}{2}
\setcounter{totalnumber}{4}
\renewcommand{\topfraction}{0.75}
\renewcommand{\bottomfraction}{0.75}
\renewcommand{\textfraction}{0.05}
\renewcommand{\floatpagefraction}{0.6}

\newcommand\cmark {\textcolor{green}{\ding{52}}}
\newcommand\xmark {\textcolor{red}{\ding{55}}}
\mdfdefinestyle{dialogue}{
    backgroundcolor=yellow!20,
    innermargin=5pt
}
\usepackage{amssymb}
\usepackage{soul}
\makeatletter

\begin{document}

\hspace{15pt}{\textbf{Section}:Prime number17\\}
\\ Context: The unproven Riemann hypothesis, dating from 1859, states that except for s = −2, −4, ..., all zeroes of the ζ-function have real part equal to 1/2. The connection to prime numbers is that it essentially says that the primes are as regularly distributed as possible.[clarification needed] From a physical viewpoint, it roughly states that the irregularity in the distribution of primes only comes from random noise. From a mathematical viewpoint, it roughly states that the asymptotic distribution of primes (about x/log x of numbers less than x are primes, the prime number theorem) also holds for much shorter intervals of length about the square root of x (for intervals near x). This hypothesis is generally believed to be correct. In particular, the simplest assumption is that primes should have no significant irregularities without good reason. CANNOTANSWER

\begin{figure}[t] \small \begin{tcolorbox}[boxsep=0pt,left=5pt,right=0pt,top=2pt,colback = yellow!5] \begin{dialogue}
 \small 
 \speak{Student}{\bf When was the Riemann hypothesis proposed? }
\speak{Teacher}\colorbox{pink!25}{$\hookrightarrow$}
{ 1859 }
\\
\speak{Student}{\bf According to the Riemann hypothesis, all zeroes of the ζ-function have real part equal to 1/2 except for what values of s? }
\speak{Teacher}\colorbox{pink!25}{$\hookrightarrow$}
{ s = −2, −4, ..., }
\\
\speak{Student}{\bf What does the Riemann hypothesis state the source of irregularity in the distribution of points comes from? }
\speak{Teacher}\colorbox{pink!25}{$\hookrightarrow$}
{ random noise }
\\
\speak{Student}{\bf What type of prime distribution does the Riemann hypothesis propose is also true for short intervals near X? }
\speak{Teacher}\colorbox{pink!25}{$\hookrightarrow$}
{ asymptotic distribution }
\\
\speak{Student}{\bf What type of prime distribution is characterized about x/log x of numbers less than x? }
\speak{Teacher}\colorbox{pink!25}{$\hookrightarrow$}
{ asymptotic distribution }
\\
\speak{Student}{\bf When was the function hypothesis proposed? }
\speak{Teacher}\colorbox{pink!25}{$\hookrightarrow$}
{ CANNOTANSWER }
\\
\speak{Student}{\bf According to the function hypothesis, all zeroes of the ζ-function have real part equal to 1/2 except for what values of s? }
\speak{Teacher}\colorbox{pink!25}{$\hookrightarrow$}
{ CANNOTANSWER }
\\
\speak{Student}{\bf What does the Riemann hypothesis state the source of irregularity in the distribution of math zeroes from? }
\speak{Teacher}\colorbox{pink!25}{$\hookrightarrow$}
{ CANNOTANSWER }
 \end{dialogue}\end{tcolorbox}\end{figure}\begin{figure}[t] \small \begin{tcolorbox}[boxsep=0pt,left=5pt,right=0pt,top=2pt,colback = yellow!5] \begin{dialogue}
 \small 
 \speak{Student}{\bf What type of zero distribution does the Riemann hypothesis propose is also true for short intervals near X? }
\speak{Teacher}\colorbox{pink!25}{$\hookrightarrow$}
{ CANNOTANSWER }
\\
\speak{Student}{\bf What type of zero distribution is characterized about x/log x of numbers less than x? }
\speak{Teacher}\colorbox{pink!25}{$\hookrightarrow$}
{ CANNOTANSWER }
\\
 \end{dialogue}\end{tcolorbox}\end{figure}

\end{document}

