\documentclass[11pt,a4paper, onecolumn]{article}
\usepackage{times}
\usepackage{latexsym}
\usepackage{url}
\usepackage{textcomp}
\usepackage{bbm}
\usepackage{amsmath}
\usepackage{booktabs}
\usepackage{tabularx}
\usepackage{graphicx}
\usepackage{dialogue}
\usepackage{mathtools}
\usepackage{hyperref}
%\hypersetup{draft}

\usepackage{multirow}
\usepackage{mdframed}
\usepackage{tcolorbox}

\usepackage{xcolor,pifont}
%\newcommand{\cmark}{\ding{51}}
%\newcommand{\xmark}{\ding{55}}

\setcounter{topnumber}{2}
\setcounter{bottomnumber}{2}
\setcounter{totalnumber}{4}
\renewcommand{\topfraction}{0.75}
\renewcommand{\bottomfraction}{0.75}
\renewcommand{\textfraction}{0.05}
\renewcommand{\floatpagefraction}{0.6}

\newcommand\cmark {\textcolor{green}{\ding{52}}}
\newcommand\xmark {\textcolor{red}{\ding{55}}}
\mdfdefinestyle{dialogue}{
    backgroundcolor=yellow!20,
    innermargin=5pt
}
\usepackage{amssymb}
\usepackage{soul}
\makeatletter

\begin{document}

\hspace{15pt}{\textbf{Section}:Warsaw40\\}
\\ Context: Several commemorative events take place every year. Gatherings of thousands of people on the banks of the Vistula on Midsummer’s Night for a festival called Wianki (Polish for Wreaths) have become a tradition and a yearly event in the programme of cultural events in Warsaw. The festival traces its roots to a peaceful pagan ritual where maidens would float their wreaths of herbs on the water to predict when they would be married, and to whom. By the 19th century this tradition had become a festive event, and it continues today. The city council organize concerts and other events. Each Midsummer’s Eve, apart from the official floating of wreaths, jumping over fires, looking for the fern flower, there are musical performances, dignitaries' speeches, fairs and fireworks by the river bank. CANNOTANSWER

\begin{figure}[t] \small \begin{tcolorbox}[boxsep=0pt,left=5pt,right=0pt,top=2pt,colback = yellow!5] \begin{dialogue}
 \small 
 \speak{Student}{\bf What is the polish word for wreaths? }
\speak{Teacher}\colorbox{pink!25}{$\hookrightarrow$}
{ Wianki }
\\
\speak{Student}{\bf How man people gather along the banks of the Vistula for the Wianki festival? }
\speak{Teacher}\colorbox{pink!25}{$\hookrightarrow$}
{ thousands }
\\
\speak{Student}{\bf When is the Wianki festival held? }
\speak{Teacher}\colorbox{pink!25}{$\hookrightarrow$}
{ Midsummer’s Night }
\\
\speak{Student}{\bf What will maidens be able to predict by floating their wreaths down the Vistula? }
\speak{Teacher}\colorbox{pink!25}{$\hookrightarrow$}
{ when they would be married }
\\
\speak{Student}{\bf What type of flower is sought on Midsummer's Eve? }
\speak{Teacher}\colorbox{pink!25}{$\hookrightarrow$}
{ the fern }
\\
\speak{Student}{\bf What is the polish word for concerts? }
\speak{Teacher}\colorbox{pink!25}{$\hookrightarrow$}
{ CANNOTANSWER }
\\
\speak{Student}{\bf How many people gather along the banks of the Vistula for the Wreaths festival? }
\speak{Teacher}\colorbox{pink!25}{$\hookrightarrow$}
{ CANNOTANSWER }
\\
\speak{Student}{\bf When is the Midsummer's festival held? }
\speak{Teacher}\colorbox{pink!25}{$\hookrightarrow$}
{ CANNOTANSWER }
 \end{dialogue}\end{tcolorbox}\end{figure}\begin{figure}[t] \small \begin{tcolorbox}[boxsep=0pt,left=5pt,right=0pt,top=2pt,colback = yellow!5] \begin{dialogue}
 \small 
 \speak{Student}{\bf What will maidens be able to predict by floating their programmes down the Vistula }
\speak{Teacher}\colorbox{pink!25}{$\hookrightarrow$}
{ CANNOTANSWER }
\\
\speak{Student}{\bf What type of flower is sought on Wianki? }
\speak{Teacher}\colorbox{pink!25}{$\hookrightarrow$}
{ CANNOTANSWER }
\\
 \end{dialogue}\end{tcolorbox}\end{figure}

\end{document}

