\documentclass[11pt,a4paper, onecolumn]{article}
\usepackage{times}
\usepackage{latexsym}
\usepackage{url}
\usepackage{textcomp}
\usepackage{bbm}
\usepackage{amsmath}
\usepackage{booktabs}
\usepackage{tabularx}
\usepackage{graphicx}
\usepackage{dialogue}
\usepackage{mathtools}
\usepackage{hyperref}
%\hypersetup{draft}

\usepackage{multirow}
\usepackage{mdframed}
\usepackage{tcolorbox}

\usepackage{xcolor,pifont}
%\newcommand{\cmark}{\ding{51}}
%\newcommand{\xmark}{\ding{55}}

\setcounter{topnumber}{2}
\setcounter{bottomnumber}{2}
\setcounter{totalnumber}{4}
\renewcommand{\topfraction}{0.75}
\renewcommand{\bottomfraction}{0.75}
\renewcommand{\textfraction}{0.05}
\renewcommand{\floatpagefraction}{0.6}

\newcommand\cmark {\textcolor{green}{\ding{52}}}
\newcommand\xmark {\textcolor{red}{\ding{55}}}
\mdfdefinestyle{dialogue}{
    backgroundcolor=yellow!20,
    innermargin=5pt
}
\usepackage{amssymb}
\usepackage{soul}
\makeatletter

\begin{document}

\hspace{15pt}{\textbf{Section}:Computational complexity theory38\\}
\\ Context: Many known complexity classes are suspected to be unequal, but this has not been proved. For instance P ⊆ NP ⊆ PP ⊆ PSPACE, but it is possible that P = PSPACE. If P is not equal to NP, then P is not equal to PSPACE either. Since there are many known complexity classes between P and PSPACE, such as RP, BPP, PP, BQP, MA, PH, etc., it is possible that all these complexity classes collapse to one class. Proving that any of these classes are unequal would be a major breakthrough in complexity theory. CANNOTANSWER

\begin{figure}[t] \small \begin{tcolorbox}[boxsep=0pt,left=5pt,right=0pt,top=2pt,colback = yellow!5] \begin{dialogue}
 \small 
 \speak{Student}{\bf What is the unproven assumption generally ascribed to the value of complexity classes? }
\speak{Teacher}\colorbox{pink!25}{$\hookrightarrow$}
{ suspected to be unequal }
\\
\speak{Student}{\bf What is an expression that can be used to illustrate the suspected inequality of complexity classes? }
\speak{Teacher}\colorbox{pink!25}{$\hookrightarrow$}
{ P ⊆ NP ⊆ PP ⊆ PSPACE }
\\
\speak{Student}{\bf Where can the complexity classes RP, BPP, PP, BQP, MA, and PH be located? }
\speak{Teacher}\colorbox{pink!25}{$\hookrightarrow$}
{ between P and PSPACE }
\\
\speak{Student}{\bf What evidence between and among complexity classes would signify a theoretical watershed for complexity theory? }
\speak{Teacher}\colorbox{pink!25}{$\hookrightarrow$}
{ Proving that any of these classes are unequal }
\\
\speak{Student}{\bf What is the proven assumption generally ascribed to the value of complexity classes? }
\speak{Teacher}\colorbox{pink!25}{$\hookrightarrow$}
{ CANNOTANSWER }
\\
\speak{Student}{\bf What is an expression that caan be used to illustrate the suspected in equality of complexity classes? }
\speak{Teacher}\colorbox{pink!25}{$\hookrightarrow$}
{ CANNOTANSWER }
\\
\speak{Student}{\bf Where can complexity classes RPP, BPP, PPP, BQP, MA, and PH be located? }
\speak{Teacher}\colorbox{pink!25}{$\hookrightarrow$}
{ CANNOTANSWER }
\\
\speak{Student}{\bf What is impossible for the complexity classes RP, BPP, PP, BQP, MA, and PH? }
\speak{Teacher}\colorbox{pink!25}{$\hookrightarrow$}
{ CANNOTANSWER }
 \end{dialogue}\end{tcolorbox}\end{figure}\begin{figure}[t] \small \begin{tcolorbox}[boxsep=0pt,left=5pt,right=0pt,top=2pt,colback = yellow!5] \begin{dialogue}
 \small 
 \speak{Student}{\bf What would not be a major breakthrough in complexity theory? }
\speak{Teacher}\colorbox{pink!25}{$\hookrightarrow$}
{ CANNOTANSWER }
\\
 \end{dialogue}\end{tcolorbox}\end{figure}

\end{document}

