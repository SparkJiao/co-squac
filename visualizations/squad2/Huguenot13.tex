\documentclass[11pt,a4paper, onecolumn]{article}
\usepackage{times}
\usepackage{latexsym}
\usepackage{url}
\usepackage{textcomp}
\usepackage{bbm}
\usepackage{amsmath}
\usepackage{booktabs}
\usepackage{tabularx}
\usepackage{graphicx}
\usepackage{dialogue}
\usepackage{mathtools}
\usepackage{hyperref}
%\hypersetup{draft}

\usepackage{multirow}
\usepackage{mdframed}
\usepackage{tcolorbox}

\usepackage{xcolor,pifont}
%\newcommand{\cmark}{\ding{51}}
%\newcommand{\xmark}{\ding{55}}

\setcounter{topnumber}{2}
\setcounter{bottomnumber}{2}
\setcounter{totalnumber}{4}
\renewcommand{\topfraction}{0.75}
\renewcommand{\bottomfraction}{0.75}
\renewcommand{\textfraction}{0.05}
\renewcommand{\floatpagefraction}{0.6}

\newcommand\cmark {\textcolor{green}{\ding{52}}}
\newcommand\xmark {\textcolor{red}{\ding{55}}}
\mdfdefinestyle{dialogue}{
    backgroundcolor=yellow!20,
    innermargin=5pt
}
\usepackage{amssymb}
\usepackage{soul}
\makeatletter

\begin{document}

\hspace{15pt}{\textbf{Section}:Huguenot13\\}
\\ Context: After the revocation of the Edict of Nantes, the Dutch Republic received the largest group of Huguenot refugees, an estimated total of 75,000 to 100,000 people. Amongst them were 200 clergy. Many came from the region of the Cévennes, for instance, the village of Fraissinet-de-Lozère. This was a huge influx as the entire population of the Dutch Republic amounted to ca. 2 million at that time. Around 1700, it is estimated that nearly 25  of the Amsterdam population was Huguenot.[citation needed] In 1705, Amsterdam and the area of West Frisia were the first areas to provide full citizens rights to Huguenot immigrants, followed by the Dutch Republic in 1715. Huguenots intermarried with Dutch from the outset. CANNOTANSWER

\begin{figure}[t] \small \begin{tcolorbox}[boxsep=0pt,left=5pt,right=0pt,top=2pt,colback = yellow!5] \begin{dialogue}
 \small 
 \speak{Student}{\bf What country initially received the largest number of Huguenot refugees? }
\speak{Teacher}\colorbox{pink!25}{$\hookrightarrow$}
{ the Dutch Republic }
\\
\speak{Student}{\bf How many refugees emigrated to the Dutch Republic? }
\speak{Teacher}\colorbox{pink!25}{$\hookrightarrow$}
{ an estimated total of 75,000 to 100,000 people }
\\
\speak{Student}{\bf What was the population of the Dutch Republic before this emigration? }
\speak{Teacher}\colorbox{pink!25}{$\hookrightarrow$}
{ ca. 2 million }
\\
\speak{Student}{\bf What two areas in the Republic were first to grant rights to the Huguenots? }
\speak{Teacher}\colorbox{pink!25}{$\hookrightarrow$}
{ Amsterdam and the area of West Frisia }
\\
\speak{Student}{\bf What declaration predicated the emigration of Huguenot refugees? }
\speak{Teacher}\colorbox{pink!25}{$\hookrightarrow$}
{ the revocation of the Edict of Nantes }
\\
\speak{Student}{\bf How many Huguenots lived in West Frisia in 1705? }
\speak{Teacher}\colorbox{pink!25}{$\hookrightarrow$}
{ CANNOTANSWER }
\\
\speak{Student}{\bf How many Huguenots lived in Amsterdam in 1705? }
\speak{Teacher}\colorbox{pink!25}{$\hookrightarrow$}
{ CANNOTANSWER }
\\
\speak{Student}{\bf In what year was the Edict of Nantes revoked? }
\speak{Teacher}\colorbox{pink!25}{$\hookrightarrow$}
{ CANNOTANSWER }
 \end{dialogue}\end{tcolorbox}\end{figure}\begin{figure}[t] \small \begin{tcolorbox}[boxsep=0pt,left=5pt,right=0pt,top=2pt,colback = yellow!5] \begin{dialogue}
 \small 
 \speak{Student}{\bf How many clergymen were there in the Dutch Republic before the influx of Huguenots? }
\speak{Teacher}\colorbox{pink!25}{$\hookrightarrow$}
{ CANNOTANSWER }
\\
\speak{Student}{\bf In what country is the Cevennes? }
\speak{Teacher}\colorbox{pink!25}{$\hookrightarrow$}
{ CANNOTANSWER }
\\
 \end{dialogue}\end{tcolorbox}\end{figure}

\end{document}

