\documentclass[11pt,a4paper, onecolumn]{article}
\usepackage{times}
\usepackage{latexsym}
\usepackage{url}
\usepackage{textcomp}
\usepackage{bbm}
\usepackage{amsmath}
\usepackage{booktabs}
\usepackage{tabularx}
\usepackage{graphicx}
\usepackage{dialogue}
\usepackage{mathtools}
\usepackage{hyperref}
%\hypersetup{draft}

\usepackage{multirow}
\usepackage{mdframed}
\usepackage{tcolorbox}

\usepackage{xcolor,pifont}
%\newcommand{\cmark}{\ding{51}}
%\newcommand{\xmark}{\ding{55}}

\setcounter{topnumber}{2}
\setcounter{bottomnumber}{2}
\setcounter{totalnumber}{4}
\renewcommand{\topfraction}{0.75}
\renewcommand{\bottomfraction}{0.75}
\renewcommand{\textfraction}{0.05}
\renewcommand{\floatpagefraction}{0.6}

\newcommand\cmark {\textcolor{green}{\ding{52}}}
\newcommand\xmark {\textcolor{red}{\ding{55}}}
\mdfdefinestyle{dialogue}{
    backgroundcolor=yellow!20,
    innermargin=5pt
}
\usepackage{amssymb}
\usepackage{soul}
\makeatletter

\begin{document}

\hspace{15pt}{\textbf{Section}:Packet switching5\\}
\\ Context: In connectionless mode each packet includes complete addressing information. The packets are routed individually, sometimes resulting in different paths and out-of-order delivery. Each packet is labeled with a destination address, source address, and port numbers. It may also be labeled with the sequence number of the packet. This precludes the need for a dedicated path to help the packet find its way to its destination, but means that much more information is needed in the packet header, which is therefore larger, and this information needs to be looked up in power-hungry content-addressable memory. Each packet is dispatched and may go via different routes; potentially, the system has to do as much work for every packet as the connection-oriented system has to do in connection set-up, but with less information as to the application's requirements. At the destination, the original message/data is reassembled in the correct order, based on the packet sequence number. Thus a virtual connection, also known as a virtual circuit or byte stream is provided to the end-user by a transport layer protocol, although intermediate network nodes only provides a connectionless network layer service. CANNOTANSWER

\begin{figure}[t] \small \begin{tcolorbox}[boxsep=0pt,left=5pt,right=0pt,top=2pt,colback = yellow!5] \begin{dialogue}
 \small 
 \speak{Student}{\bf What does each packet includ in connectionless mode  }
\speak{Teacher}\colorbox{pink!25}{$\hookrightarrow$}
{ each packet includes complete addressing information }
\\
\speak{Student}{\bf How are the packets routed  }
\speak{Teacher}\colorbox{pink!25}{$\hookrightarrow$}
{ individually, sometimes resulting in different paths and out-of-order delivery }
\\
\speak{Student}{\bf What is included with each packet label }
\speak{Teacher}\colorbox{pink!25}{$\hookrightarrow$}
{ Each packet is labeled with a destination address, source address, and port numbers. It may also be labeled with the sequence number of the packet }
\\
\speak{Student}{\bf What happens to the packet at the destination }
\speak{Teacher}\colorbox{pink!25}{$\hookrightarrow$}
{ the original message/data is reassembled in the correct order, based on the packet sequence number }
\\
\speak{Student}{\bf What results in out of order delivery? }
\speak{Teacher}\colorbox{pink!25}{$\hookrightarrow$}
{ CANNOTANSWER }
\\
\speak{Student}{\bf While packets are labeled correctly what can happen to them? }
\speak{Teacher}\colorbox{pink!25}{$\hookrightarrow$}
{ CANNOTANSWER }
\\
\speak{Student}{\bf Why do packets arrive out of order? }
\speak{Teacher}\colorbox{pink!25}{$\hookrightarrow$}
{ CANNOTANSWER }
\\
\speak{Student}{\bf Where is the data reassembled?  }
\speak{Teacher}\colorbox{pink!25}{$\hookrightarrow$}
{ CANNOTANSWER }
 \end{dialogue}\end{tcolorbox}\end{figure}\begin{figure}[t] \small \begin{tcolorbox}[boxsep=0pt,left=5pt,right=0pt,top=2pt,colback = yellow!5] \begin{dialogue}
 \small 
 \speak{Student}{\bf What is a virtual connection? }
\speak{Teacher}\colorbox{pink!25}{$\hookrightarrow$}
{ CANNOTANSWER }
\\
\speak{Student}{\bf Can a packet be sent incomplete? }
\speak{Teacher}\colorbox{pink!25}{$\hookrightarrow$}
{ CANNOTANSWER }
\\
\speak{Student}{\bf If three sequential packets are sent and the one in the middle is lost, then how is the data recompiled in a meaningful way? }
\speak{Teacher}\colorbox{pink!25}{$\hookrightarrow$}
{ CANNOTANSWER }
\\
\speak{Student}{\bf Can a packet ever be sent to the wrong number? }
\speak{Teacher}\colorbox{pink!25}{$\hookrightarrow$}
{ CANNOTANSWER }
\\
\speak{Student}{\bf If the packets travel via different routes, then how do they arrive in order? }
\speak{Teacher}\colorbox{pink!25}{$\hookrightarrow$}
{ CANNOTANSWER }
\\
\speak{Student}{\bf What is each message labeled with? }
\speak{Teacher}\colorbox{pink!25}{$\hookrightarrow$}
{ CANNOTANSWER }
\\
\speak{Student}{\bf How are the messages routed? }
\speak{Teacher}\colorbox{pink!25}{$\hookrightarrow$}
{ CANNOTANSWER }
\\
\speak{Student}{\bf Why is there a need for a dedicated path? }
\speak{Teacher}\colorbox{pink!25}{$\hookrightarrow$}
{ CANNOTANSWER }
 \end{dialogue}\end{tcolorbox}\end{figure}\begin{figure}[t] \small \begin{tcolorbox}[boxsep=0pt,left=5pt,right=0pt,top=2pt,colback = yellow!5] \begin{dialogue}
 \small 
 \speak{Student}{\bf What is included in the data in connectionless mode? }
\speak{Teacher}\colorbox{pink!25}{$\hookrightarrow$}
{ CANNOTANSWER }
\\
 \end{dialogue}\end{tcolorbox}\end{figure}

\end{document}

