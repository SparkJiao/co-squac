\documentclass[11pt,a4paper, onecolumn]{article}
\usepackage{times}
\usepackage{latexsym}
\usepackage{url}
\usepackage{textcomp}
\usepackage{bbm}
\usepackage{amsmath}
\usepackage{booktabs}
\usepackage{tabularx}
\usepackage{graphicx}
\usepackage{dialogue}
\usepackage{mathtools}
\usepackage{hyperref}
%\hypersetup{draft}

\usepackage{multirow}
\usepackage{mdframed}
\usepackage{tcolorbox}

\usepackage{xcolor,pifont}
%\newcommand{\cmark}{\ding{51}}
%\newcommand{\xmark}{\ding{55}}

\setcounter{topnumber}{2}
\setcounter{bottomnumber}{2}
\setcounter{totalnumber}{4}
\renewcommand{\topfraction}{0.75}
\renewcommand{\bottomfraction}{0.75}
\renewcommand{\textfraction}{0.05}
\renewcommand{\floatpagefraction}{0.6}

\newcommand\cmark {\textcolor{green}{\ding{52}}}
\newcommand\xmark {\textcolor{red}{\ding{55}}}
\mdfdefinestyle{dialogue}{
    backgroundcolor=yellow!20,
    innermargin=5pt
}
\usepackage{amssymb}
\usepackage{soul}
\makeatletter

\begin{document}

\hspace{15pt}{\textbf{Section}:Ctenophora10\\}
\\ Context: The outer surface bears usually eight comb rows, called swimming-plates, which are used for swimming. The rows are oriented to run from near the mouth (the ''oral pole'') to the opposite end (the ''aboral pole''), and are spaced more or less evenly around the body, although spacing patterns vary by species and in most species the comb rows extend only part of the distance from the aboral pole towards the mouth. The ''combs'' (also called ''ctenes'' or ''comb plates'') run across each row, and each consists of thousands of unusually long cilia, up to 2 millimeters (0.079 in). Unlike conventional cilia and flagella, which has a filament structure arranged in a 9 + 2 pattern, these cilia are arranged in a 9 + 3 pattern, where the extra compact filament is suspected to have a supporting function. These normally beat so that the propulsion stroke is away from the mouth, although they can also reverse direction. Hence ctenophores usually swim in the direction in which the mouth is pointing, unlike jellyfish. When trying to escape predators, one species can accelerate to six times its normal speed; some other species reverse direction as part of their escape behavior, by reversing the power stroke of the comb plate cilia. CANNOTANSWER

\begin{figure}[t] \small \begin{tcolorbox}[boxsep=0pt,left=5pt,right=0pt,top=2pt,colback = yellow!5] \begin{dialogue}
 \small 
 \speak{Student}{\bf What are the eight comb rows on the outer surface called? }
\speak{Teacher}\colorbox{pink!25}{$\hookrightarrow$}
{ swimming-plates }
\\
\speak{Student}{\bf Combs are called what? }
\speak{Teacher}\colorbox{pink!25}{$\hookrightarrow$}
{ also called ''ctenes'' or ''comb plates }
\\
\speak{Student}{\bf What does the 9 +3 pattern of cilia thought to do? }
\speak{Teacher}\colorbox{pink!25}{$\hookrightarrow$}
{ supporting function }
\\
\speak{Student}{\bf What direction do ctenophore swim? }
\speak{Teacher}\colorbox{pink!25}{$\hookrightarrow$}
{ in the direction in which the mouth is pointing, }
\\
\speak{Student}{\bf Cilia can g ow up too what length? }
\speak{Teacher}\colorbox{pink!25}{$\hookrightarrow$}
{ 2 millimeters (0.079 in) }
\\
\speak{Student}{\bf What are flagella also called? }
\speak{Teacher}\colorbox{pink!25}{$\hookrightarrow$}
{ CANNOTANSWER }
\\
\speak{Student}{\bf How many types of flagella are there? }
\speak{Teacher}\colorbox{pink!25}{$\hookrightarrow$}
{ CANNOTANSWER }
\\
\speak{Student}{\bf To what length can flagella grow to? }
\speak{Teacher}\colorbox{pink!25}{$\hookrightarrow$}
{ CANNOTANSWER }
 \end{dialogue}\end{tcolorbox}\end{figure}\begin{figure}[t] \small \begin{tcolorbox}[boxsep=0pt,left=5pt,right=0pt,top=2pt,colback = yellow!5] \begin{dialogue}
 \small 
 \speak{Student}{\bf What direction do cilia usually swim? }
\speak{Teacher}\colorbox{pink!25}{$\hookrightarrow$}
{ CANNOTANSWER }
\\
\speak{Student}{\bf What kind of behavior are cilia participating in when they reverse direction with their flagella? }
\speak{Teacher}\colorbox{pink!25}{$\hookrightarrow$}
{ CANNOTANSWER }
\\
 \end{dialogue}\end{tcolorbox}\end{figure}

\end{document}

