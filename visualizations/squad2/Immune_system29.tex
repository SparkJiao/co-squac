\documentclass[11pt,a4paper, onecolumn]{article}
\usepackage{times}
\usepackage{latexsym}
\usepackage{url}
\usepackage{textcomp}
\usepackage{bbm}
\usepackage{amsmath}
\usepackage{booktabs}
\usepackage{tabularx}
\usepackage{graphicx}
\usepackage{dialogue}
\usepackage{mathtools}
\usepackage{hyperref}
%\hypersetup{draft}

\usepackage{multirow}
\usepackage{mdframed}
\usepackage{tcolorbox}

\usepackage{xcolor,pifont}
%\newcommand{\cmark}{\ding{51}}
%\newcommand{\xmark}{\ding{55}}

\setcounter{topnumber}{2}
\setcounter{bottomnumber}{2}
\setcounter{totalnumber}{4}
\renewcommand{\topfraction}{0.75}
\renewcommand{\bottomfraction}{0.75}
\renewcommand{\textfraction}{0.05}
\renewcommand{\floatpagefraction}{0.6}

\newcommand\cmark {\textcolor{green}{\ding{52}}}
\newcommand\xmark {\textcolor{red}{\ding{55}}}
\mdfdefinestyle{dialogue}{
    backgroundcolor=yellow!20,
    innermargin=5pt
}
\usepackage{amssymb}
\usepackage{soul}
\makeatletter

\begin{document}

\hspace{15pt}{\textbf{Section}:Immune system29\\}
\\ Context: Immunodeficiencies occur when one or more of the components of the immune system are inactive. The ability of the immune system to respond to pathogens is diminished in both the young and the elderly, with immune responses beginning to decline at around 50 years of age due to immunosenescence. In developed countries, obesity, alcoholism, and drug use are common causes of poor immune function. However, malnutrition is the most common cause of immunodeficiency in developing countries. Diets lacking sufficient protein are associated with impaired cell-mediated immunity, complement activity, phagocyte function, IgA antibody concentrations, and cytokine production. Additionally, the loss of the thymus at an early age through genetic mutation or surgical removal results in severe immunodeficiency and a high susceptibility to infection. CANNOTANSWER

\begin{figure}[t] \small \begin{tcolorbox}[boxsep=0pt,left=5pt,right=0pt,top=2pt,colback = yellow!5] \begin{dialogue}
 \small 
 \speak{Student}{\bf What kind of disorders occur when part of the immune system isn't active? }
\speak{Teacher}\colorbox{pink!25}{$\hookrightarrow$}
{ Immunodeficiencies }
\\
\speak{Student}{\bf In what two age groups is the strength of the immune system reduced? }
\speak{Teacher}\colorbox{pink!25}{$\hookrightarrow$}
{ the young and the elderly }
\\
\speak{Student}{\bf At what age do immune responses typically begin to decline? }
\speak{Teacher}\colorbox{pink!25}{$\hookrightarrow$}
{ around 50 years of age }
\\
\speak{Student}{\bf What are some causes of reduced immune function in developed countries? }
\speak{Teacher}\colorbox{pink!25}{$\hookrightarrow$}
{ obesity, alcoholism, and drug use }
\\
\speak{Student}{\bf What is the most common cause of immunodeficiency in developing nations? }
\speak{Teacher}\colorbox{pink!25}{$\hookrightarrow$}
{ malnutrition }
\\
\speak{Student}{\bf What occurs when all components of the immune system are active? }
\speak{Teacher}\colorbox{pink!25}{$\hookrightarrow$}
{ CANNOTANSWER }
\\
\speak{Student}{\bf In what people is the immune system the strongest? }
\speak{Teacher}\colorbox{pink!25}{$\hookrightarrow$}
{ CANNOTANSWER }
\\
\speak{Student}{\bf What is the rarest cause of poor immune function in developing countries? }
\speak{Teacher}\colorbox{pink!25}{$\hookrightarrow$}
{ CANNOTANSWER }
 \end{dialogue}\end{tcolorbox}\end{figure}\begin{figure}[t] \small \begin{tcolorbox}[boxsep=0pt,left=5pt,right=0pt,top=2pt,colback = yellow!5] \begin{dialogue}
 \small 
 \speak{Student}{\bf What do diets with too much protein cause? }
\speak{Teacher}\colorbox{pink!25}{$\hookrightarrow$}
{ CANNOTANSWER }
\\
\speak{Student}{\bf What does the loss of the thymus at an early age prevent? }
\speak{Teacher}\colorbox{pink!25}{$\hookrightarrow$}
{ CANNOTANSWER }
\\
 \end{dialogue}\end{tcolorbox}\end{figure}

\end{document}

