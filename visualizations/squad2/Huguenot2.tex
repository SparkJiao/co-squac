\documentclass[11pt,a4paper, onecolumn]{article}
\usepackage{times}
\usepackage{latexsym}
\usepackage{url}
\usepackage{textcomp}
\usepackage{bbm}
\usepackage{amsmath}
\usepackage{booktabs}
\usepackage{tabularx}
\usepackage{graphicx}
\usepackage{dialogue}
\usepackage{mathtools}
\usepackage{hyperref}
%\hypersetup{draft}

\usepackage{multirow}
\usepackage{mdframed}
\usepackage{tcolorbox}

\usepackage{xcolor,pifont}
%\newcommand{\cmark}{\ding{51}}
%\newcommand{\xmark}{\ding{55}}

\setcounter{topnumber}{2}
\setcounter{bottomnumber}{2}
\setcounter{totalnumber}{4}
\renewcommand{\topfraction}{0.75}
\renewcommand{\bottomfraction}{0.75}
\renewcommand{\textfraction}{0.05}
\renewcommand{\floatpagefraction}{0.6}

\newcommand\cmark {\textcolor{green}{\ding{52}}}
\newcommand\xmark {\textcolor{red}{\ding{55}}}
\mdfdefinestyle{dialogue}{
    backgroundcolor=yellow!20,
    innermargin=5pt
}
\usepackage{amssymb}
\usepackage{soul}
\makeatletter

\begin{document}

\hspace{15pt}{\textbf{Section}:Huguenot2\\}
\\ Context: The availability of the Bible in vernacular languages was important to the spread of the Protestant movement and development of the Reformed church in France. The country had a long history of struggles with the papacy by the time the Protestant Reformation finally arrived. Around 1294, a French version of the Scriptures was prepared by the Roman Catholic priest, Guyard de Moulin. A two-volume illustrated folio paraphrase version based on his manuscript, by Jean de Rély, was printed in Paris in 1487. CANNOTANSWER

\begin{figure}[t] \small \begin{tcolorbox}[boxsep=0pt,left=5pt,right=0pt,top=2pt,colback = yellow!5] \begin{dialogue}
 \small 
 \speak{Student}{\bf What helped spread Protestantism in France? }
\speak{Teacher}\colorbox{pink!25}{$\hookrightarrow$}
{ availability of the Bible in vernacular languages }
\\
\speak{Student}{\bf When did the first French language bible appear? }
\speak{Teacher}\colorbox{pink!25}{$\hookrightarrow$}
{ Around 1294 }
\\
\speak{Student}{\bf Who translated this version of the scriptures? }
\speak{Teacher}\colorbox{pink!25}{$\hookrightarrow$}
{ Guyard de Moulin }
\\
\speak{Student}{\bf An illustrated, paraphrased version of this appeared when? }
\speak{Teacher}\colorbox{pink!25}{$\hookrightarrow$}
{ 1487 }
\\
\speak{Student}{\bf Jean De Rely's illustrated French-language scriptures were first published in what city? }
\speak{Teacher}\colorbox{pink!25}{$\hookrightarrow$}
{ Paris }
\\
\speak{Student}{\bf In what year did the Protestant Reformation arrive in France? }
\speak{Teacher}\colorbox{pink!25}{$\hookrightarrow$}
{ CANNOTANSWER }
\\
\speak{Student}{\bf In what country did the Protestant Reformation get its start? }
\speak{Teacher}\colorbox{pink!25}{$\hookrightarrow$}
{ CANNOTANSWER }
\\
\speak{Student}{\bf In what year did the Reformed church of France get established? }
\speak{Teacher}\colorbox{pink!25}{$\hookrightarrow$}
{ CANNOTANSWER }
 \end{dialogue}\end{tcolorbox}\end{figure}\begin{figure}[t] \small \begin{tcolorbox}[boxsep=0pt,left=5pt,right=0pt,top=2pt,colback = yellow!5] \begin{dialogue}
 \small 
 \speak{Student}{\bf Where was the Roman Catholic Priest Guyard de Moulin from? }
\speak{Teacher}\colorbox{pink!25}{$\hookrightarrow$}
{ CANNOTANSWER }
\\
\speak{Student}{\bf Where was Jean de Rely from? }
\speak{Teacher}\colorbox{pink!25}{$\hookrightarrow$}
{ CANNOTANSWER }
\\
 \end{dialogue}\end{tcolorbox}\end{figure}

\end{document}

