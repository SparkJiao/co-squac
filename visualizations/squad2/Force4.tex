\documentclass[11pt,a4paper, onecolumn]{article}
\usepackage{times}
\usepackage{latexsym}
\usepackage{url}
\usepackage{textcomp}
\usepackage{bbm}
\usepackage{amsmath}
\usepackage{booktabs}
\usepackage{tabularx}
\usepackage{graphicx}
\usepackage{dialogue}
\usepackage{mathtools}
\usepackage{hyperref}
%\hypersetup{draft}

\usepackage{multirow}
\usepackage{mdframed}
\usepackage{tcolorbox}

\usepackage{xcolor,pifont}
%\newcommand{\cmark}{\ding{51}}
%\newcommand{\xmark}{\ding{55}}

\setcounter{topnumber}{2}
\setcounter{bottomnumber}{2}
\setcounter{totalnumber}{4}
\renewcommand{\topfraction}{0.75}
\renewcommand{\bottomfraction}{0.75}
\renewcommand{\textfraction}{0.05}
\renewcommand{\floatpagefraction}{0.6}

\newcommand\cmark {\textcolor{green}{\ding{52}}}
\newcommand\xmark {\textcolor{red}{\ding{55}}}
\mdfdefinestyle{dialogue}{
    backgroundcolor=yellow!20,
    innermargin=5pt
}
\usepackage{amssymb}
\usepackage{soul}
\makeatletter

\begin{document}

\hspace{15pt}{\textbf{Section}:Force4\\}
\\ Context: Newton's First Law of Motion states that objects continue to move in a state of constant velocity unless acted upon by an external net force or resultant force. This law is an extension of Galileo's insight that constant velocity was associated with a lack of net force (see a more detailed description of this below). Newton proposed that every object with mass has an innate inertia that functions as the fundamental equilibrium ''natural state'' in place of the Aristotelian idea of the ''natural state of rest''. That is, the first law contradicts the intuitive Aristotelian belief that a net force is required to keep an object moving with constant velocity. By making rest physically indistinguishable from non-zero constant velocity, Newton's First Law directly connects inertia with the concept of relative velocities. Specifically, in systems where objects are moving with different velocities, it is impossible to determine which object is ''in motion'' and which object is ''at rest''. In other words, to phrase matters more technically, the laws of physics are the same in every inertial frame of reference, that is, in all frames related by a Galilean transformation. CANNOTANSWER

\begin{figure}[t] \small \begin{tcolorbox}[boxsep=0pt,left=5pt,right=0pt,top=2pt,colback = yellow!5] \begin{dialogue}
 \small 
 \speak{Student}{\bf Whose First Law of Motion says that unless acted upon be forces, objects would continue to move at a constant velocity? }
\speak{Teacher}\colorbox{pink!25}{$\hookrightarrow$}
{ Newton }
\\
\speak{Student}{\bf What insight of Galileo was associated with constant velocity? }
\speak{Teacher}\colorbox{pink!25}{$\hookrightarrow$}
{ lack of net force }
\\
\speak{Student}{\bf Who proposed that innate intertial is the natural state of objects? }
\speak{Teacher}\colorbox{pink!25}{$\hookrightarrow$}
{ Newton }
\\
\speak{Student}{\bf What law connects relative velocities with inertia? }
\speak{Teacher}\colorbox{pink!25}{$\hookrightarrow$}
{ Newton's First }
\\
\speak{Student}{\bf What are the laws of physics of Galileo, in reference to objest in motion and rest? }
\speak{Teacher}\colorbox{pink!25}{$\hookrightarrow$}
{ the same }
\\
\speak{Student}{\bf Newton's Second Law of Motion states what? }
\speak{Teacher}\colorbox{pink!25}{$\hookrightarrow$}
{ CANNOTANSWER }
\\
\speak{Student}{\bf The second law contradicts what belief? }
\speak{Teacher}\colorbox{pink!25}{$\hookrightarrow$}
{ CANNOTANSWER }
\\
\speak{Student}{\bf Whose law made rest physically indistinguishable from zero constant velocity? }
\speak{Teacher}\colorbox{pink!25}{$\hookrightarrow$}
{ CANNOTANSWER }
 \end{dialogue}\end{tcolorbox}\end{figure}\begin{figure}[t] \small \begin{tcolorbox}[boxsep=0pt,left=5pt,right=0pt,top=2pt,colback = yellow!5] \begin{dialogue}
 \small 
 \speak{Student}{\bf What laws are different in every inertial frame of reference? }
\speak{Teacher}\colorbox{pink!25}{$\hookrightarrow$}
{ CANNOTANSWER }
\\
 \end{dialogue}\end{tcolorbox}\end{figure}

\end{document}

