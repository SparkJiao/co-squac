\documentclass[11pt,a4paper, onecolumn]{article}
\usepackage{times}
\usepackage{latexsym}
\usepackage{url}
\usepackage{textcomp}
\usepackage{bbm}
\usepackage{amsmath}
\usepackage{booktabs}
\usepackage{tabularx}
\usepackage{graphicx}
\usepackage{dialogue}
\usepackage{mathtools}
\usepackage{hyperref}
%\hypersetup{draft}

\usepackage{multirow}
\usepackage{mdframed}
\usepackage{tcolorbox}

\usepackage{xcolor,pifont}
%\newcommand{\cmark}{\ding{51}}
%\newcommand{\xmark}{\ding{55}}

\setcounter{topnumber}{2}
\setcounter{bottomnumber}{2}
\setcounter{totalnumber}{4}
\renewcommand{\topfraction}{0.75}
\renewcommand{\bottomfraction}{0.75}
\renewcommand{\textfraction}{0.05}
\renewcommand{\floatpagefraction}{0.6}

\newcommand\cmark {\textcolor{green}{\ding{52}}}
\newcommand\xmark {\textcolor{red}{\ding{55}}}
\mdfdefinestyle{dialogue}{
    backgroundcolor=yellow!20,
    innermargin=5pt
}
\usepackage{amssymb}
\usepackage{soul}
\makeatletter

\begin{document}

\hspace{15pt}{\textbf{Section}:Imperialism13\\}
\\ Context: Cultural imperialism is when a country's influence is felt in social and cultural circles, i.e. its soft power, such that it changes the moral, cultural and societal worldview of another. This is more than just ''foreign'' music, television or film becoming popular with young people, but that popular culture changing their own expectations of life and their desire for their own country to become more like the foreign country depicted. For example, depictions of opulent American lifestyles in the soap opera Dallas during the Cold War changed the expectations of Romanians; a more recent example is the influence of smuggled South Korean drama series in North Korea. The importance of soft power is not lost on authoritarian regimes, fighting such influence with bans on foreign popular culture, control of the internet and unauthorised satellite dishes etc. Nor is such a usage of culture recent, as part of Roman imperialism local elites would be exposed to the benefits and luxuries of Roman culture and lifestyle, with the aim that they would then become willing participants. CANNOTANSWER

\begin{figure}[t] \small \begin{tcolorbox}[boxsep=0pt,left=5pt,right=0pt,top=2pt,colback = yellow!5] \begin{dialogue}
 \small 
 \speak{Student}{\bf When imperialism impacts social norms of a state, what is it called? }
\speak{Teacher}\colorbox{pink!25}{$\hookrightarrow$}
{ Cultural imperialism }
\\
\speak{Student}{\bf What is Cultural Imperialism often referred to as? }
\speak{Teacher}\colorbox{pink!25}{$\hookrightarrow$}
{ soft power }
\\
\speak{Student}{\bf Which American show changed the views of Romanians during the cold war? }
\speak{Teacher}\colorbox{pink!25}{$\hookrightarrow$}
{ Dallas }
\\
\speak{Student}{\bf Which historic empire used cultural imperialism to sway local elites? }
\speak{Teacher}\colorbox{pink!25}{$\hookrightarrow$}
{ Roman }
\\
\speak{Student}{\bf How do regimes fight against cultural imperialism? }
\speak{Teacher}\colorbox{pink!25}{$\hookrightarrow$}
{ bans }
\\
\speak{Student}{\bf When imperialism does not impact social norms of a state, what is it called? }
\speak{Teacher}\colorbox{pink!25}{$\hookrightarrow$}
{ CANNOTANSWER }
\\
\speak{Student}{\bf  What is Cultural Imperialism never referred to as? }
\speak{Teacher}\colorbox{pink!25}{$\hookrightarrow$}
{ CANNOTANSWER }
\\
\speak{Student}{\bf Which American did not show changed the views of Romanians during the cold war? }
\speak{Teacher}\colorbox{pink!25}{$\hookrightarrow$}
{ CANNOTANSWER }
 \end{dialogue}\end{tcolorbox}\end{figure}\begin{figure}[t] \small \begin{tcolorbox}[boxsep=0pt,left=5pt,right=0pt,top=2pt,colback = yellow!5] \begin{dialogue}
 \small 
 \speak{Student}{\bf  Which historic empire used cultural imperialism to sway non-local elites? }
\speak{Teacher}\colorbox{pink!25}{$\hookrightarrow$}
{ CANNOTANSWER }
\\
\speak{Student}{\bf  How do regimes not fight against cultural imperialism? }
\speak{Teacher}\colorbox{pink!25}{$\hookrightarrow$}
{ CANNOTANSWER }
\\
 \end{dialogue}\end{tcolorbox}\end{figure}

\end{document}

