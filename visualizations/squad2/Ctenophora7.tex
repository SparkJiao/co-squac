\documentclass[11pt,a4paper, onecolumn]{article}
\usepackage{times}
\usepackage{latexsym}
\usepackage{url}
\usepackage{textcomp}
\usepackage{bbm}
\usepackage{amsmath}
\usepackage{booktabs}
\usepackage{tabularx}
\usepackage{graphicx}
\usepackage{dialogue}
\usepackage{mathtools}
\usepackage{hyperref}
%\hypersetup{draft}

\usepackage{multirow}
\usepackage{mdframed}
\usepackage{tcolorbox}

\usepackage{xcolor,pifont}
%\newcommand{\cmark}{\ding{51}}
%\newcommand{\xmark}{\ding{55}}

\setcounter{topnumber}{2}
\setcounter{bottomnumber}{2}
\setcounter{totalnumber}{4}
\renewcommand{\topfraction}{0.75}
\renewcommand{\bottomfraction}{0.75}
\renewcommand{\textfraction}{0.05}
\renewcommand{\floatpagefraction}{0.6}

\newcommand\cmark {\textcolor{green}{\ding{52}}}
\newcommand\xmark {\textcolor{red}{\ding{55}}}
\mdfdefinestyle{dialogue}{
    backgroundcolor=yellow!20,
    innermargin=5pt
}
\usepackage{amssymb}
\usepackage{soul}
\makeatletter

\begin{document}

\hspace{15pt}{\textbf{Section}:Ctenophora7\\}
\\ Context: Ranging from about 1 millimeter (0.039 in) to 1.5 meters (4.9 ft) in size, ctenophores are the largest non-colonial animals that use cilia (''hairs'') as their main method of locomotion. Most species have eight strips, called comb rows, that run the length of their bodies and bear comb-like bands of cilia, called ''ctenes,'' stacked along the comb rows so that when the cilia beat, those of each comb touch the comb below. The name ''ctenophora'' means ''comb-bearing'', from the Greek κτείς (stem-form κτεν-) meaning ''comb'' and the Greek suffix -φορος meaning ''carrying''. CANNOTANSWER

\begin{figure}[t] \small \begin{tcolorbox}[boxsep=0pt,left=5pt,right=0pt,top=2pt,colback = yellow!5] \begin{dialogue}
 \small 
 \speak{Student}{\bf What are the hairs on ctenophores called? }
\speak{Teacher}\colorbox{pink!25}{$\hookrightarrow$}
{ cilia }
\\
\speak{Student}{\bf What are cilia used for? }
\speak{Teacher}\colorbox{pink!25}{$\hookrightarrow$}
{ method of locomotion }
\\
\speak{Student}{\bf Comb like bands of cilia are called what? }
\speak{Teacher}\colorbox{pink!25}{$\hookrightarrow$}
{ ctenes }
\\
\speak{Student}{\bf What does ctenophore mean in Greek? }
\speak{Teacher}\colorbox{pink!25}{$\hookrightarrow$}
{ comb-bearing }
\\
\speak{Student}{\bf How many types of cilia are there? }
\speak{Teacher}\colorbox{pink!25}{$\hookrightarrow$}
{ CANNOTANSWER }
\\
\speak{Student}{\bf How long can the cillia grow on ctenophores? }
\speak{Teacher}\colorbox{pink!25}{$\hookrightarrow$}
{ CANNOTANSWER }
\\
\speak{Student}{\bf What do cilia use their bodies for? }
\speak{Teacher}\colorbox{pink!25}{$\hookrightarrow$}
{ CANNOTANSWER }
\\
\speak{Student}{\bf What are the bodies of cilia also called? }
\speak{Teacher}\colorbox{pink!25}{$\hookrightarrow$}
{ CANNOTANSWER }
 \end{dialogue}\end{tcolorbox}\end{figure}\begin{figure}[t] \small \begin{tcolorbox}[boxsep=0pt,left=5pt,right=0pt,top=2pt,colback = yellow!5] \begin{dialogue}
 \small 
 \speak{Student}{\bf What kind of animals are cilia considered? }
\speak{Teacher}\colorbox{pink!25}{$\hookrightarrow$}
{ CANNOTANSWER }
\\
 \end{dialogue}\end{tcolorbox}\end{figure}

\end{document}

