\documentclass[11pt,a4paper, onecolumn]{article}
\usepackage{times}
\usepackage{latexsym}
\usepackage{url}
\usepackage{textcomp}
\usepackage{bbm}
\usepackage{amsmath}
\usepackage{booktabs}
\usepackage{tabularx}
\usepackage{graphicx}
\usepackage{dialogue}
\usepackage{mathtools}
\usepackage{hyperref}
%\hypersetup{draft}

\usepackage{multirow}
\usepackage{mdframed}
\usepackage{tcolorbox}

\usepackage{xcolor,pifont}
%\newcommand{\cmark}{\ding{51}}
%\newcommand{\xmark}{\ding{55}}

\setcounter{topnumber}{2}
\setcounter{bottomnumber}{2}
\setcounter{totalnumber}{4}
\renewcommand{\topfraction}{0.75}
\renewcommand{\bottomfraction}{0.75}
\renewcommand{\textfraction}{0.05}
\renewcommand{\floatpagefraction}{0.6}

\newcommand\cmark {\textcolor{green}{\ding{52}}}
\newcommand\xmark {\textcolor{red}{\ding{55}}}
\mdfdefinestyle{dialogue}{
    backgroundcolor=yellow!20,
    innermargin=5pt
}
\usepackage{amssymb}
\usepackage{soul}
\makeatletter

\begin{document}

\hspace{15pt}{\textbf{Section}:1973 oil crisis18\\}
\\ Context: The crisis reduced the demand for large cars. Japanese imports, primarily the Toyota Corona, the Toyota Corolla, the Datsun B210, the Datsun 510, the Honda Civic, the Mitsubishi Galant (a captive import from Chrysler sold as the Dodge Colt), the Subaru DL, and later the Honda Accord all had four cylinder engines that were more fuel efficient than the typical American V8 and six cylinder engines. Japanese imports became mass-market leaders with unibody construction and front-wheel drive, which became de facto standards. CANNOTANSWER

\begin{figure}[t] \small \begin{tcolorbox}[boxsep=0pt,left=5pt,right=0pt,top=2pt,colback = yellow!5] \begin{dialogue}
 \small 
 \speak{Student}{\bf Which sized cars were the least demanded cars in the crisis? }
\speak{Teacher}\colorbox{pink!25}{$\hookrightarrow$}
{ large cars }
\\
\speak{Student}{\bf Which country's cars became more highly sought after as they were more fuel efficient? }
\speak{Teacher}\colorbox{pink!25}{$\hookrightarrow$}
{ Japanese imports }
\\
\speak{Student}{\bf What type of engines does the American car typically have? }
\speak{Teacher}\colorbox{pink!25}{$\hookrightarrow$}
{ V8 and six cylinder engines }
\\
\speak{Student}{\bf Which country's imports became the de facto mass market leaders? }
\speak{Teacher}\colorbox{pink!25}{$\hookrightarrow$}
{ Japan }
\\
\speak{Student}{\bf What standards did American cars create in the auto industry? }
\speak{Teacher}\colorbox{pink!25}{$\hookrightarrow$}
{ CANNOTANSWER }
\\
\speak{Student}{\bf What are two cars with V8 engines that were more fuel efficient? }
\speak{Teacher}\colorbox{pink!25}{$\hookrightarrow$}
{ CANNOTANSWER }
\\
\speak{Student}{\bf What type of unibody construction does an American car usually have? }
\speak{Teacher}\colorbox{pink!25}{$\hookrightarrow$}
{ CANNOTANSWER }
\\
\speak{Student}{\bf What country became a leader in importing large cars? }
\speak{Teacher}\colorbox{pink!25}{$\hookrightarrow$}
{ CANNOTANSWER }
 \end{dialogue}\end{tcolorbox}\end{figure}\begin{figure}[t] \small \begin{tcolorbox}[boxsep=0pt,left=5pt,right=0pt,top=2pt,colback = yellow!5] \begin{dialogue}
 \small 
 \speak{Student}{\bf What increased demand for cars with six cylinder engines? }
\speak{Teacher}\colorbox{pink!25}{$\hookrightarrow$}
{ CANNOTANSWER }
\\
 \end{dialogue}\end{tcolorbox}\end{figure}

\end{document}

