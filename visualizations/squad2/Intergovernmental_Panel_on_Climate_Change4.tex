\documentclass[11pt,a4paper, onecolumn]{article}
\usepackage{times}
\usepackage{latexsym}
\usepackage{url}
\usepackage{textcomp}
\usepackage{bbm}
\usepackage{amsmath}
\usepackage{booktabs}
\usepackage{tabularx}
\usepackage{graphicx}
\usepackage{dialogue}
\usepackage{mathtools}
\usepackage{hyperref}
%\hypersetup{draft}

\usepackage{multirow}
\usepackage{mdframed}
\usepackage{tcolorbox}

\usepackage{xcolor,pifont}
%\newcommand{\cmark}{\ding{51}}
%\newcommand{\xmark}{\ding{55}}

\setcounter{topnumber}{2}
\setcounter{bottomnumber}{2}
\setcounter{totalnumber}{4}
\renewcommand{\topfraction}{0.75}
\renewcommand{\bottomfraction}{0.75}
\renewcommand{\textfraction}{0.05}
\renewcommand{\floatpagefraction}{0.6}

\newcommand\cmark {\textcolor{green}{\ding{52}}}
\newcommand\xmark {\textcolor{red}{\ding{55}}}
\mdfdefinestyle{dialogue}{
    backgroundcolor=yellow!20,
    innermargin=5pt
}
\usepackage{amssymb}
\usepackage{soul}
\makeatletter

\begin{document}

\hspace{15pt}{\textbf{Section}:Intergovernmental Panel on Climate Change4\\}
\\ Context: The IPCC does not carry out research nor does it monitor climate related data. Lead authors of IPCC reports assess the available information about climate change based on published sources. According to IPCC guidelines, authors should give priority to peer-reviewed sources. Authors may refer to non-peer-reviewed sources (the ''grey literature''), provided that they are of sufficient quality. Examples of non-peer-reviewed sources include model results, reports from government agencies and non-governmental organizations, and industry journals. Each subsequent IPCC report notes areas where the science has improved since the previous report and also notes areas where further research is required. CANNOTANSWER

\begin{figure}[t] \small \begin{tcolorbox}[boxsep=0pt,left=5pt,right=0pt,top=2pt,colback = yellow!5] \begin{dialogue}
 \small 
 \speak{Student}{\bf What does the IPCC not do? }
\speak{Teacher}\colorbox{pink!25}{$\hookrightarrow$}
{ does not carry out research nor does it monitor climate related data }
\\
\speak{Student}{\bf Where do IPCC reports get their information? }
\speak{Teacher}\colorbox{pink!25}{$\hookrightarrow$}
{ available information about climate change based on published sources }
\\
\speak{Student}{\bf What is 'grey literature'? }
\speak{Teacher}\colorbox{pink!25}{$\hookrightarrow$}
{ non-peer-reviewed sources }
\\
\speak{Student}{\bf What kind of non-peer-reviewed sources does the IPCC use? }
\speak{Teacher}\colorbox{pink!25}{$\hookrightarrow$}
{ model results, reports from government agencies and non-governmental organizations, and industry journals }
\\
\speak{Student}{\bf Who is responsible for monitoring climate data? }
\speak{Teacher}\colorbox{pink!25}{$\hookrightarrow$}
{ CANNOTANSWER }
\\
\speak{Student}{\bf What organization does climate related research? }
\speak{Teacher}\colorbox{pink!25}{$\hookrightarrow$}
{ CANNOTANSWER }
\\
\speak{Student}{\bf What guidelines are used for IPCC reviews? }
\speak{Teacher}\colorbox{pink!25}{$\hookrightarrow$}
{ CANNOTANSWER }
\\
\speak{Student}{\bf What must all research be according to the IPCC guidelines? }
\speak{Teacher}\colorbox{pink!25}{$\hookrightarrow$}
{ CANNOTANSWER }
 \end{dialogue}\end{tcolorbox}\end{figure}\begin{figure}[t] \small \begin{tcolorbox}[boxsep=0pt,left=5pt,right=0pt,top=2pt,colback = yellow!5] \begin{dialogue}
 \small 
 \speak{Student}{\bf What is an example of a peer-reviewed resource? }
\speak{Teacher}\colorbox{pink!25}{$\hookrightarrow$}
{ CANNOTANSWER }
\\
 \end{dialogue}\end{tcolorbox}\end{figure}

\end{document}

