\documentclass[11pt,a4paper, onecolumn]{article}
\usepackage{times}
\usepackage{latexsym}
\usepackage{url}
\usepackage{textcomp}
\usepackage{bbm}
\usepackage{amsmath}
\usepackage{booktabs}
\usepackage{tabularx}
\usepackage{graphicx}
\usepackage{dialogue}
\usepackage{mathtools}
\usepackage{hyperref}
%\hypersetup{draft}

\usepackage{multirow}
\usepackage{mdframed}
\usepackage{tcolorbox}

\usepackage{xcolor,pifont}
%\newcommand{\cmark}{\ding{51}}
%\newcommand{\xmark}{\ding{55}}

\setcounter{topnumber}{2}
\setcounter{bottomnumber}{2}
\setcounter{totalnumber}{4}
\renewcommand{\topfraction}{0.75}
\renewcommand{\bottomfraction}{0.75}
\renewcommand{\textfraction}{0.05}
\renewcommand{\floatpagefraction}{0.6}

\newcommand\cmark {\textcolor{green}{\ding{52}}}
\newcommand\xmark {\textcolor{red}{\ding{55}}}
\mdfdefinestyle{dialogue}{
    backgroundcolor=yellow!20,
    innermargin=5pt
}
\usepackage{amssymb}
\usepackage{soul}
\makeatletter

\begin{document}

\hspace{15pt}{\textbf{Section}:Oxygen2\\}
\\ Context: In the late 17th century, Robert Boyle proved that air is necessary for combustion. English chemist John Mayow (1641–1679) refined this work by showing that fire requires only a part of air that he called spiritus nitroaereus or just nitroaereus. In one experiment he found that placing either a mouse or a lit candle in a closed container over water caused the water to rise and replace one-fourteenth of the air's volume before extinguishing the subjects. From this he surmised that nitroaereus is consumed in both respiration and combustion. CANNOTANSWER

\begin{figure}[t] \small \begin{tcolorbox}[boxsep=0pt,left=5pt,right=0pt,top=2pt,colback = yellow!5] \begin{dialogue}
 \small 
 \speak{Student}{\bf Who proved that air is necessary for combustion? }
\speak{Teacher}\colorbox{pink!25}{$\hookrightarrow$}
{ Robert Boyle }
\\
\speak{Student}{\bf What English chemist showed that fire only needed nitoaereus? }
\speak{Teacher}\colorbox{pink!25}{$\hookrightarrow$}
{ John Mayow }
\\
\speak{Student}{\bf What is consumed in both combustion and respiration? }
\speak{Teacher}\colorbox{pink!25}{$\hookrightarrow$}
{ nitroaereus }
\\
\speak{Student}{\bf John Mayow died in what year? }
\speak{Teacher}\colorbox{pink!25}{$\hookrightarrow$}
{ 1679 }
\\
\speak{Student}{\bf What researcher showed that air is a necessity for combustion? }
\speak{Teacher}\colorbox{pink!25}{$\hookrightarrow$}
{ Robert Boyle }
\\
\speak{Student}{\bf What did John Mayow  name the part of air that caused combustion? }
\speak{Teacher}\colorbox{pink!25}{$\hookrightarrow$}
{ nitroaereus }
\\
\speak{Student}{\bf In what century did Mayow and Boyle perform their experiments? }
\speak{Teacher}\colorbox{pink!25}{$\hookrightarrow$}
{ 17th century }
\\
\speak{Student}{\bf Besides combustion, for what other action did Mayow show nitroaereus responsible? }
\speak{Teacher}\colorbox{pink!25}{$\hookrightarrow$}
{ respiration }
 \end{dialogue}\end{tcolorbox}\end{figure}\begin{figure}[t] \small \begin{tcolorbox}[boxsep=0pt,left=5pt,right=0pt,top=2pt,colback = yellow!5] \begin{dialogue}
 \small 
 \speak{Student}{\bf What chemist showed that fire needed only a part of air? }
\speak{Teacher}\colorbox{pink!25}{$\hookrightarrow$}
{ John Mayow }
\\
\speak{Student}{\bf What did John Mayow prove that air is necessary for? }
\speak{Teacher}\colorbox{pink!25}{$\hookrightarrow$}
{ CANNOTANSWER }
\\
\speak{Student}{\bf What years was chemist John Boyle alive? }
\speak{Teacher}\colorbox{pink!25}{$\hookrightarrow$}
{ CANNOTANSWER }
\\
\speak{Student}{\bf When did Robert Mayow prove his theories? }
\speak{Teacher}\colorbox{pink!25}{$\hookrightarrow$}
{ CANNOTANSWER }
\\
\speak{Student}{\bf Who refined Robert Mayow's work? }
\speak{Teacher}\colorbox{pink!25}{$\hookrightarrow$}
{ CANNOTANSWER }
\\
\speak{Student}{\bf What was the profession of John Boyle? }
\speak{Teacher}\colorbox{pink!25}{$\hookrightarrow$}
{ CANNOTANSWER }
\\
 \end{dialogue}\end{tcolorbox}\end{figure}

\end{document}

