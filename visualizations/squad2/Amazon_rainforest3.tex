\documentclass[11pt,a4paper, onecolumn]{article}
\usepackage{times}
\usepackage{latexsym}
\usepackage{url}
\usepackage{textcomp}
\usepackage{bbm}
\usepackage{amsmath}
\usepackage{booktabs}
\usepackage{tabularx}
\usepackage{graphicx}
\usepackage{dialogue}
\usepackage{mathtools}
\usepackage{hyperref}
%\hypersetup{draft}

\usepackage{multirow}
\usepackage{mdframed}
\usepackage{tcolorbox}

\usepackage{xcolor,pifont}
%\newcommand{\cmark}{\ding{51}}
%\newcommand{\xmark}{\ding{55}}

\setcounter{topnumber}{2}
\setcounter{bottomnumber}{2}
\setcounter{totalnumber}{4}
\renewcommand{\topfraction}{0.75}
\renewcommand{\bottomfraction}{0.75}
\renewcommand{\textfraction}{0.05}
\renewcommand{\floatpagefraction}{0.6}

\newcommand\cmark {\textcolor{green}{\ding{52}}}
\newcommand\xmark {\textcolor{red}{\ding{55}}}
\mdfdefinestyle{dialogue}{
    backgroundcolor=yellow!20,
    innermargin=5pt
}
\usepackage{amssymb}
\usepackage{soul}
\makeatletter

\begin{document}

\hspace{15pt}{\textbf{Section}:Amazon rainforest3\\}
\\ Context: There is evidence that there have been significant changes in Amazon rainforest vegetation over the last 21,000 years through the Last Glacial Maximum (LGM) and subsequent deglaciation. Analyses of sediment deposits from Amazon basin paleolakes and from the Amazon Fan indicate that rainfall in the basin during the LGM was lower than for the present, and this was almost certainly associated with reduced moist tropical vegetation cover in the basin. There is debate, however, over how extensive this reduction was. Some scientists argue that the rainforest was reduced to small, isolated refugia separated by open forest and grassland; other scientists argue that the rainforest remained largely intact but extended less far to the north, south, and east than is seen today. This debate has proved difficult to resolve because the practical limitations of working in the rainforest mean that data sampling is biased away from the center of the Amazon basin, and both explanations are reasonably well supported by the available data. CANNOTANSWER

\begin{figure}[t] \small \begin{tcolorbox}[boxsep=0pt,left=5pt,right=0pt,top=2pt,colback = yellow!5] \begin{dialogue}
 \small 
 \speak{Student}{\bf What does LGM stands for? }
\speak{Teacher}\colorbox{pink!25}{$\hookrightarrow$}
{ Last Glacial Maximum }
\\
\speak{Student}{\bf What did the analysis from the sediment deposits indicate?  }
\speak{Teacher}\colorbox{pink!25}{$\hookrightarrow$}
{ rainfall in the basin during the LGM was lower than for the present }
\\
\speak{Student}{\bf What are some of scientists arguments?  }
\speak{Teacher}\colorbox{pink!25}{$\hookrightarrow$}
{ the rainforest was reduced to small, isolated refugia separated by open forest and grassland }
\\
\speak{Student}{\bf How has this debate been proven? }
\speak{Teacher}\colorbox{pink!25}{$\hookrightarrow$}
{ This debate has proved difficult }
\\
\speak{Student}{\bf How are the explanations supported? }
\speak{Teacher}\colorbox{pink!25}{$\hookrightarrow$}
{ explanations are reasonably well supported }
\\
\speak{Student}{\bf There have been major changes in Amazon rainforest vegetation over the last how many years? }
\speak{Teacher}\colorbox{pink!25}{$\hookrightarrow$}
{ 21,000 }
\\
\speak{Student}{\bf What caused changes in the Amazon rainforest vegetation? }
\speak{Teacher}\colorbox{pink!25}{$\hookrightarrow$}
{ the Last Glacial Maximum (LGM) and subsequent deglaciation }
\\
\speak{Student}{\bf What has been analyzed to compare Amazon rainfall in the past and present? }
\speak{Teacher}\colorbox{pink!25}{$\hookrightarrow$}
{ sediment deposits }
 \end{dialogue}\end{tcolorbox}\end{figure}\begin{figure}[t] \small \begin{tcolorbox}[boxsep=0pt,left=5pt,right=0pt,top=2pt,colback = yellow!5] \begin{dialogue}
 \small 
 \speak{Student}{\bf What has the lower rainfall in the Amazon during the LGM been attributed to? }
\speak{Teacher}\colorbox{pink!25}{$\hookrightarrow$}
{ reduced moist tropical vegetation cover in the basin }
\\
\speak{Student}{\bf Many changes in the vegetation of the amazon rainforest took place since the  Last Glacial Maximum, which was how many years ago? }
\speak{Teacher}\colorbox{pink!25}{$\hookrightarrow$}
{ 21,000 }
\\
\speak{Student}{\bf Analysis of what kind of deposits from the Amazon Fan indicates a change in rainfall in the Amazon basin? }
\speak{Teacher}\colorbox{pink!25}{$\hookrightarrow$}
{ sediment deposits }
\\
\speak{Student}{\bf Changes in rainfall reduced what kind of vegetation cover in the Amazon basin? }
\speak{Teacher}\colorbox{pink!25}{$\hookrightarrow$}
{ moist tropical vegetation cover }
\\
\speak{Student}{\bf Scientists disagree with how the Amazon rainforest changed over time with some arguing that it was reduced to isolated refugia seperated by what? }
\speak{Teacher}\colorbox{pink!25}{$\hookrightarrow$}
{ open forest and grassland }
\\
\speak{Student}{\bf Why is it difficult to resolve disagreements about the changes in the Amazon rainforest? }
\speak{Teacher}\colorbox{pink!25}{$\hookrightarrow$}
{ data sampling is biased away from the center of the Amazon basin }
\\
\speak{Student}{\bf There have been insignificant changes in the Amazon rain forest vegetation through the last what }
\speak{Teacher}\colorbox{pink!25}{$\hookrightarrow$}
{ CANNOTANSWER }
\\
\speak{Student}{\bf What was higher during the LGM than the present? }
\speak{Teacher}\colorbox{pink!25}{$\hookrightarrow$}
{ CANNOTANSWER }
 \end{dialogue}\end{tcolorbox}\end{figure}\begin{figure}[t] \small \begin{tcolorbox}[boxsep=0pt,left=5pt,right=0pt,top=2pt,colback = yellow!5] \begin{dialogue}
 \small 
 \speak{Student}{\bf What has been easily proven about the rain forest }
\speak{Teacher}\colorbox{pink!25}{$\hookrightarrow$}
{ CANNOTANSWER }
\\
\speak{Student}{\bf Data sampling strongly supports that what remained largely intact?? }
\speak{Teacher}\colorbox{pink!25}{$\hookrightarrow$}
{ CANNOTANSWER }
\\
 \end{dialogue}\end{tcolorbox}\end{figure}

\end{document}

